%%  SDV REFMAN BEGINN %%
\section{Referenz}

%%%%%%%%%%%%%%%%%%%%%%%%%%%%%%%%%%%%%%%%%%%%%%%%%%%%%%%%%%%%%%%
\subsection{Package}

%%%%%%%%%%%%%%%%%%
\paragraph{Module}

\begin{itemize}
\item \wancicode{sdv.py} enth\"alt die Klasse \wancicode{SDV}, die
  Hauptklasse von SDV.
  
\item \wancicode{error.py} enth\"alt die Klasse \wancicode{Error}, die
  zu abgefangenen Fehlern entsprechende Fehlermeldungen in graphischer
  Form ausgibt.
  
\item \wancicode{lib.py} enth\"alt zus\"atzliche Funktionen.
  
\item \wancicode{data.py} enth\"alt die Klasse \wancicode{Data}, die
  Eingabedatenobjekte erzeugt, die Klasse \wancicode{File}, die
  Dateiobjekte f\"ur Bilddateien erzeugt und die Klasse
  \wancicode{Size}, die 2-Tupel-Objekte f\"ur Bildgr\"o"senangaben
  erzeugt.
  
\item \wancicode{colors.py} enth\"alt die Klassen \wancicode{Color}
  und \wancicode{Colorlist}, die Farb-, bzw. Farblistenobjekte
  erzeugen.
  
\item \wancicode{file.py} enth\"alt die Klassen \wancicode{Font} und
  \wancicode{File}, die jeweils Fontdatei- (\wancicode{.pil}) und
  Bilddateiobjekte (\wancicode{.jpg}, \wancicode{.png} usw.) erzeugt.
  
\item \wancicode{title.py} enth\"alt die Klasse \wancicode{Title}, die
  ein Titelfeldobjekt erzeugt.
  
\item \wancicode{legend.py} enth\"alt die Klasse \wancicode{Legend},
  die ein Legendenfeldobjekt erzeugt.
  
\item \wancicode{graphic.py} enth\"alt die Klasse \wancicode{Graphic},
  die ein Grafikfeldobjekt erzeugt.
\end{itemize}

%%%%%%%%%%%%%%%%%%%%%%%%%
\paragraph{Dokumentation}

\begin{itemize}
\item Die Plain-Text-Datei \wancicode{read.me} enth\"alt allgemeine
Informationen zum \textit{Package} und entspricht in etwa
diesem Abschnitt.

\item \wancicode{techdoc.tex}, also dieser Text, stellt die vollst\"andige 
technische Dokumentation zu \wancicode{dmerce}-SDV dar. 

\item \wancicode{userdoc.tex} ist eine Benutzerreferenz, die anhand
von Beispielaufrufen die Benutzung von SDV illustriert.
\end{itemize}

%%%%%%%%%%%%%%%%%%%%%%%%%%%%%%%%%%%%%%%%%%%%%%%%%%%%%%%%%%%%%%%%%%%%%
\subsection{Klassen}
Systematische Darstellung der Klassen, ihrer Attribute und Methoden,
dabei wird wie folgt vorgegangen:
\begin{itemize}
\item \textbf{Modul} gibt an, in welchem Modul des Packages sich die
  Klassendeklaration befindet.
\item \textbf{Parameter} gibt die Variablen an, die vom Benutzer der
  Klasse mitgegeben werden k\"onnen und auf mit welchen Methoden dies
  geschehen kann. Dazu ist zu sagen, da"s SDV grunds\"atzlich keine
  automatischen \wancicode{\_\_init\_\_()}-Methoden verwendet, sondern
  nur eigene \wancicode{Init()}-Methoden zur Anwendung kommen.
\item \textbf{Prim\"are Attribute} beschreibt die Klassenattribute,
  die direkt aus den Parametern gewonnen werden. In den meisten
  F\"allen entsprechen sie den Parametern genau.
\item \textbf{Sekund\"are Attribute} beschreibt diejenigen
  Klassenattribute, die nachtr\"aglich durch Methoden erzeugt werden,
  ohne da"s daf\"ur Parameter angegeben wurden. Der Zusammenhang mit
  Parametern ist allenfalls indirekt.
\item \textbf{Methoden} beschreibt die Methoden der jeweiligen Klasse.
  Dazu z\"ahlen neben anderen in den meisten F\"allen \wancicode{Set}-
  und\wancicode{Get}-Methoden, die also dazu dienen, Attribute zu
  setzen, bzw. auszugeben, eine \wancicode{Init}-Methode, die dazu
  geeignet ist, einen ganzen Satz an Parametern anzugeben,
  gelegentlich auch eine \wancicode{SetUp}-Methode, die zum Festlegen
  von Attributen genutzt wird.
\item \textbf{Objektfunktionalit\"at} erg\"anzt die Information um den
  funktionalen Zusammenhang der Attribute und Methoden nachdem ein
  Objekt erzeugt wurde.
\end{itemize}


%%%%%%%%%%%%%%%%%%%%%%%%%%%%%%%%%%%%%%%%%%%%%%%%%%%%%%%%%%%%%%
% SDV
%%%%%%%%%%%%%%%%%%%%%%%%%%%%%%%%%%%%%%%%%%%%%%%%%%%%%%%%%%%%%%

\paragraph{\wancicode{SDV}}

\begin{description}
\item[Modul] \wancicode{sdv.py}

%%%%%%%% Parameter %%%%%%%%%%%%%%%%%%%%%%%%%%%%%
  
\item[Parameter] Parameter sind die Werte, die der Klasse vom Benutzer
  mitgegeben werden

\begin{itemize}
\item \wancicode{filename} (obligatorisch)
\begin{enumerate}
\item[\textit{Methods}] \wancicode{Init(..., filename =
    <foo.bar.png>,...)}  oder \wancicode{SetFilename(<foo.bar.png>)}
\item[\textit{Valids}] Alle g\"ultigen Dateinamen mit Formaten:
  \wancicode{.jpg}, \wancicode{.gif}, \wancicode{.png} oder
  \wancicode{.bmp}
\item[\textit{Default}] \wancicode{<foo.bar.png>}
\item[\textit{Description}] Dateiname der fertigen Graphik
  einschlie"slich des Dateiformats.
\item[\textit{Attribute}] \wancicode{self.\_\_filename},
  \wancicode{self.\_\_fileformat}
\end{enumerate}

\item \wancicode{size} (optional)
\begin{enumerate}
\item[\textit{Methods}] \wancicode{Init(..., size = (<n>, <m>),...)}
  oder \wancicode{SetSize(<n>, <m>)}
\item[\textit{Valids}] ein 2-\textit{Tupel}, dessen Elemente positive
  \textit{Integer} (also $n, m > 0$) sind. Unabh\"angig von SDV
  sollten ggf. Maxima der Bilddarstellung beachtet werden.
\item[\textit{Default}] \wancicode{(800,600)}
\item[\textit{Description}] beschreibt die Gr\"o"se (in Pixeln) der
  resultierenden Gesamtgraphik
\item[\textit{Attribute}] \wancicode{self.\_\_size}
\end{enumerate}

\item \wancicode{topmargin} (optional)
\begin{enumerate}
\item[\textit{Methods}] \wancicode{Init(..., topmargin = <n>,...)}
  oder \wancicode{SetTopmargin(<n>)}
\item[\textit{Valids}] Jede positive \textit{Integer}. Au"serdem
  sollte \wancicode{topmargin} $+$ \wancicode{bottommargin} $<$
  \wancicode{size[1]} sein.
\item[\textit{Default}] \wancicode{30}
\item[\textit{Description}] Gibt die Breite des oberen Randes der
  Graphik bis zum oberen Rand der obersten Bildelemente an
\item[\textit{Attribute}] \wancicode{self.\_\_topmargin}
\end{enumerate}

\item \wancicode{bottommargin} (optional)
\begin{enumerate}
\item[\textit{Methods}] \wancicode{Init(..., bottommargin = <n>,...)}
  oder \wancicode{SetBottommargin(<n>)}
\item[\textit{Valids}] Jede positive \textit{Integer}. Au"serdem
  sollte \wancicode{topmargin} $+$ \wancicode{bottommargin} $<$
  \wancicode{size[1]} sein.
\item[\textit{Default}] \wancicode{30}
\item[\textit{Description}] Gibt die Breite des unteren Randes der
  Graphik bis zum unteren Rand der untersten Bildelemente an
\item[\textit{Attribute}] \wancicode{self.\_\_bottommargin}
\end{enumerate}

\item \wancicode{leftmargin} (optional)
\begin{enumerate}
\item[\textit{Methods}] \wancicode{Init(..., leftmargin = <n>,...)}
  oder \wancicode{SetLeftmargin(<n>)}
\item[\textit{Valids}] Jede positive \textit{Integer}. Au"serdem
  sollte \wancicode{leftmargin} $+$ \wancicode{rightmargin} $<$
  \wancicode{size[1]} sein.
\item[\textit{Default}] \wancicode{30}
\item[\textit{Description}] Gibt die Breite des linken Randes der
  Graphik bis zum linken Rand derjenigen Bildelemente an, die sich am
  weitesten links befinden.
\item[\textit{Attribute}] \wancicode{self.\_\_leftmargin}
\end{enumerate}

\item \wancicode{rightmargin} (optional)
\begin{enumerate}
\item[\textit{Methods}] \wancicode{Init(..., rightmargin = <n>,...)}
  oder \wancicode{SetRightmargin(<n>)}
\item[\textit{Valids}] Jede positive \textit{Integer}. Au"serdem
  sollte \wancicode{leftmargin} $+$ \wancicode{rightmargin} $<$
  \wancicode{size[1]} sein.
\item[\textit{Default}] \wancicode{30}
\item[\textit{Description}] Gibt die Breite des rechten Randes der
  Graphik bis zum rechten Rand derjenigen Bildelemente an, die sich am
  weitesten rechts befinden.
\item[\textit{Attribute}] \wancicode{self.\_\_rightmargin}
\end{enumerate}

\item \wancicode{dist} (optional)
\begin{enumerate}
\item[\textit{Methods}] \wancicode{Init(..., dist = <n>,...)}  oder
  \wancicode{SetDist(<n>)}
\item[\textit{Valids}] Jede positive \textit{Integer}.
\item[\textit{Default}] \wancicode{10}
\item[\textit{Description}] Gibt den Abstand der einzelnen
  Bildelemente zueinander an.
\item[\textit{Attribute}] \wancicode{self.\_\_dist}
\end{enumerate}

\item \wancicode{colormode} (optional)
\begin{enumerate}
\item[\textit{Methods}] \wancicode{Init(..., colormode =
    <``mode''>,...)}  oder \wancicode{SetColormode(<``mode''>)}
\item[\textit{Valids}] \wancicode{``RGB''}
\item[\textit{Default}] \wancicode{``RGB''}
\item[\textit{Description}] Legt den Farbmodus der Graphik fest.
  (Z.Zt. ist nur der RGB-Modus vorgesehen)
\item[\textit{Attribute}] \wancicode{self.\_\_colormode}
\end{enumerate}

\item \wancicode{color} (optional)
\begin{enumerate}
\item[\textit{Methods}] \wancicode{Init(..., color = <color>,...)}
  oder \wancicode{SetColor(<color>)}
\item[\textit{Valids}] Jede zul\"assige RGB-Farbcodierung
\item[\textit{Default}] \wancicode{(255,255,255)} (wei"s)
\item[\textit{Description}] Legt die Hintergrundfarbe der Grundgraphik
  fest.
\item[\textit{Attribute}] \wancicode{self.\_\_color}
\end{enumerate}

\item \wancicode{wallpaper} (optional)
\begin{enumerate}
\item[\textit{Methods}] \wancicode{SetWallpaper(<'wallpaper.foo'>)}
\item[\textit{Valids}] Jeder zul\"assige Dateiname einer Graphikdatei
  mit denselben Formaten von \wancicode{fileformat}.  NB!
  \wancicode{size} mu"s dieselbe Gr\"o"se haben, wie das Wallpaper!
\item[\textit{Default}] \wancicode{''}
\item[\textit{Description}] Legt den Dateinamen des Hintergrundbildes
  fest.
\item[\textit{Attribute}] \wancicode{self.\_\_wallpaper}
\end{enumerate}

\end{itemize}

%%%%%%%% Primaere Attribute %%%%%%%%%%%%%%%%%%%%%%%%%%%%%%%%%%

\newpage

\item[Prim\"are Attribute] Prim\"are Attribute sind Attribute, die
  unver\"andert aus Parametern \"ubernommen werden.

\begin{itemize}
  
\item \wancicode{self.\_\_filename}
\begin{enumerate}
\item[\textit{Methods}] \wancicode{Init(..., filename =
    <foo.bar.png>,...)}  oder \wancicode{SetFilename(<foo.bar.png>)}
\item[\textit{Valids}] Alle g\"ultigen Dateinamen mit Formaten:
  \wancicode{.jpg}, \wancicode{.gif}, \wancicode{.png} oder
  \wancicode{.bmp}
\item[\textit{Default}] \wancicode{<foo.bar.png>}
\item[\textit{Description}] Dateiname der fertigen Graphik
  einschlie"slich des Dateiformats.
\item[\textit{Parametre}] \wancicode{filename},
\end{enumerate}

\item \wancicode{self.\_\_size}
\begin{enumerate}
\item[\textit{Methods}] \wancicode{Init(..., size = (<n>, <m>),...)}
  oder \wancicode{SetSize(<n>, <m>)}
\item[\textit{Valids}] ein 2-\textit{Tupel}, dessen Elemente positive
  \textit{Integer} (also $n, m > 0$) sind. Unabh\"angig von SDV
  sollten ggf. Maxima der Bilddarstellung beachtet werden.
\item[\textit{Default}] \wancicode{(800,600)}
\item[\textit{Description}] beschreibt die Gr\"o"se (in Pixeln) der
  resultierenden Gesamtgraphik
\item[\textit{Parametre}] \wancicode{size}
\end{enumerate}

\item \wancicode{self.\_\_topmargin}
\begin{enumerate}
\item[\textit{Methods}] \wancicode{Init(..., topmargin = <n>,...)}
  oder \wancicode{SetTopmargin(<n>)}
\item[\textit{Valids}] Jede positive \textit{Integer}. Au"serdem
  sollte \wancicode{topmargin} $+$ \wancicode{bottommargin} $<$
  \wancicode{size[1]} sein.
\item[\textit{Default}] \wancicode{30}
\item[\textit{Description}] Gibt die Breite des oberen Randes der
  Graphik bis zum oberen Rand der obersten Bildelemente an
\item[\textit{Parametre}] \wancicode{topmargin}
\end{enumerate}

\item \wancicode{self.\_\_bottommargin}
\begin{enumerate}
\item[\textit{Methods}] \wancicode{Init(..., bottommargin = <n>,...)}
  oder \wancicode{SetBottommargin(<n>)}
\item[\textit{Valids}] Jede positive \textit{Integer}. Au"serdem
  sollte \wancicode{topmargin} $+$ \wancicode{bottommargin} $<$
  \wancicode{size[1]} sein.
\item[\textit{Default}] \wancicode{30}
\item[\textit{Description}] Gibt die Breite des unteren Randes der
  Graphik bis zum unteren Rand der untersten Bildelemente an
\item[\textit{Parametre}] \wancicode{bottommargin}
\end{enumerate}

\item \wancicode{self.\_\_leftmargin}
\begin{enumerate}
\item[\textit{Methods}] \wancicode{Init(..., leftmargin = <n>,...)}
  oder \wancicode{SetLeftmargin(<n>)}
\item[\textit{Valids}] Jede positive \textit{Integer}. Au"serdem
  sollte \wancicode{leftmargin} $+$ \wancicode{rightmargin} $<$
  \wancicode{size[1]} sein.
\item[\textit{Default}] \wancicode{30}
\item[\textit{Description}] Gibt die Breite des linken Randes der
  Graphik bis zum linken Rand derjenigen Bildelemente an, die sich am
  weitesten links befinden.
\item[\textit{Parametre}] \wancicode{leftmargin}
\end{enumerate}

\item \wancicode{self.\_\_rightmargin}
\begin{enumerate}
\item[\textit{Methods}] \wancicode{Init(..., rightmargin = <n>,...)}
  oder \wancicode{SetRightmargin(<n>)}
\item[\textit{Valids}] Jede positive \textit{Integer}. Au"serdem
  sollte \wancicode{leftmargin} $+$ \wancicode{rightmargin} $<$
  \wancicode{size[1]} sein.
\item[\textit{Default}] \wancicode{30}
\item[\textit{Description}] Gibt die Breite des rechten Randes der
  Graphik bis zum rechten Rand derjenigen Bildelemente an, die sich am
  weitesten rechts befinden.
\item[\textit{Parametre}] \wancicode{rightmargin}
\end{enumerate}

\item \wancicode{self.\_\_dist}
\begin{enumerate}
\item[\textit{Methods}] \wancicode{Init(..., dist = <n>,...)}  oder
  \wancicode{SetDist(<n>)}
\item[\textit{Valids}] Jede positive \textit{Integer}.
\item[\textit{Default}] \wancicode{10}
\item[\textit{Description}] Gibt den Abstand der einzelnen
  Bildelementen zueinander an.
\item[\textit{Parametre}] \wancicode{dist}
\end{enumerate}

\item \wancicode{self.\_\_colormode}
\begin{enumerate}
\item[\textit{Methods}] \wancicode{Init(..., colormode =
    <``mode''>,...)}  oder \wancicode{SetColormode(<``mode''>)}
\item[\textit{Valids}] \wancicode{``RGB''}
\item[\textit{Default}] \wancicode{``RGB''}
\item[\textit{Description}] Legt den Farbmodus der Graphik fest.
  (Z.Zt. ist nur der RGB-Modus vorgesehen)
\item[\textit{Parametre}] \wancicode{colormode}
\end{enumerate}

\item \wancicode{self.\_\_color}
\begin{enumerate}
\item[\textit{Methods}] \wancicode{Init(..., color = <color>,...)}
  oder \wancicode{SetColor(<color>)}
\item[\textit{Valids}] Jede zul\"assige RGB-Farbcodierung
\item[\textit{Default}] \wancicode{(255,255,255)} (wei"s)
\item[\textit{Description}] Legt die Hintergrundfarbe der Grundgraphik
  fest.
\item[\textit{Parametre}] \wancicode{color}
\end{enumerate}

\item \wancicode{self.\_\_wallpaper}
\begin{enumerate}
\item[\textit{Methods}] \wancicode{SetWallpaper(<'wallpaper.foo'>)}
\item[\textit{Valids}] Jeder zul\"assige Dateiname einer Graphikdatei
  mit denselben Formaten von \wancicode{fileformat}.  NB!
  \wancicode{size} mu"s dieselbe Gr\"o"se haben, wie das Wallpaper!
\item[\textit{Default}] \wancicode{''}
\item[\textit{Description}] Legt den Dateinamen des Hintergrundbildes
  fest.
\item[\textit{Parametre}] \wancicode{wallpaper}
\end{enumerate}

\end{itemize}

%%%%%%%% Sekundaere Attribute %%%%%%%%%%%%%%%%%%%%%%%%%%%%%

\newpage

\item[Sekund\"are Attribute] Sekund\"are Attribute werden im Gegensatz
  zu pri\"aren durch eine mehr oder weniger komplizierte Funktion,
  bzw. Methode ermittelt, ohne vorher an die Klasse \"ubergeben worden
  zu sein.


\begin{itemize}
\item \wancicode{self.\_\_fileformat}
\begin{enumerate}
\item[\textit{Methods}] \wancicode{Init(..., filename =
    <foo.bar.png>,...)}  oder \wancicode{SetFilename(<foo.bar.png>)}
\item[\textit{Valids}] \wancicode{.jpg}, \wancicode{.gif},
  \wancicode{.png} oder \wancicode{.bmp}
\item[\textit{Description}] Dateiformat der fertigen Graphik
\end{enumerate}

\item \wancicode{self.\_\_topmost}
\begin{enumerate}
\item[\textit{Methods}] \wancicode{TopTitle()}
\item[\textit{Valids}] nat\"urliche Zahlen
\item[\textit{Description}] Bezeichnet den kleinsten y-Wert, an dem
  weitere Bildelemente lokalisiert werden k\"onnen
\end{enumerate}

\item \wancicode{self.\_\_bottommost}
\begin{enumerate}
\item[\textit{Methods}] \wancicode{BottomTitle()} und
  \wancicode{BottomLegend()}
\item[\textit{Valids}] nat\"urliche Zahlen
\item[\textit{Description}] Bezeichnet den h\"ochsten y-Wert, an dem
  weitere Bildelemente lokalisiert werden k\"onnen
\end{enumerate}

\item \wancicode{self.\_\_leftmost}
\begin{enumerate}
\item[\textit{Methods}] \wancicode{LeftLegend()}
\item[\textit{Valids}] nat\"urliche Zahlen
\item[\textit{Description}] Bezeichnet den kleinsten x-Wert, an dem
  weitere Bildelemente lokalisiert werden k\"onnen
\end{enumerate}

\item \wancicode{self.\_\_rightmost}
\begin{enumerate}
\item[\textit{Methods}] \wancicode{RightLegend()}
\item[\textit{Valids}] nat\"urliche Zahlen
\item[\textit{Description}] Bezeichnet den h\"ochsten x-Wert, an dem
  weitere Bildelemente lokalisiert werden k\"onnen
\end{enumerate}

\item \wancicode{self.\_\_titlesize}
\begin{enumerate}
\item[\textit{Methods}] \wancicode{TopTitle()} und
  \wancicode{Bottomtitle}
\item[\textit{Valids}] 2-Tupel von nat\"urlichen Zahlen
\item[\textit{Description}] Bezeichnet die Gr\"o"se, die f\"ur das
  Titel-Bildobjekt vorgesehen ist
\end{enumerate}

\item \wancicode{self.\_\_titlelocation}
\begin{enumerate}
\item[\textit{Methods}] \wancicode{TopTitle()}
\item[\textit{Valids}] 2-Tupel von nat\"urlichen Zahlen
\item[\textit{Description}] Bezeichnet die Koordinaten, an denen das
  Titel-Bilobjekt lokalisiert wird
\end{enumerate}

\item \wancicode{self.\_\_legendsize}
\begin{enumerate}
\item[\textit{Methods}] \wancicode{LeftLegend()},
  \wancicode{RightLegend()}, \wancicode{BottomLegend()}
\item[\textit{Valids}] 2-Tupel von nat\"urlichen Zahlen
\item[\textit{Description}] Bezeichnet die Gr\"o"se, die f\"ur das
  Legenden-Bildobjekt vorgesehen ist
\end{enumerate}

\item \wancicode{self.\_\_legendlocation}
\begin{enumerate}
\item[\textit{Methods}] \wancicode{LeftLegend()},
  \wancicode{RightLegend()}, \wancicode{BottomLegend()}
\item[\textit{Valids}] 2-Tupel von nat\"urlichen Zahlen
\item[\textit{Description}] Bezeichnet die Koordinaten, an denen das
  Legenden-Bilobjekt lokalisiert wird
\end{enumerate}

\item \wancicode{self.\_\_graphicsize}
\begin{enumerate}
\item[\textit{Methods}] \wancicode{Graphic()}
\item[\textit{Valids}] 2-Tupel von nat\"urlichen Zahlen
\item[\textit{Description}] Bezeichnet die Gr\"o"se, die f\"ur das
  Graphic-Bildobjekt vorgesehen ist
\end{enumerate}

\item \wancicode{self.\_\_graphiclocation}
\begin{enumerate}
\item[\textit{Methods}] \wancicode{Graphic()}
\item[\textit{Valids}] 2-Tupel von nat\"urlichen Zahlen
\item[\textit{Description}] Bezeichnet die Koordinaten, an denen das
  Legenden-Bilobjekt lokalisiert wird
\end{enumerate}

\item \wancicode{self.\_\_pict}
\begin{enumerate}
\item[\textit{Methods}] \wancicode{Draw()},
  \wancicode{AttachWallpaper()}, \wancicode{Paste},
  \wancicode{Paste()}, \wancicode{Save()}, \wancicode{Show()}
\item[\textit{Valids}] Objekte, die erzeugt werden von
  \wancicode{Image.new()} der PIL Image-Moduls.
\item[\textit{Description}] Das gesamte Bildobjekt

\end{enumerate}

\end{itemize}

%%%%%%%% Methoden %%%%%%%%%%%%%%%%%%%%%%%%%%%%%%%%%%%%%%%%%%%

\newpage

\item[Methoden] Die Methoden von \wancicode{SDV}.
\begin{itemize}
  
\item \wancicode{Init()}
\begin{enumerate}
\item[\textit{Arguments}] \wancicode{filename = 'foo.bar.png'},
  \wancicode{size = (800,600)}, \wancicode{leftmargin = 30},
  \wancicode{rightmargin = 30}, \wancicode{topmargin = 30},
  \wancicode{bottommargin = 30}, \wancicode{dist = 10},
  \wancicode{colormode = ``RGB''}, \wancicode{color = (255,255,255)},
\item[\textit{Description}] Initialisiert alle ben\"otigten Parameter
  auf einmal. Alle Argument haben ein \textit{default}.
\item[\textit{Results}] \wancicode{self.\_\_filename},
  \wancicode{self.\_\_fileformat}, \wancicode{self.\_\_size},
  \wancicode{self.\_\_topmargin}, \wancicode{self.\_\_bottommargin},
  \wancicode{self.\_\_leftmargin}, \wancicode{self.\_\_rightmargin},
  \wancicode{self.\_\_dist}, \wancicode{self.\_\_colormode},
  \wancicode{self.\_\_color}


\end{enumerate}

\item \wancicode{SetUp()}
\begin{enumerate}
\item[\textit{Arguments}] keine
\item[\textit{Description}] aktiviert das Layout f\"ur die
  Gesamtgraphik
\item[\textit{Results}] \wancicode{self.\_\_topmost},
  \wancicode{self.\_\_bottommost}, \wancicode{self.\_\_leftmost},
  \wancicode{self.\_\_rightmost},
\end{enumerate}

\item \wancicode{SetFilename()}
\begin{enumerate}
\item[\textit{Arguments}] \wancicode{filename = '}
\item[\textit{Description}] Setzen des Dateinamens und -formates.
\item[\textit{Results}] \wancicode{self.\_\_filename},
  \wancicode{self.\_\_fileformat}
\end{enumerate}

\item \wancicode{SetSize()}
\begin{enumerate}
\item[\textit{Arguments}] \wancicode{size}
\item[\textit{Description}] Setzen der Gesamtbildgr\"o"se
\item[\textit{Results}] \wancicode{self.\_\_size}
\end{enumerate}

\item \wancicode{SetTopmargin()}
\begin{enumerate}
\item[\textit{Arguments}] \wancicode{topmargin}
\item[\textit{Description}] Setzen des oberen Bildrandes
\item[\textit{Results}] \wancicode{self.\_\_topmargin}
\end{enumerate}

\item \wancicode{SetBottommargin()}
\begin{enumerate}
\item[\textit{Arguments}] \wancicode{bottommargin}
\item[\textit{Description}] Setzen des unteren Bildrandes
\item[\textit{Results}] \wancicode{self.\_\_bottommargin}
\end{enumerate}

\item \wancicode{SetLeftmargin()}
\begin{enumerate}
\item[\textit{Arguments}] \wancicode{leftmargin}
\item[\textit{Description}] Setzen des linken Bildrandes
\item[\textit{Results}] \wancicode{self.\_\_leftmargin}
\end{enumerate}

\item \wancicode{SetRightmargin()}
\begin{enumerate}
\item[\textit{Arguments}] \wancicode{rightmargin}
\item[\textit{Description}] Setzen des rechten Bildrandes
\item[\textit{Results}] \wancicode{self.\_\_rightmargin}
\end{enumerate}

\item \wancicode{SetDist()}
\begin{enumerate}
\item[\textit{Arguments}] \wancicode{dist}
\item[\textit{Description}] Setzen des gegenseitigen Abstands der
  Bildelemente
\item[\textit{Results}] \wancicode{self.\_\_dist}
\end{enumerate}

\item \wancicode{SetColormode()}
\begin{enumerate}
\item[\textit{Arguments}] \wancicode{colormode}
\item[\textit{Description}] Setzen des Farbmodus
\item[\textit{Results}] \wancicode{self.\_\_colormode}
\end{enumerate}

\item \wancicode{SetColor()}
\begin{enumerate}
\item[\textit{Arguments}] \wancicode{color}
\item[\textit{Description}] Setzen der Hintergrundfarbe
\item[\textit{Results}] \wancicode{self.\_\_color}
\end{enumerate}

\item \wancicode{SetWallpaper()}
\begin{enumerate}
\item[\textit{Arguments}] \wancicode{wallpaper}
\item[\textit{Description}] \"Offnet ein Bildobjekt f\"ur ein
  Hintergrundbild
\item[\textit{Results}] \wancicode{self.\_\_wallpaper}
\end{enumerate}

\item \wancicode{TopTitle()}
\begin{enumerate}
\item[\textit{Arguments}] \wancicode{font}, \wancicode{text = 'test'}
\item[\textit{Description}] Plazieren und konfektionieren des
  Titel-Bildobjektes am oberen Bildrand
\item[\textit{Results}] \wancicode{self.\_\_titlesize},
  \wancicode{self.\_\_titlelocation}
\end{enumerate}

\item \wancicode{BottomTitle()}
\begin{enumerate}
\item[\textit{Arguments}] \wancicode{font}, \wancicode{text = 'test'}
\item[\textit{Description}] Plazieren und konfektionieren des
  Titel-Bildobjektes am unteren Bildrand
\item[\textit{Results}] \wancicode{self.\_\_titlesize},
  \wancicode{self.\_\_titlelocation}
\end{enumerate}

\item \wancicode{GetTitlesize()}
\begin{enumerate}
\item[\textit{Arguments}] keine
\item[\textit{Description}] Liefert die Gr\"o"se des Titel-
  Bildobjektes als 2-Tupel
\item[\textit{Results}] keine
\end{enumerate}

\item \wancicode{GetTitlelocation()}
\begin{enumerate}
\item[\textit{Arguments}] keine
\item[\textit{Description}] Liefert die Plazierung der linken oberen
  Ecke des Titel-Bildobjektes im Gesamtbild als 2-Tupel
\item[\textit{Results}] keine
\end{enumerate}

\item \wancicode{LeftLegend()}
\begin{enumerate}
\item[\textit{Arguments}] \wancicode{x}, \wancicode{y},
  \wancicode{font}
\item[\textit{Description}] Plazieren und konfektionieren des
  Legenden-Bildobjektes am linken Bildrand
\item[\textit{Results}] \wancicode{self.\_\_legendsize},
  \wancicode{self.\_\_legendlocation}
\end{enumerate}

\item \wancicode{RightLegend()}
\begin{enumerate}
\item[\textit{Arguments}] \wancicode{x}, \wancicode{y},
  \wancicode{font}
\item[\textit{Description}] Plazieren und konfektionieren des
  Legenden-Bildobjektes am rechten Bildrand
\item[\textit{Results}] \wancicode{self.\_\_legendsize},
  \wancicode{self.\_\_legendlocation}
\end{enumerate}

\item \wancicode{BottomLegend()}
\begin{enumerate}
\item[\textit{Arguments}] \wancicode{x}, \wancicode{y},
  \wancicode{font}
\item[\textit{Description}] Plazieren und konfektionieren des
  Legenden-Bildobjektes am unteren Bildrand
\item[\textit{Results}] \wancicode{self.\_\_legendsize},
  \wancicode{self.\_\_legendlocation}
\end{enumerate}

\item \wancicode{GetLegendsize()}
\begin{enumerate}
\item[\textit{Arguments}] keine
\item[\textit{Description}] Liefert die Gr\"o"se des Legenden-
  Bildobjektes als 2-Tupel
\item[\textit{Results}] keine
\end{enumerate}

\item \wancicode{GetLegendLocation()}
\begin{enumerate}
\item[\textit{Arguments}]
\item[\textit{Description}] Liefert die Plazierung der linken oberen
  Ecke des Legenden-Bildobjektes im Gesamtbild als 2-Tupel
\item[\textit{Results}] keine
\end{enumerate}

\item \wancicode{Graphic()}
\begin{enumerate}
\item[\textit{Arguments}] keine
\item[\textit{Description}]Plazieren und konfektionieren des
  Graphik-Bildobjektes
\item[\textit{Results}] \wancicode{self.\_\_graphicsize},
  \wancicode{self.\_\_graphiclocation}
\end{enumerate}

\item \wancicode{GetGraphicsize()}
\begin{enumerate}
\item[\textit{Arguments}] keine
\item[\textit{Description}] Liefert die Gr\"o"se des
  Graphik-Bildobjektes als 2-Tupel
\item[\textit{Results}] keine
\end{enumerate}

\item \wancicode{GetGraphiclocation()}
\begin{enumerate}
\item[\textit{Arguments}] keine
\item[\textit{Description}] Liefert die Plazierung der linken oberen
  Ecke des Graphik-Bildobjektes im Gesamtbild als 2-Tupel
\item[\textit{Results}] keine
\end{enumerate}

\item \wancicode{Draw()}
\begin{enumerate}
\item[\textit{Arguments}] keine
\item[\textit{Description}] Generiert das Gesamt-Bildobjekt
\item[\textit{Results}] \wancicode{self.\_\_pict}
\end{enumerate}

\item \wancicode{AttachWallpaper()}
\begin{enumerate}
\item[\textit{Arguments}] \wancicode{alpha = 0.0}
\item[\textit{Description}] Unterlegt das Hintergrundbild (sofern
  vorhanden) unter das fertige Gesamtbild
\item[\textit{Results}] keine
\end{enumerate}

\item \wancicode{Paste()}
\begin{enumerate}
\item[\textit{Arguments}] \wancicode{part}, \wancicode{location}
\item[\textit{Description}] F\"ugt ein Teil-Bildobjekt
  (\wancicode{part}) an der Stelle \wancicode{location} in das
  Gesamt-Bildobjekt ein.
\item[\textit{Results}] keine
\end{enumerate}

\item \wancicode{Save()}
\begin{enumerate}
\item[\textit{Arguments}] keine
\item[\textit{Description}] Speichert das Gesamt-Bildobjekt unter
  \wancicode{self.\_\_filename} und \wancicode{self.\_\_fileformat}
\item[\textit{Results}] keine
\end{enumerate}

\item \wancicode{Show()}
\begin{enumerate}
\item[\textit{Arguments}] keine
\item[\textit{Description}] Zeigt das Gesamt-Bildobjekt auf dem
  Bildschirm an.
\item[\textit{Results}] keine
\end{enumerate}

\end{itemize}

%%%%%%%%%% Objektfunktionalitaet %%%%%%%%%%%%%%%%%%%%%%%%%%%%%%

\newpage

\item[Objektfunktionalit\"at]
\end{description}

%%%%%%%%%%%%%%%%%%%%%%%%%%%%%%%%%%%%%%%%%%%%%%%%%%%%%%%%%%%%%%%%%
% Error
%%%%%%%%%%%%%%%%%%%%%%%%%%%%%%%%%%%%%%%%%%%%%%%%%%%%%%%%%%%%%%%%%

\newpage

\paragraph{\wancicode{Error}}
\subparagraph{Modul} \wancicode{error.py}

%%%%%%%%% Parameter %%%%%%%%%%%%%%%%%%%%%%%%%%%%%%%%%%%%%%%

\subparagraph{Parameter} 
\begin{itemize}
\item \wancicode{message}
\begin{enumerate}
\item[\textit{Methods}]
\wancicode{\_\_init\_\_(..., message = '',...)}
\item[\textit{Valids}] Jeder String
\item[\textit{Default}] \wancicode{''}
\item[\textit{Description}] Der Text der Fehlermeldung
\item[\textit{Attribute}] \wancicode{self.\_\_message}
\end{enumerate}

\item \wancicode{size}
\begin{enumerate}
\item[\textit{Methods}]
\wancicode{\_\_init\_\_(..., size = (640,480),...)}
\item[\textit{Valids}] Jedes 2-Tupel von positiven ganzen Zahlen
\item[\textit{Default}] \wancicode{(640,480}
\item[\textit{Description}] Die Gr\"o"se des Fehlergraphik
\item[\textit{Attribute}] \wancicode{self.\_\_size}
\end{enumerate}

\item \wancicode{font}
\begin{enumerate}
\item[\textit{Methods}]
\wancicode{\_\_init\_\_(..., font = '/fonts/helvB12.pil',...)}
\item[\textit{Valids}] Jeder g\"ultige Fontpfad
\item[\textit{Default}] \wancicode{'/fonts/helvB12.pil'}
\item[\textit{Description}] Der Font der Fehlermeldung
\item[\textit{Attribute}] \wancicode{self.\_\_font}
\end{enumerate}

\end{itemize}

%%%%%%%% Primaere Attribute %%%%%%%%%%%%%%%%%%%%%%%%%%%%%%%%

\newpage

\subparagraph{Prim\"are Attribute} 
\begin{itemize}
\item \wancicode{self.\_\_message}
\begin{enumerate}
\item[\textit{Methods}]
\wancicode{\_\_init\_\_(..., message = '',...)}
\item[\textit{Valids}] Jeder String
\item[\textit{Default}] \wancicode{''}
\item[\textit{Description}] Der Text der Fehlermeldung
\item[\textit{Parametre}] \wancicode{message}
\end{enumerate}

\item \wancicode{self.\_\_size}
\begin{enumerate}
\item[\textit{Methods}]
\wancicode{\_\_init\_\_(..., size = (640,480),...)}
\item[\textit{Valids}] Jedes 2-Tupel von positiven ganzen Zahlen  
\item[\textit{Default}] \wancicode{''}
\item[\textit{Description}] Die Gr\"o"se der Fehlergraphik
\item[\textit{Parametre}] \wancicode{size}
\end{enumerate}

\item \wancicode{self.\_\_font}
\begin{enumerate}
\item[\textit{Methods}]
\wancicode{\_\_init\_\_(..., font = '/fonts/helvB12.pil',...)}
\item[\textit{Valids}] Jeder g\"ultige Fontpfad
\item[\textit{Default}] \wancicode{'/fonts/helvB12.pil'}
\item[\textit{Description}] Der Font der Fehlermeldung
\item[\textit{Parametre}] \wancicode{font}
\end{enumerate}

\end{itemize}

%%%%%%%%%% Sekundaere Attribute %%%%%%%%%%%%%%%%%%%%%%%%%%

\newpage

\subparagraph{Sekund\"are Attribute} 
keine

%%%%%%%%%% Methoden %%%%%%%%%%%%%%%%%%%%%%%%%%%%%%%%%5

\newpage

\subparagraph{Methoden}
\begin{itemize}
\item \wancicode{\_\_init\_\_()}
\begin{enumerate}
\item[\textit{Arguments}] \wancicode{message}, \wancicode{size},
\wancicode{font}
\item[\textit{Description}] Initialisiert ein \wancicode{Error}-Objekt
\item[\textit{Results}] keine
\end{enumerate}

\item \wancicode{Draw()}
\begin{enumerate}
\item[\textit{Arguments}] keine
\item[\textit{Description}] Erzeugt ein Bildobjekt, das die
Fehlermeldung als Graphik ausgibt
\item[\textit{Results}]
\end{enumerate}

\end{itemize}

%%%%%%%%%%%%%%%%%%%%%%%%%%%%%%%%%%%%%%%%%%%%%%%%%%%%%%%%%%%%%%%%%%%
% Data
%%%%%%%%%%%%%%%%%%%%%%%%%%%%%%%%%%%%%%%%%%%%%%%%%%%%%%%%%%%%%%%%%%%

\newpage

\paragraph{\wancicode{Data}}
\subparagraph{Modul} \wancicode{data.py}

%%%%%%%% Parameter %%%%%%%%%%%%%%%%%%%%%%%%%%%%%%%%%%%%

\subparagraph{Parameter} 
\begin{itemize}

\item \wancicode{x}
\begin{enumerate}
\item[\textit{Methods}] \wancicode{SetX(x)}
\item[\textit{Valids}] Jede Liste von Strings oder Zahlen
\item[\textit{Default}] keiner
\item[\textit{Description}] Die Argumentliste
\end{enumerate}

\item \wancicode{y}
\begin{enumerate}
\item[\textit{Methods}] \wancicode{SetY(y)}
\item[\textit{Valids}] Jede Liste von Zahlen, die die 
gleiche L\"ange hat, wie \wancicode{x}
\item[\textit{Default}] keiner
\item[\textit{Description}] Die Wertliste
\end{enumerate}

\end{itemize}

%%%%%%% Primaere Attribute %%%%%%%%%%%%%%%%%%%%%%%%%%%%

\newpage

\subparagraph{Prim\"are Attribute}
\begin{itemize}

\item \wancicode{self.\_\_x}
\begin{enumerate}
\item[\textit{Methods}]
\wancicode{SetX(x)}
\item[\textit{Valids}] Jede Liste von Strings oder Zahlen
\item[\textit{Default}] keiner
\item[\textit{Description}] Die Argumentliste
\item[\textit{Parametre}] \wancicode{x}
\end{enumerate}

\item \wancicode{self.\_\_y}
\begin{enumerate}
\item[\textit{Methods}]
\wancicode{SetY(y)}
\item[\textit{Valids}] Jede Liste von Zahlen, die die 
gleiche L\"ange hat, wie \wancicode{x}  
\item[\textit{Default}] keiner
\item[\textit{Description}] Die Wertliste
\item[\textit{Parametre}] \wancicode{y}
\end{enumerate}

\end{itemize}

%%%%%%%% Sekundaere Attribute %%%%%%%%%%%%%%%%%%%%%%%%%%%

\newpage

\subparagraph{Sekund\"are Attribute}
keine

%%%%%%%% Methoden %%%%%%%%%%%%%%%%%%%%%%%%%%%%%%%%%%%%%%%%

\newpage

\subparagraph{Methoden}
\begin{itemize}

\item \wancicode{SetX()}
\begin{enumerate}
\item[\textit{Arguments}] \wancicode{x}
\item[\textit{Description}] Setzt die Argumentliste
\item[\textit{Results}] \wancicode{self.\_\_x}
\end{enumerate}

\item \wancicode{SetY()}
\begin{enumerate}
\item[\textit{Arguments}] \wancicode{y}
\item[\textit{Description}] Setzt die Wertliste
\item[\textit{Results}] \wancicode{self.\_\_y}
\end{enumerate}

\item \wancicode{GetX()}
\begin{enumerate}
\item[\textit{Arguments}] keine
\item[\textit{Description}] Liefert die Argumentliste
\item[\textit{Results}] keine
\end{enumerate}

\item \wancicode{GetY()}
\begin{enumerate}
\item[\textit{Arguments}] keine
\item[\textit{Description}] Liefert die Wertliste
\item[\textit{Results}] keine
\end{enumerate}

\end{itemize}

%%%%%%%%% Objektfunktionalitaet %%%%%%%%%%%%%%%%%%%%%%%%%%

\newpage

\subparagraph{Objektfunktionalit\"at}

%%%%%%%%%%%%%%%%%%%%%%%%%%%%%%%%%%%%%%%%%%%%%%%%%%%%%%%%%%%%%%%%%
% Size
%%%%%%%%%%%%%%%%%%%%%%%%%%%%%%%%%%%%%%%%%%%%%%%%%%%%%%%%%%%%%%%%%

\newpage

\paragraph{\wancicode{Size}}
\subparagraph{Modul}
\wancicode{data.py}

%%%%%%% Parameter %%%%%%%%%%%%%%%%%%%%%%%%%%%%%%%%%%

\subparagraph{Parameter}
\begin{itemize}
\item \wancicode{size}
\begin{enumerate}
\item[\textit{Methods}]
\wancicode{Init(size)} oder 
\wancicode{(SetSize(size)}
\item[\textit{Valids}] Jedes 2-Tupel von positiven ganzen Zahlen
\item[\textit{Default}] keine
\item[\textit{Description}] Die Gr\"o"senangabe
\item[\textit{Attribute}] \wancicode{self.\_\_size}
\end{enumerate}

\end{itemize}

%%%%%%% Primaere Attribute %%%%%%%%%%%%%%%%%%%%%%%%%%

\newpage

\subparagraph{Prim\"are Attribute}
\begin{itemize}
\item \wancicode{self.\_\_size}
\begin{enumerate}
\item[\textit{Methods}]
\wancicode{Init(size)}
\wancicode{SetSize(size)}
\item[\textit{Valids}] Jedes 2-Tupel von positiven ganzen Zahlen
\item[\textit{Default}] keiner
\item[\textit{Description}] Die Gr\"ossenangabe
\item[\textit{Parametre}] \wancicode{size}
\end{enumerate}

\end{itemize}

%%%%%%%%% Sekundaere Attribute %%%%%%%%%%%%%%%%%%%%%%

\newpage

\subparagraph{Sekund\"are Attribute}
keine

%%%%%%%%% Methoden %%%%%%%%%%%%%%%%%%%%%%%%%%%%%%%%%%

\newpage

\subparagraph{Methoden}
\begin{itemize}

\item \wancicode{Init(size)}
\begin{enumerate}
\item[\textit{Arguments}] \wancicode{size}
\item[\textit{Description}] Initialisiert die Klasse
\item[\textit{Results}] \wancicode{self.\_\_size}
\end{enumerate}

\item \wancicode{SetSize(size)}
\begin{enumerate}
\item[\textit{Arguments}] \wancicode{size}
\item[\textit{Description}] Setzt die Gr\"o"enangabe
\item[\textit{Results}] \wancicode{self.\_\_size}
\end{enumerate}

\item \wancicode{GetSize()}
\begin{enumerate}
\item[\textit{Arguments}] keine
\item[\textit{Description}] Liefert die Gr\"o"enangabe
\item[\textit{Results}] keine
\end{enumerate}

\end{itemize}

%%%%%%%% Objektfuntkionalitaet %%%%%%%%%%%%%%%%%%%%%%%%%%%%%%

\newpage

\subparagraph{Objektfunktionalit\"at}

%%%%%%%%%%%%%%%%%%%%%%%%%%%%%%%%%%%%%%%%%%%%%%%%%%%%%%%%%%%%%%%%%%
% Colormode
%%%%%%%%%%%%%%%%%%%%%%%%%%%%%%%%%%%%%%%%%%%%%%%%%%%%%%%%%%%%%%%%%%

\newpage

\paragraph{\wancicode{Colormode}}

\subparagraph{Modul} \wancicode{colors.py}

%%%%%%%%%% Parameter %%%%%%%%%%%%%%%%%%%%%%%%%%%%%%%%%%%%%

%%%%%%%%%%  %%%%%%%%%%%%%%%%%%%%%%%%%%%%%%%%%%%%%%%%
\subparagraph{Parameter}
\begin{itemize}

\item \wancicode{mode}
\begin{enumerate}
\item[\textit{Methods}]
\wancicode{Init(mode = ``RGB'')},
\wancicode{SetMode(mode = ``RGB'')}
\item[\textit{Valids}] \wancicode{``RGB'', ``RGBA'', ``CMYK'', 
``1'', ``P'', ``L'', ``XYZ''}
\item[\textit{Default}] \wancicode{``RGB''}
\item[\textit{Description}] Der Farbmodus
\item[\textit{Attribute}] \wancicode{self.\_\_mode}
\end{enumerate}

\end{itemize}

%%%%%%%%%% Primaere Attribute %%%%%%%%%%%%%%%%%%%%%%%%%%%%%%%%%%%%

\newpage

\subparagraph{Prim\"are Attribute}
\begin{itemize}

\item \wancicode{self.\_\_mode}
\begin{enumerate}
\item[\textit{Methods}] \wancicode{Init(mode = ``RGB'')}
\item[\textit{Valids}] \wancicode{``RGB'', ``RGBA'', ``CMYK'', 
``1'', ``P'', ``L'', ``XYZ''}
\item[\textit{Default}] ``RGB''
\item[\textit{Description}] Der Farbmodus
\item[\textit{Parametre}] \wancicode{mode}
\end{enumerate}

\end{itemize}

%%%%%%%%%% Sekundaere Attribute %%%%%%%%%%%%%%%%%%%%%%%%%%%%%%%%%%%

\newpage

\subparagraph{Sekund\"are Attribute}
keine

%%%%%%%%%%  Methoden %%%%%%%%%%%%%%%%%%%%%%%%%%%%%%%%%%%%%%%%

\newpage

\subparagraph{Methoden}
\begin{itemize}

\item \wancicode{Init()}
\begin{enumerate}
\item[\textit{Arguments}] \wancicode{mode}
\item[\textit{Description}] Initialisiert die Klasse
\item[\textit{Results}] \wancicode{self.\_\_mode}
\end{enumerate}

\item \wancicode{SetMode()}
\begin{enumerate}
\item[\textit{Arguments}] \wancicode{mode} 
\item[\textit{Description}] Setzt den Farbmodus
\item[\textit{Results}] \wancicode{self.\_\_mode}
\end{enumerate}

\item \wancicode{GetMode()}
\begin{enumerate}
\item[\textit{Arguments}] keine 
\item[\textit{Description}] Liefert den Farbmodus
\item[\textit{Results}] keine
\end{enumerate}

\end{itemize}

%%%%%%%%%% Objektfunktionalitaet  %%%%%%%%%%%%%%%%%%%%%%%%%%%%%%%%

\newpage

\subparagraph{Objektfunktionalit\"at}

%%%%%%%%%%%%%%%%%%%%%%%%%%%%%%%%%%%%%%%%%%%%%%%%%%%%%%%%%%%%%%%%%%
% Color
%%%%%%%%%%%%%%%%%%%%%%%%%%%%%%%%%%%%%%%%%%%%%%%%%%%%%%%%%%%%%%%%%%

\newpage

\paragraph{\wancicode{Color}}
\subparagraph{Modul}
\wancicode{colors.py}

%%%%%%%%% Parameter %%%%%%%%%%%%%%%%%%%%%%%%%%%%%%%%%%%%%%%%%%

\subparagraph{Parameter}

\begin{itemize}
\item \wancicode{mode}
\begin{enumerate}
\item[\textit{Methods}]
\wancicode{Init(mode = ``RGB'')}
\wancicode{SetMode(mode)}
\item[\textit{Valids}] \wancicode{``RGB'', ``RGBA'', ``CMYK'', 
``1'', ``P'', ``L'', ``XYZ''}
\item[\textit{Default}] \wancicode{``RGB''}
\item[\textit{Description}] Der Farbmodus des Farbobjektes
\item[\textit{Attribute}] \wancicode{self.\_\_mode}
\end{enumerate}

\item \wancicode{value}
\begin{enumerate}
\item[\textit{Methods}] 
\wancicode{SetValue(value)}
\item[\textit{Valids}] Jede korrekte Farbangabe in Abh\"angigkeit
von \wancicode{self.\_\_mode}
\item[\textit{Default}] keiner
\item[\textit{Description}] Der Farbwert
\item[\textit{Attribute}] \wancicode{self.\_\_value}
\end{enumerate}

\item \wancicode{shade}
\begin{enumerate}
\item[\textit{Methods}] 
\wancicode{ShadeDown(), ShadeUp()}
\item[\textit{Valids}] Jede ganze Zahl
\item[\textit{Default}] keiner
\item[\textit{Description}] Wert, um den \wancicode{self.\_\_value} in
ab- bzw. aufschattiert wird
\item[\textit{Attribute}] keine
\end{enumerate}

\end{itemize}

\newpage

\subparagraph{Prim\"are Attribute}
\begin{description}
\item \wancicode{self.\_\_mode}
\begin{enumerate}
\item[\textit{Methods}]
\wancicode{Init(), SetMode()}
\item[\textit{Valids}] 
\wancicode{``RGB'', ``RGBA'', ``CMYK'', ``1'', ``P'', ``L'', ``XYZ''}
\item[\textit{Default}] \wancicode{``RGB''}
\item[\textit{Description}] Der Farbmodus
\item[\textit{Parametre}] \wancicode{mode}
\end{enumerate}

\item \wancicode{self.\_\_value}
\begin{enumerate}
\item[\textit{Methods}] \wancicode{SetValue()}
\item[\textit{Valids}] Jede korrekte Farbangabe in Abh\"angigkeit
von \wancicode{self.\_\_mode}
\item[\textit{Default}] keiner
\item[\textit{Description}] Der Farbwert
\item[\textit{Parametre}] \wancicode{value}
\end{enumerate}

\end{description}

\newpage

\subparagraph{Sekund\"are Attribute}
keine

\newpage

\subparagraph{Methoden}
\begin{description}
\item \wancicode{Init()}
\begin{enumerate}
\item[\textit{Arguments}] \wancicode{mode}
\item[\textit{Description}] Initialisiert ein Objekt
\item[\textit{Results}] \wancicode{self.\_\_mode}
\end{enumerate}

\item \wancicode{TestRGB()}
\begin{enumerate}
\item[\textit{Arguments}] \wancicode{value}
\item[\textit{Description}] Testet die Eingabe auf Korrektheit bzgl.
dem RGB-Format
\item[\textit{Results}] \wancicode{1} oder \wancicode{0}
\end{enumerate}

\item \wancicode{Test()}
\begin{enumerate}
\item[\textit{Arguments}] \wancicode{value}
\item[\textit{Description}] Testet die Eingabe auf Korrektkeit bzgl.
\wancicode{self.\_\_mode}
\item[\textit{Results}] \wancicode{1} oder \wancicode{0}
\end{enumerate}

\item \wancicode{SetMode()}
\begin{enumerate}
\item[\textit{Arguments}] \wancicode{mode}
\item[\textit{Description}] Setzt den Farbmodus
\item[\textit{Results}] \wancicode{self.\_\_mode}
\end{enumerate}

\item \wancicode{SetValue()}
\begin{enumerate}
\item[\textit{Arguments}]  \wancicode{value}
\item[\textit{Description}] Setzt den Farbwert
\item[\textit{Results}] \wancicode{self.\_\_value}
\end{enumerate}

\item \wancicode{ShadeDown()}
\begin{enumerate}
\item[\textit{Arguments}] \wancicode{shade}
\item[\textit{Description}] Schattiert \wancicode{self.\_\_value}
um \wancicode{shade} ab. \wancicode{self.\_\_value} bleibt unver\"andert
\item[\textit{Results}] Abschattierter Farbwert
\end{enumerate}

\item \wancicode{ShadeUp()}
\begin{enumerate}
\item[\textit{Arguments}] \wancicode{shade}
\item[\textit{Description}] Schattiert \wancicode{self.\_\_value}
um \wancicode{shade} auf. \wancicode{self.\_\_value} bleibt unver\"andert
\item[\textit{Results}] Aufschattierter Farbwert
\end{enumerate}

\item \wancicode{GetMode()}
\begin{enumerate}
\item[\textit{Arguments}] keine
\item[\textit{Description}] Liefert \wancicode{self.\_\_mode}
\item[\textit{Results}] keine
\end{enumerate}

\item \wancicode{GetValue()}
\begin{enumerate}
\item[\textit{Arguments}] keine
\item[\textit{Description}] Liefert \wancicode{self.\_\_value}
\item[\textit{Results}] keine
\end{enumerate}

\end{description}

\subparagraph{Objektfunktionalit\"at}

%%%%%%%%%%%%%%%%%%%%%%%%%%%%%%%%%%%%%%%%%%%%%%%%%%%%%%%%%%%%%%%%%%%%%%%%
% Colorlist
%%%%%%%%%%%%%%%%%%%%%%%%%%%%%%%%%%%%%%%%%%%%%%%%%%%%%%%%%%%%%%%%%%%%%%%%


\newpage

\paragraph{\wancicode{Colorlist}}
\subparagraph{Modul} \wancicode{colors.py}

%%%%%%%%% Parameter %%%%%%%%%%%%%%%%%%%%%%%%%%%%%%%%%%%%%%%%%%%%%

\subparagraph{Parameter}
\begin{itemize}
\item \wancicode{list}
\begin{enumerate}
\item[\textit{Methods}] \wancicode{Init(..., list, ...)},
\wancicode{SetLight(list)}
\item[\textit{Valids}] Jede Liste von Farben, die 
\wancicode{self.\_\_mode} entsprechen
\item[\textit{Default}] keiner
\item[\textit{Description}] Die Farbliste
\item[\textit{Attribute}] \wancicode{self.\_\_light}
\end{enumerate}

\item \wancicode{mode}
\begin{enumerate}
\item[\textit{Methods}]
\wancicode{Init(..., mode = ``RGB'', ...)},
\wancicode{SetMode(mode = ``RGB'')}
\item[\textit{Valids}] 
\wancicode{``RGB'', ``RGBA'', ``CMYK'', ``1'', ``P'', ``L'', ``XYZ''}
\item[\textit{Default}] \wancicode{``RGB''}
\item[\textit{Description}] Der Farbmodus f\"ur alle Farben in
\wancicode{self.\_\_light, self.\_\_medium, self.\_\_dark}
\item[\textit{Attribute}] \wancicode{self.\_\_mode}
\end{enumerate}

\end{itemize}

%%%%%%%%%% Primaere Attribute%%%%%%%%%%%%%%%%%%%%%%%%%%%%%%%%%%%

\newpage

\subparagraph{Prim\"are Attribute}
\begin{itemize}
\item \wancicode{self.\_\_light}
\begin{enumerate}
\item[\textit{Methods}] \wancicode{Init()}, \wancicode{SetLight}
\item[\textit{Valids}] Jede Liste von Farben, die 
\wancicode{self.\_\_mode} entsprechen
\item[\textit{Default}] keiner
\item[\textit{Description}] Die hellste der drei resultierenden
Farblisten
\item[\textit{Parametre}] \wancicode{list}
\end{enumerate}

\item \wancicode{self.\_\_mediumshade}
\begin{enumerate}
\item[\textit{Methods}] \wancicode{SetMediumshade()}
\item[\textit{Valids}] Jede ganze Zahl
\item[\textit{Default}] keiner
\item[\textit{Description}] Mittlerer Abschattierungskoeffizient
\item[\textit{Parametre}] \wancicode{shade}
\end{enumerate}

\item \wancicode{self.\_\_darkshade}
\begin{enumerate}
\item[\textit{Methods}] \wancicode{SetDarkshade()}
\item[\textit{Valids}] Jede ganze Zahl
\item[\textit{Default}] keiner
\item[\textit{Description}] Dunkler Abbschattierungskoeefizient
\item[\textit{Parametre}] \wancicode{shade}
\end{enumerate}

\item \wancicode{self.\_\_mode}
\begin{enumerate}
\item[\textit{Methods}] \wancicode{Init()}, \wancicode{SetMode()}
\item[\textit{Valids}] 
\wancicode{``RGB'', ``RGBA'', ``CMYK'', ``1'', ``P'', ``L'', ``XYZ''}
\item[\textit{Default}] \wancicode{``RGB''}
\item[\textit{Description}] Der Farbmodus f\"ur 
\wancicode{self.\_\_light, self.\_\_medium, self.\_\_dark}
\item[\textit{Parametre}] \wancicode{mode}
\end{enumerate}

\end{itemize}

%%%%%%%% Sekundaere Attribute %%%%%%%%%%%%%%%%%%%%%%%%%%%%%%

\newpage

\subparagraph{Sekund\"are Attribute}
\begin{itemize}

\item \wancicode{self.\_\_medium}
\begin{enumerate}
\item[\textit{Methods}] \wancicode{SetMedium()}
\item[\textit{Valids}] Jede Liste von Farben, die 
\wancicode{self.\_\_mode} entsprechen
\item[\textit{Description}] Die Farbliste mittlerer Helligkeit
\end{enumerate}

\item \wancicode{self.\_\_dark}
\begin{enumerate}
\item[\textit{Methods}] \wancicode{SetDark()}
\item[\textit{Valids}] Jede Liste von Farben, die 
\wancicode{self.\_\_mode} entsprechen
\item[\textit{Description}] Die dunkelste Farbliste
\end{enumerate}

\end{itemize}

%%%%%%%%%% Methoden %%%%%%%%%%%%%%%%%%%%%%%%%%%%%%%%%%%%%%%%%%

\newpage

\subparagraph{Methoden}
\begin{itemize}

\item \wancicode{Init()}
\begin{enumerate}
\item[\textit{Arguments}] \wancicode{list}, \wancicode{mode}
\item[\textit{Description}] Initialisiert ein 
Farblistenobjekt
\item[\textit{Results}] \wancicode{self.\_\_light, 
self.\_\_mediumshade = 0, self.\_\_darshade = 0}
\end{enumerate}

\item \wancicode{SetLight()}
\begin{enumerate}
\item[\textit{Arguments}] \wancicode{list}
\item[\textit{Description}] Setzt die Liste \wancicode{self.\_\_light}
\item[\textit{Results}] \wancicode{self.\_\_light}
\end{enumerate}

\item \wancicode{SetMode()}
\begin{enumerate}
\item[\textit{Arguments}] \wancicode{mode}
\item[\textit{Description}] Setzt den Farbmodus der Listen
\item[\textit{Results}] \wancicode{self.\_\_mode}
\end{enumerate}

\item \wancicode{SetMediumshade()}
\begin{enumerate}
\item[\textit{Arguments}] \wancicode{shade}
\item[\textit{Description}] Setzt den mittleren 
Abschattierungskoeffizienten
\item[\textit{Results}] \wancicode{self.\_\_mediumshade}
\end{enumerate}

\item \wancicode{SetDarkshade()}
\begin{enumerate}
\item[\textit{Arguments}] \wancicode{shade}
\item[\textit{Description}] Setzt den dunklen 
Abschattierungskoeffizienten 
\item[\textit{Results}] \wancicode{self.\_\_darkshade}
\end{enumerate}

\item \wancicode{SetMedium()}
\begin{enumerate}
\item[\textit{Arguments}] keine
\item[\textit{Description}] Setzt die 
mittlere Farbliste
\item[\textit{Results}] \wancicode{self.\_\_medium}
\end{enumerate}

\item \wancicode{SetDark()}
\begin{enumerate}
\item[\textit{Arguments}] keine
\item[\textit{Description}] Setzt die 
dunkle Farbliste
\item[\textit{Results}] \wancicode{self.\_\_dark}
\end{enumerate}

\item \wancicode{GetLight()}
\begin{enumerate}
\item[\textit{Arguments}] keine
\item[\textit{Description}] Liefert die helle 
Farbliste
\item[\textit{Results}] keine
\end{enumerate}

\item \wancicode{GetMedium()} 
\begin{enumerate}
\item[\textit{Arguments}] keine
\item[\textit{Description}] Liefert die mittlere
Farbliste
\item[\textit{Results}] keine
\end{enumerate}

\item \wancicode{GetDark()}
\begin{enumerate}
\item[\textit{Arguments}] keine
\item[\textit{Description}] Liefert die dunkle
Farbliste
\item[\textit{Results}] keine
\end{enumerate}

\end{itemize}

\newpage

\subparagraph{Objektfunktionalit\"at}

%%%%%%%% Font %%%%%%%%%%%%%%%%%%%%%%%%%%%%%%%%%%%%%%%%%%%%%%%%

\newpage

\paragraph{\wancicode{Font}}
\subparagraph{Modul} \wancicode{file.py}

%%%%%%%% Parameter %%%%%%%%%%%%%%%%%%%%%%%%%%%%%%%%%%%%%%%%

\subparagraph{Parameter}
\begin{itemize}

\item \wancicode{name}
\begin{enumerate}
\item[\textit{Methods}] \wancicode{SetName(name)}
\item[\textit{Valids}] Jeder g\"ultige Font-Dateiname
\item[\textit{Default}] keiner
\item[\textit{Description}] Der Name der Fontdatei
\item[\textit{Attribute}] \wancicode{self.\_\_name}
\end{enumerate}

\item \wancicode{dir}
\begin{enumerate}
\item[\textit{Methods}] \wancicode{SetDir(dir) = '/fonts/'}
\item[\textit{Valids}] Jeder g\"ultige Verzeichnisname
\item[\textit{Default}] \wancicode{'/fonts/'}
\item[\textit{Description}] Der Name des Fontverzeichnisses
\item[\textit{Attribute}] \wancicode{self.\_\_dir}
\end{enumerate}

\item \wancicode{path}
\begin{enumerate}
\item[\textit{Methods}] \wancicode{SetPath(path)}
\item[\textit{Valids}] Jeder g\"ultige Pfadname
\item[\textit{Default}] keiner
\item[\textit{Description}] Der komplett Pfadname, also
die Konkatenation von Datei- und Verzeichnisname
\item[\textit{Attribute}] \wancicode{self.\_\_path}
\end{enumerate}

\end{itemize}

%%%%%%%% Primaere Attribute %%%%%%%%%%%%%%%%%%%%%%%%%%%%%%%%%%%

\newpage

\subparagraph{Prim\"are Attribute}
\begin{itemize}
\item \wancicode{self.\_\_name}
\begin{enumerate}
\item[\textit{Methods}] \wancicode{SetName(),
Path2Name()}
\item[\textit{Description}] Der Fontdateiname
\item[\textit{Parametre}] \wancicode{name}
\end{enumerate}

\item \wancicode{self.\_\_dir}
\begin{enumerate}
\item[\textit{Methods}] \wancicode{SetDir()}
\item[\textit{Description}] Der Fontverzeichnisname
\item[\textit{Parametre}] \wancicode{dir}
\end{enumerate}

\item \wancicode{self.\_\_path}
\begin{enumerate}
\item[\textit{Methods}] \wancicode{SetPath()}
\item[\textit{Description}] Der Fondpfadname
\item[\textit{Parametre}] \wancicode{path}
\end{enumerate}

\end{itemize}

%%%%%%%% Sekundaere Attribute %%%%%%%%%%%%%%%%%%%%%%%%%%%%%%%%

\newpage

\subparagraph{Sekund\"are Attribute}
keine

%%%%%%%%%% Methoden %%%%%%%%%%%%%%%%%%%%%%%%%%%%%%%%%%%%%%%%%%%

\newpage

\subparagraph{Methoden}
\begin{itemize}
\item \wancicode{SetName()}
\begin{enumerate}
\item[\textit{Arguments}] \wancicode{name}
\item[\textit{Description}] Setzt den Fontdateinamen
\item[\textit{Results}] \wancicode{self.\_\_name}
\end{enumerate}

\item \wancicode{SetDir()}
\begin{enumerate}
\item[\textit{Arguments}] \wancicode{dir}
\item[\textit{Description}] Setzt den Fontverzeichnisnamen
\item[\textit{Results}] \wancicode{self.\_\_dir}
\end{enumerate}

\item \wancicode{SetPath()}
\begin{enumerate}
\item[\textit{Arguments}] \wancicode{path}
\item[\textit{Description}] Setzt den Fontpfadnamen
\item[\textit{Results}] \wancicode{self.\_\_path}
\end{enumerate}

\item \wancicode{Name2Path()}
\begin{enumerate}
\item[\textit{Arguments}] keine
\item[\textit{Description}] Konkateniert Verzeichnis- und
Dateinamen und setzt mit dem Ergebnis den Pfadnamen
\item[\textit{Results}] \wancicode{self.\_\_path}
\end{enumerate}

\item \wancicode{Path2NameDir()}
\begin{enumerate}
\item[\textit{Arguments}] keine
\item[\textit{Description}] Splittet den Pfadnamen und
setzt mit dem Ergebnis Datei- und Verzeichnisnamen
\item[\textit{Results}] \wancicode{self.\_\_name, 
self.\_\_dir}
\end{enumerate}

\item \wancicode{GetName()}
\begin{enumerate}
\item[\textit{Arguments}] keine
\item[\textit{Description}] Liefert den Fontdateinamen
\item[\textit{Results}] keine
\end{enumerate}

\item \wancicode{GetPath()}
\begin{enumerate}
\item[\textit{Arguments}] keine
\item[\textit{Description}] Liefert den Fontpfadnamen
\item[\textit{Results}] keine
\end{enumerate}

\item \wancicode{GetSize()}
\begin{enumerate}
\item[\textit{Arguments}] \wancicode{text}
\item[\textit{Description}] Liefert Pixelbreite und -h\"ohe
von \wancicode{text} im Font des Objektes als 2-Tupel
\item[\textit{Results}] keine
\end{enumerate}

\end{itemize}

%%%%%%%% OF %%%%%%%%%%%%%%%%%%%%%%%%%%%%%%%%%%%%%%%%%%%%%%%%

\newpage

\subparagraph{Objektfunktionalit\"at}

%%%%%%%%%%%%%%%%%%%%%%%%%%%%%%%%%%%%%%%%%%%%%%%%%%%%%%%%%%%%%%
% File
%%%%%%%%%%%%%%%%%%%%%%%%%%%%%%%%%%%%%%%%%%%%%%%%%%%%%%%%%%%%%%

\newpage

\paragraph{\wancicode{File}}

\subparagraph{Modul} \wancicode{file.py}

%%%%%%%%% Parameter %%%%%%%%%%%%%%%%%%%%%%%%%%%%%%%%%%%%%

\subparagraph{Parameter}
\begin{itemize}
\item \wancicode{file}
\begin{enumerate}
\item[\textit{Methods}] \wancicode{SetFile()}
\item[\textit{Valids}] jeder g\"ultige Dateiname 
mit einem Format aus \wancicode{'jpg', 'gif', 'png', 'bmp'}
\item[\textit{Default}] keiner
\item[\textit{Description}] Der Dateiname 
\item[\textit{Attribute}] \wancicode{self.\_\_file}
\end{enumerate}

\item \wancicode{dir}
\begin{enumerate}
\item[\textit{Methods}] \wancicode{SetDir()}
\item[\textit{Valids}] Jeder g\"ultige Verzeichnisname
\item[\textit{Default}] keiner
\item[\textit{Description}] Der Verzeichnisname
\item[\textit{Attribute}] \wancicode{self.\_\_dir}
\end{enumerate}

\item \wancicode{path}
\begin{enumerate}
\item[\textit{Methods}] \wancicode{SetPath()}
\item[\textit{Valids}] entsprechend \wancicode{dir} und
\wancicode{file}
\item[\textit{Default}] keiner
\item[\textit{Description}] Der komplette Pfadname
\item[\textit{Attribute}] \wancicode{self.\_\_path}
\end{enumerate}

\end{itemize}

%%%%%%%% Primaere Attribute %%%%%%%%%%%%%%%%%%%%%%%%%%%%%%%%

\newpage 

\subparagraph{Prim\"are Attribute}
\begin{itemize}

\item \wancicode{self.\_\_file}
\begin{enumerate}
\item[\textit{Methods}] \wancicode{SetFile()},
\wancicode{Path2File()}
\item[\textit{Description}] Der Dateiname
\item[\textit{Parametre}] \wancicode{file}
\end{enumerate}

\item \wancicode{self.\_\_dir}
\begin{enumerate}
\item[\textit{Methods}] \wancicode{SetDir()}
\item[\textit{Description}] Der Verzeichnisname
\item[\textit{Parametre}] \wancicode{dir}
\end{enumerate}

\item \wancicode{self.\_\_path}
\begin{enumerate}
\item[\textit{Methods}] \wancicode{SetPath()}, 
\wancicode{File2Path()}
\item[\textit{Description}] Der komplette Pfadname
\item[\textit{Parametre}] \wancicode{path}
\end{enumerate}

\end{itemize}

%%%%%%%%%% Sekundaere Attribute %%%%%%%%%%%%%%%%%%%%

\newpage

\subparagraph{Sekund\"are Attribute}
keine

%%%%%%%%%% Methoden %%%%%%%%%%%%%%%%%%%%%%%%%%%%%%%%%%%

\newpage

\subparagraph{Methoden}

\begin{itemize}

\item \wancicode{SetFile()}
\begin{enumerate}
\item[\textit{Arguments}] \wancicode{file}
\item[\textit{Description}] Setzt den Dateinamen
\item[\textit{Results}] \wancicode{self.\_\_file}
\end{enumerate}

\item \wancicode{SetDir()}
\begin{enumerate}
\item[\textit{Arguments}] \wancicode{dir}
\item[\textit{Description}] Setzt den Verzeichnisnamen
\item[\textit{Results}] \wancicode{self.\_\_dir}
\end{enumerate}

\item \wancicode{SetPath()}
\begin{enumerate}
\item[\textit{Arguments}] \wancicode{path}
\item[\textit{Description}] Setzt den Pfadnamen
\item[\textit{Results}] \wancicode{self.\_\_path}
\end{enumerate}

\item \wancicode{File2Path()}
\begin{enumerate}
\item[\textit{Arguments}] keine
\item[\textit{Description}] Konkateniert Datei-
und Verzeichnisnamen und setzt mit dem Ergebnis
Den Pfadnamen
\item[\textit{Results}] \wancicode{self.\_\_path}
\end{enumerate}

\item \wancicode{Path2File()}
\begin{enumerate}
\item[\textit{Arguments}] keine 
\item[\textit{Description}] Splittet den Pfadnamen und 
setzt mit den Ergebnissen den Datei- und Verzeichnisnamen
\item[\textit{Results}] \wancicode{self.\_\_file}, 
\wancicode{self.\_\_dir}
\end{enumerate}

\item \wancicode{GetFile()}
\begin{enumerate}
\item[\textit{Arguments}] keine
\item[\textit{Description}] Liefert den Dateinamen
\item[\textit{Results}] keine
\end{enumerate}

\item \wancicode{GetPath()}
\begin{enumerate}
\item[\textit{Arguments}] keine 
\item[\textit{Description}] Liefert den Pfadnamen
\item[\textit{Results}] keine
\end{enumerate}

\item \wancicode{GetDir()}
\begin{enumerate}
\item[\textit{Arguments}] keine
\item[\textit{Description}] Liefert den Verzeichnisnamen
\item[\textit{Results}] keine
\end{enumerate}

\end{itemize}


%%%%%%%%%% OF %%%%%%%%%%%%%%%%%%%%%%%%%%%%%%%%%%%%%%%%

\newpage

\subparagraph{Objektfunktionalit\"at}


%%%%%%%%%%%%%%%%%%%%%%%%%%%%%%%%%%%%%%%%%%%%%%%%%%%%%%%%%%%%%%%
% Title
%%%%%%%%%%%%%%%%%%%%%%%%%%%%%%%%%%%%%%%%%%%%%%%%%%%%%%%%%%%%%%%

\newpage

\paragraph{\wancicode{Title}}
\subparagraph{Modul} \wancicode{title.py}

%%%%%%%%%% Parameter %%%%%%%%%%%%%%%%%%%%%%%%%%%%%%%%%%%%%%

\subparagraph{Parameter}
\begin{itemize}

\item \wancicode{size}
\begin{enumerate}
\item[\textit{Methods}]
\wancicode{Init(..., size, ...)}, 
\wancicode{SetSize(size)}
\item[\textit{Valids}] Jedes g\"ultige Gr\"o"sen- 2-Tupel
\item[\textit{Default}] keiner
\item[\textit{Description}] Die Gr\"o"senangabe
f\"ur das Titel- Bildelement
\item[\textit{Attribute}] \wancicode{self.\_\_size}
\end{enumerate}

\item \wancicode{colormode}
\begin{enumerate}
\item[\textit{Methods}] 
\wancicode{Init(...,colormode = ``RGB'', ...)},
\wancicode{SetColormode(colormode = ``RGB'')}
\item[\textit{Valids}] \wancicode{``RGB'', ``RGBA'', ``CMYK'',
``1'', ``L'', ``P'', ``XYZ''}
\item[\textit{Default}] \wancicode{``RGB''}
\item[\textit{Description}] Das Farbformat f\"ur das
Titel- Bildelement
\item[\textit{Attribute}] \wancicode{colormode}
\end{enumerate}

\item \wancicode{text}
\begin{enumerate}
\item[\textit{Methods}] \wancicode{Init(...,text = '', ...)},
\wancicode{SetText(text)}
\item[\textit{Valids}] Jeder String
\item[\textit{Default}] \wancicode{''}
\item[\textit{Description}] Der Text der Titelzeile
\item[\textit{Attribute}] \wancicode{self.\_\_text}
\end{enumerate}

\item \wancicode{font}
\begin{enumerate}
\item[\textit{Methods}] 
\wancicode{Init(...,font = 'helvB12.pil', ...)},
\wancicode{SetFont(font)}
\item[\textit{Valids}] Jeder g\"ultige Fontpfad
\item[\textit{Default}] \wancicode{/fonts/helvB12.pil'}
\item[\textit{Description}] Der Font f\"ur den Titeltext
\item[\textit{Attribute}] \wancicode{self.\_\_font}
\end{enumerate}

\item \wancicode{textcolor}
\begin{enumerate}
\item[\textit{Methods}] 
\wancicode{Init(...,textcolor = (0,0,0), ...)},
\wancicode{SetTextcolor()}
\item[\textit{Valids}] Jede g\"ultige Farbangabe entsprechend
\wancicode{colorformat}
\item[\textit{Default}] \wancicode{(0,0,0)}
\item[\textit{Description}] Die Farbe f\"ur den Titeltext
\item[\textit{Attribute}] \wancicode{self.\_\_textcolor}
\end{enumerate}

\item \wancicode{color}
\begin{enumerate}
\item[\textit{Methods}] 
\wancicode{Init(...,color = (255,255,255), ...)},
\wancicode{SetColor(color)}
\item[\textit{Valids}] Jede g\"ultige Farbangabe entsprechend
\wancicode{colorformat} 
\item[\textit{Default}] \wancicode{(255,255,255)}
\item[\textit{Description}] Die Hintergrundfarbe des 
Titelfeldes
\item[\textit{Attribute}] \wancicode{self.\_\_color}
\end{enumerate}

\end{itemize}

%%%%%%%%% Primaere Attribute %%%%%%%%%%%%%%%%%%%%%%%%%%%%%%%%

\newpage

\subparagraph{Prim\"are Attribute}
\begin{itemize}

\item \wancicode{self.\_\_size}
\begin{enumerate}
\item[\textit{Methods}] 
\wancicode{Init(..., size, ...)}, 
\wancicode{SetSize(size)}
\item[\textit{Valids}] Jedes g\"ultige Gr\"o"sen- 2-Tupel
\item[\textit{Description}] Die Gr\"o"enangabe f\"ur das 
Titel- Bildelememt
\item[\textit{Parametre}] \wancicode{size}
\end{enumerate}

\item \wancicode{self.\_\_colormode}
\begin{enumerate}
\item[\textit{Methods}]
\wancicode{Init(...,colormode = ``RGB'', ...)},
\wancicode{SetColormode(colormode = ``RGB'')}
\item[\textit{Valids}] 
\wancicode{``RGB'', ``RGBA'', ``CMYK'',
``1'', ``L'', ``P'', ``XYZ''}
\item[\textit{Description}] Der Farbformat f\"ur das
Titel- Bildelement
\item[\textit{Parametre}] \wancicode{colormode}
\end{enumerate}

\item \wancicode{self.\_\_text}
\begin{enumerate}
\item[\textit{Methods}] \wancicode{Init(...,text = '', ...)},
\wancicode{SetText(text)}
\item[\textit{Valids}] Jeder String
\item[\textit{Description}] Der Text der Titelzeile
\item[\textit{Parametre}] \wancicode{self.\_\_text}
\end{enumerate}

\item \wancicode{self.\_\_color}
\begin{enumerate}
\item[\textit{Methods}] 
\wancicode{Init(...,color = (255,255,255), ...)},
\wancicode{SetColor(color)}
\item[\textit{Valids}] Jede g\"ultige Farbangabe entsprechend
\wancicode{colorformat} 
\item[\textit{Description}] Die Hintergrundfarbe des 
Titelfeldes
\item[\textit{Parametre}] \wancicode{self.\_\_color}
\end{enumerate}

\item \wancicode{self.\_\_textcolor}
\begin{enumerate}
\item[\textit{Methods}] 
\wancicode{Init(...,textcolor = (0,0,0), ...)},
\wancicode{SetTextcolor()}
\item[\textit{Valids}] Jede g\"ultige Farbangabe entsprechend
\wancicode{colorformat}
\item[\textit{Description}] Die Farbe f\"ur den Titeltext
\item[\textit{Parametre}] \wancicode{self.\_\_textcolor}
\end{enumerate}

\item \wancicode{self.\_\_font}
\begin{enumerate}
\item[\textit{Methods}] 
\wancicode{Init(...,font = 'helvB12.pil', ...)},
\wancicode{SetFont(font)}
\item[\textit{Valids}] Jeder g\"ultige Fontpfad
\item[\textit{Description}] Der Font f\"ur den Titeltext
\item[\textit{Parametre}] \wancicode{self.\_\_font}
\end{enumerate}

\end{itemize}

%%%%%%% Sekundaere Attribute %%%%%%%%%%%%%%%%%%%%%%%%%%%

\newpage

\subparagraph{Sekund\"are Attribute}
\begin{itemize}

\item \wancicode{self.\_\_pict}
\begin{enumerate}
\item[\textit{Methods}] \wancicode{Draw()}
\item[\textit{Valids}] keine
\item[\textit{Description}] Das \wancicode{Image}-Objekt der KLasse
\end{enumerate}

\item \wancicode{self.\_\_pic}
\begin{enumerate}
\item[\textit{Methods}] \wancicode{Draw()}
\item[\textit{Valids}] keine
\item[\textit{Description}] Das \wancicode{ImageDraw}-Objekt
zu \wancicode{self.\_\_pict}
\end{enumerate}

\end{itemize}

%%%%%%% Methoden %%%%%%%%%%%%%%%%%%%%%%%%%%%%%%%%%%%%%%%%

\newpage

\subparagraph{Methoden}
\begin{itemize}
  
\item \wancicode{Init()}
\begin{enumerate}
\item[\textit{Arguments}] \wancicode{size, colormode, text, font,
    textcolor, color}
\item[\textit{Description}] Synchrones Setzen aller prim\"aren
  Attribute
\item[\textit{Results}] \wancicode{self.\_\_size, self.\_\_colormode,
    self.\_\_text, self.\_\_color, self.\_\_textcolor, self.\_\_font}
\end{enumerate}

\item \wancicode{SetSize()}
\begin{enumerate}
\item[\textit{Arguments}] \wancicode{size}
\item[\textit{Description}] Setzt die Bildgr\"o"se
\item[\textit{Results}] \wancicode{self.\_\_size}
\end{enumerate}

\item \wancicode{SetColormode()}
\begin{enumerate}
\item[\textit{Arguments}] \wancicode{colormode}
\item[\textit{Description}] Setzt das Farbformat
\item[\textit{Results}] \wancicode{self.\_\_colormode}
\end{enumerate}

\item \wancicode{SetText()}
\begin{enumerate}
\item[\textit{Arguments}] \wancicode{text}
\item[\textit{Description}] Setzt den Titeltext
\item[\textit{Results}] \wancicode{self.\_\_text}
\end{enumerate}

\item \wancicode{SetColor()}
\begin{enumerate}
\item[\textit{Arguments}] \wancicode{color}
\item[\textit{Description}] Setzt die Hintergrundfarbe
\item[\textit{Results}] \wancicode{self.\_\_color}
\end{enumerate}

\item \wancicode{SetTextcolor()}
\begin{enumerate}
\item[\textit{Arguments}] \wancicode{textcolor}
\item[\textit{Description}] Setzt die Textfarbe
\item[\textit{Results}] \wancicode{self.\_\_textcolor}
\end{enumerate}

\item \wancicode{SetFont()}
\begin{enumerate}
\item[\textit{Arguments}] \wancicode{font}
\item[\textit{Description}] Setzt den Textfont f\"ur die Titelzeile
\item[\textit{Results}] \wancicode{self.\_\_font}
\end{enumerate}

\item \wancicode{Draw()}
\begin{enumerate}
\item[\textit{Arguments}] keine
\item[\textit{Description}] Erzeugt ein PIL-Bildobjekt
\item[\textit{Results}] \wancicode{self.\_\_pict},
  \wancicode{self.\_\_pic}
\end{enumerate}

\item \wancicode{GetPicture()}
\begin{enumerate}
\item[\textit{Arguments}] keine
\item[\textit{Description}] Liefert das fertige Titel- Bilobjekt
\item[\textit{Results}] keine
\end{enumerate}

\end{itemize}

%%%%%%%%% OF %%%%%%%%%%%%%%%%%%%%%%%%%%%%%%%%%%%%%%%%%%%%

\newpage

\subparagraph{Objektfunktionalit\"at}

%%%%%%%%%%%%%%%%%%%%%%%%%%%%%%%%%%%%%%%%%%%%%%%%%%%%%%%%%%%%%%
% Legend
%%%%%%%%%%%%%%%%%%%%%%%%%%%%%%%%%%%%%%%%%%%%%%%%%%%%%%%%%%%%%%

\newpage

\paragraph{\wancicode{Legend}}
\subparagraph{Modul} \wancicode{legend.py}

%%%%%%%%%%%%% Parameter %%%%%%%%%%%%%%%%%%%%%%%%%%%%%%%%%

\subparagraph{Parameter}
\begin{itemize}
  
\item \wancicode{x}
\begin{enumerate}
\item[\textit{Methods}] \wancicode{Init(..., x = [0], ...)},
  \wancicode{SetX(x)}
\item[\textit{Valids}] Jede Liste von Zahlen oder Strings
\item[\textit{Default}] \wancicode{[0]}
\item[\textit{Description}] Die Argumentliste
\item[\textit{Attribute}] \wancicode{self.\_\_x}
\end{enumerate}

\item \wancicode{y}
\begin{enumerate}
\item[\textit{Methods}] \wancicode{Init(..., y = [0], ...)},
  \wancicode{SetY(y)}
\item[\textit{Valids}] Jede Liste von Zahlen
\item[\textit{Default}] \wancicode{[0]}
\item[\textit{Description}] Die Werteliste
\item[\textit{Attribute}] \wancicode{self.\_\_y}
\end{enumerate}

\item \wancicode{names}
\begin{enumerate}
\item[\textit{Methods}] \wancicode{Init(..., names = [], ...)}
  \wancicode{SetNames(names)}
\item[\textit{Valids}] Jede Liste von Strings
\item[\textit{Default}] \wancicode{[]}
\item[\textit{Description}] Die Liste von Legendennamen
\item[\textit{Attribute}] \wancicode{self.\_\_names}
\end{enumerate}

\item \wancicode{size}
\begin{enumerate}
\item[\textit{Methods}] \wancicode{Init(..., size = (10,10), ...)},
  \wancicode{SetSize(size)}
\item[\textit{Valids}] Jedes g\"ultige Gr\"o"sen- 2-Tupel
\item[\textit{Default}] \wancicode{(10,10}
\item[\textit{Description}] Die Gr\"o"se des Legenden- Bildobjektes
\item[\textit{Attribute}] \wancicode{self.\_\_size}
\end{enumerate}

\item \wancicode{text}
\begin{enumerate}
\item[\textit{Methods}] \wancicode{Init(..., text = '', ...)},
  \wancicode{SetText(text)}
\item[\textit{Valids}] Jeder String
\item[\textit{Default}] \wancicode{''}
\item[\textit{Description}] Optionaler Text, z.B. f\"ur eine
  Legenden\"uberschrift
\item[\textit{Attribute}] \wancicode{self.\_\_text}
\end{enumerate}

\item \wancicode{color}
\begin{enumerate}
\item[\textit{Methods}] \wancicode{Init(..., color = (255,255,255),
    ...)}, \wancicode{SetColor(color)}
\item[\textit{Valids}] Jede in \"Uberstimmung mit dem Farbformat
  g\"ultige Farbdarstellung
\item[\textit{Default}] \wancicode{(255,255,255)}
\item[\textit{Description}] Die Hintergrundfarbe der Legende
\item[\textit{Attribute}] \wancicode{self.\_\_color}
\end{enumerate}

\item \wancicode{colorlist}
\begin{enumerate}
\item[\textit{Methods}] \wancicode{Init(..., colorlist = [], ...)},
  \wancicode{SetColorlist(colorlist)}
\item[\textit{Valids}] Jede Liste von in \"Ubereinstimmung mit dem
  Farbformat g\"ultige Liste von Farben
\item[\textit{Default}] \wancicode{[]}
\item[\textit{Description}] Die Farbliste zur farblich abgesetzten
  Darstellung der Legendentexte
\item[\textit{Attribute}] \wancicode{colorlist}
\end{enumerate}

\item \wancicode{font}
\begin{enumerate}
\item[\textit{Methods}] \wancicode{Init(..., font =
    '/fonts/helvB12.pil', ...)}, \wancicode{SetFont(font)}
\item[\textit{Valids}] Jeder g\"ultige Fontpfad
\item[\textit{Default}] \wancicode{'/fonts/helvB12.pil'}
\item[\textit{Description}] Der Font f\"ur den Legendentext
\item[\textit{Attribute}] \wancicode{self.\_\_font}
\end{enumerate}

\item \wancicode{textcolor}
\begin{enumerate}
\item[\textit{Methods}] \wancicode{Init(..., textcolor = (0,0,0),
    ...)}, \wancicode{SetTextcolor(textcolor)}
\item[\textit{Valids}] Jede in \"Ubereinstimmung mit dem Farbformat
  g\"ultige Farbe
\item[\textit{Default}] \wancicode{(0,0,0)}
\item[\textit{Description}] Die Farbe zur einheitlichen F\"arbung der
  Legendentexte
\item[\textit{Attribute}] \wancicode{self.\_\_textcolor}
\end{enumerate}

\item \wancicode{colormode}
\begin{enumerate}
\item[\textit{Methods}] \wancicode{Init(..., colormode, ...)},
  \wancicode{SetColormode(colormode)}
\item[\textit{Valids}] \wancicode{``RGB'', ``RGBA'', ``CMYK'', ``1'',
    ``L'', ``P'', ``XYZ''}
\item[\textit{Default}] \wancicode{``RGB''}
\item[\textit{Description}] Der Farbmodus
\item[\textit{Attribute}] \wancicode{self.\_\_colormode}
\end{enumerate}

\end{itemize}

%%%%%%% Primaere Attribute %%%%%%%%%%%%%%%%%%%%%%%%%%%%%%%%%

\newpage

\subparagraph{Prim\"are Attribute}
\begin{itemize}
  
\item \wancicode{self.\_\_x}
\begin{enumerate}
\item[\textit{Methods}] \wancicode{Init(..., x = [0], ...)},
  \wancicode{SetX(x)}
\item[\textit{Description}] Die Argumentliste
\item[\textit{Parametre}] \wancicode{x}
\end{enumerate}

\item \wancicode{self.\_\_y}
\begin{enumerate}
\item[\textit{Methods}] \wancicode{Init(..., y = [0], ...)},
  \wancicode{SetY(y)}
\item[\textit{Description}] Die Werteliste
\item[\textit{Parametre}] \wancicode{y}
\end{enumerate}

\item \wancicode{self.\_\_names}
\begin{enumerate}
\item[\textit{Methods}] \wancicode{Init(..., names = [], ...)}
  \wancicode{SetNames(names)}
\item[\textit{Description}] Die Liste von Legendennamen
\item[\textit{Parametre}] \wancicode{names}
\end{enumerate}

\item \wancicode{self.\_\_size}
\begin{enumerate}
\item[\textit{Methods}] \wancicode{Init(..., size = (10,10), ...)},
  \wancicode{SetSize(size)}
\item[\textit{Description}] Die Gr\"o"se des Legenden- Bildobjektes
\item[\textit{Parametre}] \wancicode{size}
\end{enumerate}

\item \wancicode{self.\_\_font}
\item \wancicode{font}
\begin{enumerate}
\item[\textit{Methods}] \wancicode{Init(..., font =
    '/fonts/helvB12.pil', ...)}, \wancicode{SetFont(font)}
\item[\textit{Description}] Der Font f\"ur den Legendentext
\item[\textit{Parametre}] \wancicode{font}
\end{enumerate}

\item \wancicode{self.\_\_text}
\begin{enumerate}
\item[\textit{Methods}] \wancicode{Init(..., text = '', ...)},
  \wancicode{SetText(text)}
\item[\textit{Description}] Optionaler Text, z.B. f\"ur eine
  Legenden\"uberschrift
\item[\textit{Parametre}] \wancicode{text}
\end{enumerate}

\item \wancicode{self.\_\_textcolor}
\begin{enumerate}
\item[\textit{Methods}] \wancicode{Init(..., textcolor = (0,0,0),
    ...)}, \wancicode{SetTextcolor(textcolor)}
\item[\textit{Description}] Die Farbe zur einheitlichen F\"arbung der
  Legendentexte
\item[\textit{Parametre}] \wancicode{textcolor}
\end{enumerate}

\item \wancicode{self.\_\_color}
\begin{enumerate}
\item[\textit{Methods}] \wancicode{Init(..., color = (255,255,255),
    ...)}, \wancicode{SetColor(color)}
\item[\textit{Description}] Die Hintergrundfarbe der Legende
\item[\textit{Parametre}] \wancicode{color}
\end{enumerate}

\item \wancicode{self.\_\_colorlist}
\begin{enumerate}
\item[\textit{Methods}] \wancicode{Init(..., colorlist = [], ...)},
  \wancicode{SetColorlist(colorlist)}
\item[\textit{Description}] Die Farbliste zur farblich abgesetzten
  Darstellung der Legendentexte
\item[\textit{Parametre}] \wancicode{colorlist}
\end{enumerate}

\item \wancicode{self.\_\_colormode}
\begin{enumerate}
\item[\textit{Methods}] \wancicode{Init(..., colormode, ...)},
  \wancicode{SetColormode(colormode)}
\item[\textit{Description}] Der Farbmodus
\item[\textit{Parametre}] \wancicode{colormode}
\end{enumerate}

\end{itemize}

%%%%%% Sekundaere Attribute %%%%%%%%%%%%%%%%%%%%%%%%%%%%

\newpage

\subparagraph{Sekund\"are Attribute}
\begin{itemize}
  
\item \wancicode{self.\_\_rows}
\begin{enumerate}
\item[\textit{Methods}] \wancicode{Init()}
\item[\textit{Valids}] keine
\item[\textit{Description}] Die Anzahl der Text-Reihen im Legendenfeld
\end{enumerate}

\item \wancicode{self.\_\_cols}
\begin{enumerate}
\item[\textit{Methods}] \wancicode{Init()}
\item[\textit{Valids}] keine
\item[\textit{Description}] Die Anzahl der Text-Spalten im
  Legendenfeld
\end{enumerate}

\end{itemize}

%%%%%%%% Methoden %%%%%%%%%%%%%%%%%%%%%%%%%%%%%%%%%%%%%%%%%

\newpage

\subparagraph{Methoden}
\begin{itemize}
  
\item \wancicode{Init()}
\begin{enumerate}
\item[\textit{Arguments}] \wancicode{x, y, names, size, text,
    colorlist, font, textcolor, colormode}
\item[\textit{Description}] Initialisiert die Attribute auf einmal
\item[\textit{Results}] \wancicode{self.\_\_x, self.\_\_y,
    self.\_\_names, self.\_\_size, self.\_\_text, self.\_\_colorlist,
    self.\_\_font, self.\_\_textcolor, self.\_\_colormode}
\end{enumerate}

\item \wancicode{SetX()}
\begin{enumerate}
\item[\textit{Arguments}] \wancicode{x}
\item[\textit{Description}] Setzt die Argumentliste
\item[\textit{Results}] \wancicode{self.\_\_x}
\end{enumerate}

\item \wancicode{SetY()}
\begin{enumerate}
\item[\textit{Arguments}] \wancicode{y}
\item[\textit{Description}] Setzt die Werteliste
\item[\textit{Results}] \wancicode{self.\_\_y}
\end{enumerate}

\item \wancicode{SetNames()}
\begin{enumerate}
\item[\textit{Arguments}] \wancicode{names}
\item[\textit{Description}] Setzt die Liste von Legendennamen
\item[\textit{Results}] \wancicode{self.\_\_names}
\end{enumerate}

\item \wancicode{SetSize()}
\begin{enumerate}
\item[\textit{Arguments}] \wancicode{size}
\item[\textit{Description}] Setzt das Gr\"o"sentupel
\item[\textit{Results}] \wancicode{self.\_\_size}
\end{enumerate}

\item \wancicode{SetText()}
\begin{enumerate}
\item[\textit{Arguments}] \wancicode{text}
\item[\textit{Description}] Setzt die optionale Textzeile
\item[\textit{Results}] \wancicode{self.\_\_text}
\end{enumerate}

\item \wancicode{SetColor()}
\begin{enumerate}
\item[\textit{Arguments}] \wancicode{color}
\item[\textit{Description}] Setzt die Hintergrundfarbe
\item[\textit{Results}] \wancicode{self.\_\_color}
\end{enumerate}

\item \wancicode{SetTextcolor()}
\begin{enumerate}
\item[\textit{Arguments}] \wancicode{textcolor}
\item[\textit{Description}] Setzt die Einzel-Textfarbe
\item[\textit{Results}] \wancicode{self.\_\_textcolor}
\end{enumerate}

\item \wancicode{SetColorlist()}
\begin{enumerate}
\item[\textit{Arguments}] \wancicode{colorlist}
\item[\textit{Description}] Setzt die Farbliste
\item[\textit{Results}] \wancicode{self.\_\_colorlist}
\end{enumerate}

\item \wancicode{SetFont()}
\begin{enumerate}
\item[\textit{Arguments}] \wancicode{font}
\item[\textit{Description}] Setzt und l\"adt den Textfont
\item[\textit{Results}] \wancicode{self.\_\_font}
\end{enumerate}

\item \wancicode{ListDraw()}
\begin{enumerate}
\item[\textit{Arguments}] keine
\item[\textit{Description}] Erzeugt ein Bildobjekt und zeichnet eine
  Legendenliste
\item[\textit{Results}] \wancicode{self.\_\_pict}
\end{enumerate}

\item \wancicode{NameDraw()}
\begin{enumerate}
\item[\textit{Arguments}] keine
\item[\textit{Description}] Erzeugt ein Bildobjekt und zeichnet eine
  Namenliste
\item[\textit{Results}] \wancicode{self.\_\_pict}
\end{enumerate}

\item \wancicode{GetPicture()}
\begin{enumerate}
\item[\textit{Arguments}] keine
\item[\textit{Description}] Liefert das fertige Legenden- Bildobjekt
\item[\textit{Results}] keine
\end{enumerate}

\end{itemize}

%%%%%%%%%% OF %%%%%%%%%%%%%%%%%%%%%%%%%%%%%%%%%%%%%%%%%%%

\newpage

\subparagraph{Objektfunktionalit\"at}

%%%%%%%%%%%%%%%%%%%%%%%%%%%%%%%%%%%%%%%%%%%%%%%%%%%%%%%%%%%%
% Graphic
%%%%%%%%%%%%%%%%%%%%%%%%%%%%%%%%%%%%%%%%%%%%%%%%%%%%%%%%%%%%

\newpage

\paragraph{\wancicode{Graphic}}
\subparagraph{Modul} \wancicode{graphic.py}

\subparagraph{Parameter}
\begin{itemize}
  
\item \wancicode{x}
\begin{enumerate}
\item[\textit{Methods}] \wancicode{Init(..., x, ...)},
  \wancicode{SetX(x)}
\item[\textit{Valids}] Jede Liste von Strings oder Zahlen
\item[\textit{Default}] keiner
\item[\textit{Description}] Die Argumentliste
\item[\textit{Attribute}] \wancicode{self.\_\_x}
\end{enumerate}

\item \wancicode{y}
\begin{enumerate}
\item[\textit{Methods}] \wancicode{Init(..., y, ...)},
  \wancicode{SetY(y)}
\item[\textit{Valids}] Jede Liste von Zahlen, die in ihrer L\"ange
  \wancicode{x} entspricht
\item[\textit{Default}] keiner
\item[\textit{Description}] Die Werteliste
\item[\textit{Attribute}] \wancicode{self.\_\_y}
\end{enumerate}

\item \wancicode{size}
\begin{enumerate}
\item[\textit{Methods}] \wancicode{Init(..., size = (10,10), ...)},
  \wancicode{SetSize(size)}
\item[\textit{Valids}] Jedes g\"ultige Gr\"o"sen-Tupel
\item[\textit{Default}] \wancicode{(10,10)}
\item[\textit{Description}] Die Gr\"o"senangabe
\item[\textit{Attribute}] \wancicode{self.\_\_size}
\end{enumerate}

\item \wancicode{color}
\begin{enumerate}
\item[\textit{Methods}] \wancicode{Init(..., color = (255,255,255),
    ...)}, \wancicode{SetColor()}
\item[\textit{Valids}] Jede g\"ultige Farbangabe in \"ubereinstimmung
  mit dem Farbformat
\item[\textit{Default}] \wancicode{(255,255,255)}
\item[\textit{Description}] Die Hintergrundfarbe
\item[\textit{Attribute}] \wancicode{self.\_\_color}
\end{enumerate}

\item \wancicode{light}
\begin{enumerate}
\item[\textit{Methods}] \wancicode{Init(..., light = [], ...)},
  \wancicode{SetLight(light)}
\item[\textit{Valids}] Jede Farbliste in \"Ubereinstimmung mir dem
  Farbformat
\item[\textit{Default}] \wancicode{[]}
\item[\textit{Description}] die helle Farbliste
\item[\textit{Attribute}] \wancicode{self.\_\_light}
\end{enumerate}

\item \wancicode{medium}
\begin{enumerate}
\item[\textit{Methods}] \wancicode{Init(..., medium = [], ...)},
  \wancicode{SetMedium(medium)}
\item[\textit{Valids}] Jede Farbliste in \"Ubereinstimmung mir dem
  Farbformat
\item[\textit{Default}] \wancicode{[]}
\item[\textit{Description}] Die mittlere Farbliste
\item[\textit{Attribute}] \wancicode{self.\_\_medium}
\end{enumerate}

\item \wancicode{dark}
\begin{enumerate}
\item[\textit{Methods}] \wancicode{Init(..., dark = [], ...)},
  \wancicode{SetDark(dark)}
\item[\textit{Valids}] Jede Farbliste in \"Ubereinstimmung mir dem
  Farbformat
\item[\textit{Default}] \wancicode{[]}
\item[\textit{Description}] Die dunkle Farbliste
\item[\textit{Attribute}] \wancicode{self.\_\_dark}
\end{enumerate}

\item \wancicode{textcolor}
\begin{enumerate}
\item[\textit{Methods}] \wancicode{Init(..., textcolor = (0,0,0),
    ...)}, \wancicode{SetTextcolor(textcolor)}
\item[\textit{Valids}] Jede Farbdarstellung in \"Ubereinstimmung mit
  dem Farbformat
\item[\textit{Default}] \wancicode{(0,0,0)}
\item[\textit{Description}] Die Farbe f\"ur die Beschriftungen
\item[\textit{Attribute}] \wancicode{self.\_\_textcolor}
\end{enumerate}

\item \wancicode{font}
\begin{enumerate}
\item[\textit{Methods}] \wancicode{Init(..., font =
    '/fonts/helvB12.pil', ...)}, \wancicode{SetFont(font)}
\item[\textit{Valids}] Jeder g\"ultige Fontpfad
\item[\textit{Default}] \wancicode{/fonts/helvB08.pil'}
\item[\textit{Description}] Der Font f\"ur die Beschriftungen
\item[\textit{Attribute}] \wancicode{self.\_\_font}
\end{enumerate}

\item \wancicode{linecolor}
\begin{enumerate}
\item[\textit{Methods}] \wancicode{Init(..., linecolor = (0,0,0),
    ...)}, \wancicode{SetLinecolor(linecolor)}
\item[\textit{Valids}] Jede Farbdarstellung in \"Ubereinstimmung mit
  dem Farbformat
\item[\textit{Default}] \wancicode{(0,0,0)}
\item[\textit{Description}] Die Farbe f\"ur Koordinatenlinien
\item[\textit{Attribute}] \wancicode{self.\_\_linecolor}
\end{enumerate}

\item \wancicode{backlinecolor}
\begin{enumerate}
\item[\textit{Methods}] \wancicode{Init(..., backlinecolor =
    (150,150,150), ...)}, \wancicode{SetBacklinecolor(backlinecolor)}
\item[\textit{Valids}] Jede Farbdarstellung in \"Ubereinstimmung mit
  dem Farbformat
\item[\textit{Default}] \wancicode{(150,150,150)}
\item[\textit{Description}] Die Farbe f\"ur perspektivisch
  zur\"uckgesetzte Koordinatenlinien
\item[\textit{Attribute}] \wancicode{self.\_\_backlinecolor}
\end{enumerate}

\item \wancicode{mc}
\begin{enumerate}
\item[\textit{Methods}] \wancicode{Init(..., mc = 1.1, ...)},
  \wancicode{SetMc(mc)}
\item[\textit{Valids}] Jede positive Zahl
\item[\textit{Default}] \wancicode{1.1}
\item[\textit{Description}] \"Uberschu"skoeffizient f\"ur die H\"ohe
  der y-Achse
\item[\textit{Attribute}] \wancicode{self.\_\_mc}
\end{enumerate}

\item \wancicode{devc}
\begin{enumerate}
\item[\textit{Methods}] \wancicode{Init(..., devc = (1,1), ...)},
  \wancicode{SetDevc(devc)}
\item[\textit{Valids}] Jedes 2-Tupel von positiven Zahlen
\item[\textit{Default}] \wancicode{(1,1)}
\item[\textit{Description}] Koeefizientenpaar f\"ur die
  perspektivische Verk\"urzung von 3D-Objekten
\item[\textit{Attribute}] \wancicode{self.\_\_devc}
\end{enumerate}

\item \wancicode{xres}
\begin{enumerate}
\item[\textit{Methods}] \wancicode{Init(..., xres = 1, ...)},
  \wancicode{SetXres(xres)}
\item[\textit{Valids}] Jede positive Zahl
\item[\textit{Default}] \wancicode{1}
\item[\textit{Description}] Aufl\"osung der x-Achsen-Rasterung
\item[\textit{Attribute}] \wancicode{self.\_\_xres}
\end{enumerate}

\item \wancicode{yres}
\begin{enumerate}
\item[\textit{Methods}] \wancicode{Init(..., yres = 1, ...)},
  \wancicode{SetYres(yres)}
\item[\textit{Valids}] Jede positive Zahl
\item[\textit{Default}] \wancicode{1}
\item[\textit{Description}] Aufl\"osung der y-Achsen-Rasterung
\item[\textit{Attribute}] \wancicode{self.\_\_yres}
\end{enumerate}

\item \wancicode{xname}
\begin{enumerate}
\item[\textit{Methods}] \wancicode{Init(..., xname = '', ...)},
  \wancicode{SetXname(xname)}
\item[\textit{Valids}] Jeder String
\item[\textit{Default}] \wancicode{''}
\item[\textit{Description}] x-Achsen-Beschriftung
\item[\textit{Attribute}] \wancicode{self.\_\_xname}
\end{enumerate}

\item \wancicode{yname}
\begin{enumerate}
\item[\textit{Methods}] \wancicode{Init(..., yname = '', ...)},
  \wancicode{SetYname(yname)}
\item[\textit{Valids}] Jeder String
\item[\textit{Default}] \wancicode{''}
\item[\textit{Description}] y-Achsen-Beschriftung
\item[\textit{Attribute}] \wancicode{self.\_\_yname}
\end{enumerate}

\item \wancicode{colormode}
\begin{enumerate}
\item[\textit{Methods}] \wancicode{Init(..., colormode = ``RGB'',
    ...)}, \wancicode{SetColormode(colormode)}
\item[\textit{Valids}] \wancicode{``RGB'', ``RGBA'', ``CMYK'', ``1'',
    ``P'', ``L'', ``XYZ''}
\item[\textit{Default}] \wancicode{``RGB''}
\item[\textit{Description}] Das Farbformat
\item[\textit{Attribute}] \wancicode{self.\_\_colorformat}
\end{enumerate}

\end{itemize}

%%%%%%%% Primaere Attribute %%%%%%%%%%%%%%%%%%%%%%%%%%%%%%%%%

\newpage

\subparagraph{Prim\"are Attribute}
\begin{itemize}
  
\item \wancicode{self.\_\_x}
\begin{enumerate}
\item[\textit{Methods}] \wancicode{Init(..., x, ...)},
  \wancicode{SetX(x)}
\item[\textit{Description}] Die Argumentliste
\item[\textit{Parametre}] \wancicode{x}
\end{enumerate}

\item \wancicode{self.\_\_y}
\begin{enumerate}
\item[\textit{Methods}] \wancicode{Init(..., y, ...)},
  \wancicode{SetY(y)}
\item[\textit{Description}] Die Werteliste
\item[\textit{Parametre}] \wancicode{y}
\end{enumerate}

\item \wancicode{self.\_\_size}
\begin{enumerate}
\item[\textit{Methods}] \wancicode{Init(..., size = (10,10), ...)},
  \wancicode{SetSize(size)}
\item[\textit{Description}] Die Gr\"o"senangabe
\item[\textit{Parametre}] \wancicode{size}
\end{enumerate}

\item \wancicode{self.\_\_colormode}
\begin{enumerate}
\item[\textit{Methods}] \wancicode{Init(..., colormode = ``RGB'',
    ...)}, \wancicode{SetColormode(colormode)}
\item[\textit{Description}] Das Farbformat
\item[\textit{Parametre}] \wancicode{colorformat}
\end{enumerate}

\item \wancicode{self.\_\_color}
\begin{enumerate}
\item[\textit{Methods}] \wancicode{Init(..., color = (255,255,255),
    ...)}, \wancicode{SetColor()}
\item[\textit{Description}] Die Hintergrundfarbe
\item[\textit{Parametre}] \wancicode{color}
\end{enumerate}

\item \wancicode{self.\_\_light}
\begin{enumerate}
\item[\textit{Methods}] \wancicode{Init(..., light = [], ...)},
  \wancicode{SetLight(light)}
\item[\textit{Description}] die helle Farbliste
\item[\textit{Parametre}] \wancicode{light}
\end{enumerate}

\item \wancicode{self.\_\_medium}
\begin{enumerate}
\item[\textit{Methods}] \wancicode{Init(..., medium = [], ...)},
  \wancicode{SetMedium(medium)}
\item[\textit{Description}] Die mittlere Farbliste
\item[\textit{Parametre}] \wancicode{medium}
\end{enumerate}

\item \wancicode{self.\_\_dark}
\begin{enumerate}
\item[\textit{Methods}] \wancicode{Init(..., dark = [], ...)},
  \wancicode{SetDark(dark)}
\item[\textit{Description}] Die dunkle Farbliste
\item[\textit{Parametre}] \wancicode{dark}
\end{enumerate}

\item \wancicode{self.\_\_textcolor}
\begin{enumerate}
\item[\textit{Methods}] \wancicode{Init(..., textcolor = (0,0,0),
    ...)}, \wancicode{SetTextcolor(textcolor)}
\item[\textit{Description}] Die Farbe f\"ur die Beschriftungen
\item[\textit{Parametre}] \wancicode{textcolor}
\end{enumerate}

\item \wancicode{self.\_\_font}
\begin{enumerate}
\item[\textit{Methods}] \wancicode{Init(..., font =
    '/fonts/helvB08.pil', ...)}, \wancicode{SetTextcolor(textcolor)}
\item[\textit{Description}] Der Font f\"ur die Beschriftungen
\item[\textit{Parametre}] \wancicode{font}
\end{enumerate}

\item \wancicode{self.\_\_linecolor}
\begin{enumerate}
\item[\textit{Methods}] \wancicode{Init(..., linecolor = (0,0,0),
    ...)}, \wancicode{SetLinecolor(linecolor)}
\item[\textit{Description}] Die Farbe f\"ur Koordinatenlinien
\item[\textit{Parametre}] \wancicode{linecolor}
\end{enumerate}

\item \wancicode{self.\_\_backlinecolor}
\begin{enumerate}
\item[\textit{Methods}] \wancicode{Init(..., backlinecolor =
    (150,150,150), ...)}, \wancicode{SetBacklinecolor(backlinecolor)}
\item[\textit{Description}] Die Farbe f\"ur perspektivisch
  zur\"uckgesetzte Koordinatenlinien
\item[\textit{Parametre}] \wancicode{backlinecolor}
\end{enumerate}

\item \wancicode{self.\_\_mc}
\begin{enumerate}
\item[\textit{Methods}] \wancicode{Init(..., mc = 1.1, ...)},
  \wancicode{SetMc(mc)}
\item[\textit{Description}] \"Uberschu"skoeffizient f\"ur die H\"ohe
  der y-Achse
\item[\textit{Parametre}] \wancicode{mc}
\end{enumerate}

\item \wancicode{self.\_\_devc}
\begin{enumerate}
\item[\textit{Methods}] \wancicode{Init(..., devc = (1,1), ...)},
  \wancicode{SetDevc(devc)}
\item[\textit{Description}] Koeefizientenpaar f\"ur die
  perspektivische Verk\"urzung von 3D-Objekten
\item[\textit{Parametre}] \wancicode{devc}
\end{enumerate}

\item \wancicode{self.\_\_xname}
\begin{enumerate}
\item[\textit{Methods}] \wancicode{Init(..., xname = '', ...)},
  \wancicode{SetXname(xname)}
\item[\textit{Description}] x-Achsen-Beschriftung
\item[\textit{Parametre}] \wancicode{xname}
\end{enumerate}

\item \wancicode{self.\_\_yname}
\begin{enumerate}
\item[\textit{Methods}] \wancicode{Init(..., yname = '', ...)},
  \wancicode{SetYname(yname)}
\item[\textit{Description}] y-Achsen-Beschriftung
\item[\textit{Parametre}] \wancicode{yname}
\end{enumerate}

\item \wancicode{self.\_\_xres}
\begin{enumerate}
\item[\textit{Methods}] \wancicode{Init(..., xres = 1, ...)},
  \wancicode{SetXres(xres)}
\item[\textit{Description}] Aufl\"osung der x-Achsen-Rasterung
\item[\textit{Parametre}] \wancicode{xres}
\end{enumerate}

\item \wancicode{self.\_\_yres}
\begin{enumerate}
\item[\textit{Methods}] \wancicode{Init(..., yres = 1, ...)},
  \wancicode{SetYres(yres)}
\item[\textit{Description}] Aufl\"osung der y-Achsen-Rasterung
\item[\textit{Parametre}] \wancicode{yres}
\end{enumerate}

\end{itemize}

%%%%%%%% Sekundaere Attribute %%%%%%%%%%%%%%%%%%%%%%%%%%%%%%%%

\newpage

\subparagraph{Sekund\"are Attribute}
\begin{itemize}
  
\item \wancicode{self.\_\_ymax}
\begin{enumerate}
\item[\textit{Methods}] \wancicode{Init()}, \wancicode{SetYmax()}
\item[\textit{Valids}] Jede positive Zahl
\item[\textit{Description}] Maximum der dargestellten y-Werte
\end{enumerate}

\item \wancicode{self.\_\_dev}
\begin{enumerate}
\item[\textit{Methods}] \wancicode{SetUp()}
\item[\textit{Description}] Explizite perspektivische Abweichung
\end{enumerate}

\item \wancicode{self.\_\_namexmax}
\begin{enumerate}
\item[\textit{Methods}] \wancicode{SetUp()}
\item[\textit{Description}] Maximum der horizontalen Ausdehnungen der
  y-Achsen-Beschriftungen
\end{enumerate}

\item \wancicode{self.\_\_leftmargin}
\begin{enumerate}
\item[\textit{Methods}] \wancicode{SetUp()}
\item[\textit{Description}] Offset der y-Achse vom linken Bildrand
\end{enumerate}

\item \wancicode{self.\_\_textscale}
\begin{enumerate}
\item[\textit{Methods}] \wancicode{SetUp()}
\item[\textit{Description}] Feldh\"ohe f\"ur die Textfelder unter der
  x-Achse
\end{enumerate}

\item \wancicode{self.\_\_xwidth}
\begin{enumerate}
\item[\textit{Methods}] \wancicode{SetUp()}
\item[\textit{Description}] Zur Verf\"ugung stehende Breite
\end{enumerate}

\item \wancicode{self.\_\_xscale}
\begin{enumerate}
\item[\textit{Methods}] \wancicode{SetUp()}
\item[\textit{Description}] Einheit der x-Achsen-Rasterung
\end{enumerate}

\item \wancicode{self.\_\_yheight}
\begin{enumerate}
\item[\textit{Methods}] \wancicode{SetUp()}
\item[\textit{Description}] Gesamth\"ohe der y-Achse
\end{enumerate}

\item \wancicode{self.\_\_yscale}
\begin{enumerate}
\item[\textit{Methods}] \wancicode{SetUp()}
\item[\textit{Description}] Einheit der y-Achsen-Rasterung
\end{enumerate}

\item \wancicode{self.\_\_yunit}
\begin{enumerate}
\item[\textit{Methods}] \wancicode{SetUp()}
\item[\textit{Description}] Absolute Einheitsl\"ange der
  x-Achsen-Rasterung
\end{enumerate}

\item \wancicode{self.\_\_ysteps}
\begin{enumerate}
\item[\textit{Methods}] \wancicode{SetUp()}
\item[\textit{Description}] y-Achsen -Rasterungsschritte
\end{enumerate}

\item \wancicode{self.\_\_xunit}
\begin{enumerate}
\item[\textit{Methods}] \wancicode{GraphMeasures()},
  \wancicode{ColumnMeasures()}
\item[\textit{Description}] Absolute Einheitsl\"ange der
  y-Achsen-Rasterung
\end{enumerate}

\item \wancicode{self.\_\_zero}
\begin{enumerate}
\item[\textit{Methods}] \wancicode{GraphMeasures()},
  \wancicode{ColumnMeasures()}
\item[\textit{Description}] Koordinaten des Nullpunktes des
  Koordinatensystems
\end{enumerate}

\item \wancicode{self.\_\_xaxis}
\begin{enumerate}
\item[\textit{Methods}] \wancicode{GraphMeasures()},
  \wancicode{ColumnMeasures()}
\item[\textit{Description}] Koordinaten der Endpunkte der x-Achse
\end{enumerate}

\item \wancicode{self.\_\_xfields}
\begin{enumerate}
\item[\textit{Methods}] \wancicode{GraphMeasures()},
  \wancicode{ColumnMeasures()}
\item[\textit{Description}] Koordinaten der Einheits-Rasterpunkte der
  x-Achse
\end{enumerate}

\item \wancicode{self.\_\_xmarks}
\begin{enumerate}
\item[\textit{Methods}] \wancicode{GraphMeasures()},
  \wancicode{ColumnMeasures()}
\item[\textit{Description}] Koordinaten der tats\"achlichen
  Rasterpunkte der x-Achse
\end{enumerate}

\item \wancicode{self.\_\_xsteps}
\begin{enumerate}
\item[\textit{Methods}] \wancicode{GraphMeasures()},
  \wancicode{ColumnMeasures()}
\item[\textit{Description}] ???
\end{enumerate}

\item \wancicode{self.\_\_yaxis}
\begin{enumerate}
\item[\textit{Methods}] \wancicode{GraphMeasures()},
  \wancicode{ColumnMeasures()}
\item[\textit{Description}] Koordinaten der Endpunkte der y-Achse
\end{enumerate}

\item \wancicode{self.\_\_ypoints}
\begin{enumerate}
\item[\textit{Methods}] \wancicode{GraphMeasures()},
  \wancicode{ColumnMeasures()}
\item[\textit{Description}] Rasterpunkte der x-Achse
\end{enumerate}

\item \wancicode{self.\_\_ysum}
\begin{enumerate}
\item[\textit{Methods}] \wancicode{SetYsum()}
\item[\textit{Description}] Additionsgr\"o"se f\"ur mehrfache
  \wancicode{ymax}
\end{enumerate}

\end{itemize}

%%%%%%% Methoden %%%%%%%%%%%%%%%%%%%%%%%%%%%%%%%%%%%%%

\newpage

\subparagraph{Methoden}
\begin{itemize}
  
\item \wancicode{Init()}
\begin{enumerate}
\item[\textit{Arguments}] \wancicode{x, y, size, color, light, medium,
    dark, textcolor, font, linecolor, backlinecolor, mc, devc, xres,
    yres, xname, ynamem, colormode}
\item[\textit{Description}] Initialisiert alle prim\"aren Attribute
  auf einmal
\item[\textit{Results}] \wancicode{self.\_\_x, self.\_\_y,
    self.\_\_ymax, self.\_\_size, self.\_\_colormode, self.\_\_color,
    self.\_\_light, self.\_\_medium, self.\_\_dark,
    self.\_\_textcolor, self.\_\_font, self.\_\_linecolor,
    self.\_\_backlinecolor, self.\_\_mc, self.\_\_devc,
    self.\_\_xname, self.\_\_yname, self.\_\_xres, self.\_\_yres}
\end{enumerate}

\item \wancicode{SetUp()}
\begin{enumerate}
\item[\textit{Arguments}] keine
\item[\textit{Description}] Setzt verschiedene sekund\"are Attribute
\item[\textit{Results}] \wancicode{self.\_\_dev, self.\_\_namexmax,
    self.\_\_leftmargin, self.\_\_textscale, self.\_\_xwidth,
    self.\_\_xscale, self.\_\_yheight, self.\_\_yscale,
    self.\_\_yunit, self.\_\_ysteps}
\end{enumerate}

\item \wancicode{SetX()}
\begin{enumerate}
\item[\textit{Arguments}] \wancicode{x}
\item[\textit{Description}] Setzt die Argumentliste
\item[\textit{Results}] \wancicode{self.\_\_x}
\end{enumerate}

\item \wancicode{SetY()}
\begin{enumerate}
\item[\textit{Arguments}] \wancicode{y}
\item[\textit{Description}] Setzt die Werteliste
\item[\textit{Results}] \wancicode{self.\_\_y}
\end{enumerate}

\item \wancicode{SetSize()}
\begin{enumerate}
\item[\textit{Arguments}] \wancicode{size}
\item[\textit{Description}] Setzt die Bildgr\"o"se
\item[\textit{Results}] \wancicode{self.\_\_size}
\end{enumerate}

\item \wancicode{SetYmax()}
\begin{enumerate}
\item[\textit{Arguments}] \wancicode{ymax}
\item[\textit{Description}] Setzt das y-Maximum abweichend vom
  tats\"achlichen Wert
\item[\textit{Results}] \wancicode{self.\_\_ymax}
\end{enumerate}

\item \wancicode{SetYsum()}
\begin{enumerate}
\item[\textit{Arguments}] \wancicode{ysum}
\item[\textit{Description}] Setzt die Summe mehrerer \wancicode{ymax}
  abweichend vom tats\"achlichen Wert
\item[\textit{Results}] \wancicode{self.\_\_ysum}
\end{enumerate}

\item \wancicode{SetColormode()}
\begin{enumerate}
\item[\textit{Arguments}] \wancicode{colormode}
\item[\textit{Description}] Setzt das Farbformat
\item[\textit{Results}] \wancicode{self.\_\_colormode}
\end{enumerate}

\item \wancicode{SetColor()}
\begin{enumerate}
\item[\textit{Arguments}] \wancicode{color}
\item[\textit{Description}] Setzt die Hintergrundfarbe
\item[\textit{Results}] \wancicode{self.\_\_color}
\end{enumerate}

\item \wancicode{SetLight()}
\begin{enumerate}
\item[\textit{Arguments}] \wancicode{light}
\item[\textit{Description}] Setzt die helle Farbliste
\item[\textit{Results}] \wancicode{self.\_\_light}
\end{enumerate}

\item \wancicode{SetMedium()}
\begin{enumerate}
\item[\textit{Arguments}] \wancicode{medium}
\item[\textit{Description}] Setzt die mittlere Farbliste
\item[\textit{Results}] \wancicode{self.\_\_medium}
\end{enumerate}

\item \wancicode{SetDark()}
\begin{enumerate}
\item[\textit{Arguments}] \wancicode{dark}
\item[\textit{Description}] Setzt die dunkle Farbliste
\item[\textit{Results}] \wancicode{self.\_\_dark}
\end{enumerate}

\item \wancicode{SetTextcolor()}
\begin{enumerate}
\item[\textit{Arguments}] \wancicode{textcolor}
\item[\textit{Description}] Setzt die Farbe der Beschriftungen
\item[\textit{Results}] \wancicode{self.\_\_textcolor}
\end{enumerate}

\item \wancicode{SetFont()}
\begin{enumerate}
\item[\textit{Arguments}] \wancicode{font}
\item[\textit{Description}] Setzt und l\"adt den Font der
  Beschriftungen
\item[\textit{Results}] \wancicode{self.\_\_font}
\end{enumerate}

\item \wancicode{SetLinecolor()}
\begin{enumerate}
\item[\textit{Arguments}] \wancicode{linecolor}
\item[\textit{Description}] Setzt die Farbe der Koordinatenlinien
\item[\textit{Results}] \wancicode{self.\_\_linecolor}
\end{enumerate}

\item \wancicode{SetBacklinecolor()}
\begin{enumerate}
\item[\textit{Arguments}] \wancicode{backlinecolor}
\item[\textit{Description}] Setzt die Farbe der perspektivisch
  verschobenen Koordinatenlinien
\item[\textit{Results}] \wancicode{self.\_\_backlincolor}
\end{enumerate}

\item \wancicode{SetMc()}
\begin{enumerate}
\item[\textit{Arguments}] \wancicode{mc}
\item[\textit{Description}] Setzt der Overspill-Koeffizienten
\item[\textit{Results}] \wancicode{self.\_\_mc}
\end{enumerate}

\item \wancicode{SetDevc()}
\begin{enumerate}
\item[\textit{Arguments}] \wancicode{devc}
\item[\textit{Description}] Setzt den Koeffizienten der
  perspektivischen Verk\"urzung
\item[\textit{Results}] \wancicode{self.\_\_devc}
\end{enumerate}

\item \wancicode{SetDev()}
\begin{enumerate}
\item[\textit{Arguments}] \wancicode{dev}
\item[\textit{Description}] Setzt die Perspektivische Verschiebung,
  abweichend vom eigentlichen Wert
\item[\textit{Results}] \wancicode{self.\_\_dev}
\end{enumerate}

\item \wancicode{SetXname()}
\begin{enumerate}
\item[\textit{Arguments}] \wancicode{xname}
\item[\textit{Description}] Setzt die Beschriftung der x-Achse
\item[\textit{Results}] \wancicode{self.\_\_xname}
\end{enumerate}

\item \wancicode{SetYname()}
\begin{enumerate}
\item[\textit{Arguments}] \wancicode{yname}
\item[\textit{Description}] Setzt die Beschriftung der y-Achse
\item[\textit{Results}] \wancicode{self.\_\_yname}
\end{enumerate}

\item \wancicode{SetXres()}
\begin{enumerate}
\item[\textit{Arguments}] \wancicode{xres}
\item[\textit{Description}] Setzt die Aufl\"osung der x-Achsen
  Rasterung
\item[\textit{Results}] \wancicode{self.\_\_xres}
\end{enumerate}

\item \wancicode{SetYres()}
\begin{enumerate}
\item[\textit{Arguments}] \wancicode{yres}
\item[\textit{Description}] Setzt die Aufl\"osung der y-Achsen
  Rasterung
\item[\textit{Results}] \wancicode{self.\_\_yres}
\end{enumerate}

\item \wancicode{ColumnMeasures()}
\begin{enumerate}
\item[\textit{Arguments}] keine
\item[\textit{Description}] Setzt verschiedene sekund\"are Attribute
  abh\"angig von der Darstellungsform
\item[\textit{Results}] \wancicode{self.\_\_xunit, self.\_\_zero,
    self.\_\_xaxis, self.\_\_xfields, self.\_\_xmarks,
    self.\_\_xsteps, self.\_\_yaxis, self.\_\_ypoints}
\end{enumerate}

\item \wancicode{GraphMeasures()}
\begin{enumerate}
\item[\textit{Arguments}] keine
\item[\textit{Description}] Setzt verschiedene sekund\"are Attribute
  abh\"angig von der Darstellungsform
\item[\textit{Results}] \wancicode{self.\_\_xunit, self.\_\_zero,
    self.\_\_xaxis, self.\_\_xfields, self.\_\_xmarks,
    self.\_\_xsteps, self.\_\_yaxis, self.\_\_ypoints}
\end{enumerate}

\item \wancicode{Draw()}
\begin{enumerate}
\item[\textit{Arguments}] keine
\item[\textit{Description}] Erzeugt ein Bildobjekt
\item[\textit{Results}] \wancicode{self.\_\_pict},
  \wancicode{self.\_\_pic}
\end{enumerate}

\item \wancicode{GetPicture()}
\begin{enumerate}
\item[\textit{Arguments}] keine
\item[\textit{Description}] Liefert das fertige Bildobjekt
\item[\textit{Results}] keine
\end{enumerate}

\item \wancicode{BackgroundLining()}
\begin{enumerate}
\item[\textit{Arguments}] keine
\item[\textit{Description}] Zeichnet die Hintergrundlinien des
  Koordinatensystems
\item[\textit{Results}] keine
\end{enumerate}

\item \wancicode{Points(col)}
\begin{enumerate}
\item[\textit{Arguments}] keine
\item[\textit{Description}] Zeichnet einen Punktgraph mit Punkten in
  der Farbe \wancicode{col}
\item[\textit{Results}] keine
\end{enumerate}

\item \wancicode{Line(col)}
\begin{enumerate}
\item[\textit{Arguments}] keine
\item[\textit{Description}] Zeichnet einen Kurvengraph mit der Kurve
  in der Farbe \wancicode{col}
\item[\textit{Results}] keine
\end{enumerate}

\item \wancicode{Plane(co)}
\begin{enumerate}
\item[\textit{Arguments}] keine
\item[\textit{Description}] Zeichnet einen ausgef\"ullten Kurvengraph
  in der Farbe \wancicode{col}
\item[\textit{Results}] keine
\end{enumerate}

\item \wancicode{CubicColumns()}
\begin{enumerate}
\item[\textit{Arguments}] keine
\item[\textit{Description}] Zeichnet Quader als S\"aulendiagramm
\item[\textit{Results}] keine
\end{enumerate}

\item \wancicode{CylindricColumns()}
\begin{enumerate}
\item[\textit{Arguments}] keine
\item[\textit{Description}] Zeichnet Zylinder als S\"aulendiagramm
\item[\textit{Results}] keine
\end{enumerate}

\item \wancicode{Multicolumn()}
\begin{enumerate}
\item[\textit{Arguments}] keine
\item[\textit{Description}] Zeichnet eine einzelne geschichtete
  S\"aule
\item[\textit{Results}] keine
\end{enumerate}

\item \wancicode{Pie()}
\begin{enumerate}
\item[\textit{Arguments}] keine
\item[\textit{Description}] Zeichnet ein Tortendiagramm
\item[\textit{Results}] keine
\end{enumerate}

\item \wancicode{StandardCoordinates()}
\begin{enumerate}
\item[\textit{Arguments}] keine
\item[\textit{Description}] Zeichnet ein Koordiatensystem f\"ur
  Graphen
\item[\textit{Results}] keine
\end{enumerate}

\item \wancicode{EnumeratedColumns()}
\begin{enumerate}
\item[\textit{Arguments}] keine
\item[\textit{Description}] Zeichnet ein Koordinatensystem f\"ur
  S\"aulen und numeriert die Standpl\"atze
\item[\textit{Results}] keine
\end{enumerate}

\item \wancicode{StandardColumns()}
\begin{enumerate}
\item[\textit{Arguments}] keine
\item[\textit{Description}] Zeichnet ein Koordinatensystem f\"ur
  S\"aulen und schreibt die Argumente an die Standpl\"atze
\item[\textit{Results}] keine
\end{enumerate}

\item \wancicode{ExtendedColumns()} z.Zt. funktionslos
%\begin{enumerate}
%\item[\textit{Arguments}] 
%\item[\textit{Description}] 
%\item[\textit{Results}] 
%\end{enumerate}

\end{itemize}

%%%%%%%%%% OF %%%%%%%%%%%%%%%%%%%%%%%%%%%%%%%%%%%%%%%%%%

\newpage

\subparagraph{Objektfunktionalit\"at}

%%%%%%%%%%%%%%%%%%%%%%%%%%%%%%%%%%%%%%%%%%%%%%%%%%%%%%%%%%%%%%%%%
%%%%%%%%%%%%%%%%%%%%%%%%%%%%%%%%%%%%%%%%%%%%%%%%%%%%%%%%%%%%%%%%%

\newpage

\subsection{Syntax}

%%%%%%%%%%%%%%%%%%%%%%%%%%%%%%%%%%%%%%%%%%%%%%%%%%%%%%%%%%%%%%%%%
\subsection{Sonstiges}

\paragraph{Empfohlene Einstellungen}

\paragraph{Bemerkungen}
%%  SDV REFMAN ENDE %%
