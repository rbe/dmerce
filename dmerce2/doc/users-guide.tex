%%%
\part{Users Guide}

%%
\chapter{Der Ablauf}

Dieses Kapitel beschreibt den Ablauf eines Requests und Reply von
dmerce. Derzeit basiert dmerce auf dem Konzept des Common Gateway
Interface. Dies bedeutet ...

\newpage
\section{Aufbau und Struktur eines Aufrufes}

Ein dmerce Aufruf per Hyperlink sieht z. B. wie folgt aus:

\wancicodeblock{/1Ci/dmerce?qTemplate=Hallo,dmerce,Welt\&a=1\&b=2\&c=3}

Die Bedeutung der einzelnen Parameter:

\begin{itemize}
\item \wancicode{/1Ci/dmerce?} leitet die folgenden Parameter an
  dmerce weiter, das Fragezeichen bezeichnet hierbei den Beginn der
  \"ubergebenen Parameter. Achtung: das Fragezeichen darf nur einmal
  benutzt werden!
\item \wancicode{qTemplate=Hallo,dmerce,Welt} \"ubergibt dmerce den
  Parameter qTemplate mit dem Wert Hallo,dmerce,Welt, was dmerce sagt,
  dass das Template \wancicode{templates/Hallo/dmerce/Welt.html}
  genutzt werden soll \wancisee{Templates}.
\item \wancicode{\&} bezeichnet die Trennung zwischen verschiedenen
  Parametern und sollte nur dann verwendet werden, wenn es noch
  weitere Parameter gibt
\item \wancicode{a=1} \"ubergibt dmerce den Parameter a mit dem Wert
  1, hierbei handelt es sich um einen willk\"urlichen Parameter und
  Wert, den Sie aus eigenem Antrieb heraus definiert haben
\item \wancicode{b=2} \"ubergibt dmerce den Parameter b mit dem Wert
  2, hierbei handelt es sich um einen willk\"urlichen Parameter und
  Wert, den Sie aus eigenem Antrieb heraus definiert haben
\item \wancicode{c=3} \"ubergibt dmerce den Parameter c mit dem Wert
  3, hierbei handelt es sich um einen willk\"urlichen Parameter und
  Wert, den Sie aus eigenem Antrieb heraus definiert haben
\end{itemize}

\medskip

Das gleiche Beispiel per Formular:

\wancicodeblock{
  <form method=''post'' action=''/1Ci/dmerce''>\\
  <input type=''hidden'' name=''qTemplate'' value=''Hallo,dmerce,Welt''>\\
  <input type=''hidden'' name=''a'' value=''1''>\\
  <input type=''hidden'' name=''b'' value=''2''>\\
  <input type=''hidden'' name=''c'' value=''3''>\\
  </form>
}

\section{Alles der Reihe nach}

\begin{enumerate}
\item Entgegennehmen, parsen und auswerten des Requests
\item Auswerten der privaten Variablen
\item Einlesen und Auswerten des Templates
\item Ggf. pr\"ufen SAM und UXS den Request, sofern es keine
  Beanstandungen gibt erfolgt die weitere Ausf\"uhrung
\item Ausf\"uhren von Triggern (per \wancicode{qTrigger})
\item Datenbankmanipulation
\item Ausf\"uhren von Triggern (per \wancicode{qTriggerAfter})
\item Ausgabe
\end{enumerate}

\chapter{Datenbanken und Tabellen}

\section{Tabellen}

Folgende Felder/Feldnamen gelten in einer dmerce-Umgebung als
reserviert:

\medskip

\begin{tabular*}{128mm}{p{35mm} p{50mm} p{30mm}}
\hline 
\wancitableheader{Feldname} & \wancitableheader{Bedeutung} & \wancitableheader{Notwendig?}\\
\hline
ID & Eindeutige ID eines Datensatzes; wird zur Referenzierung verwendet & Ja\\
\hline
active & Ist ein Datensatz aktiv? Je nach Einstellung werden Datens\"atze mit active = 0 nicht angezeigt & Empfohlen, f\"ur Verwendung des Rechtesystems UXS erforderlich\\
\hline
CreatedBy & ID eines Users, der den Datensatz angelegt hat& Ja, wenn SAM/UXS benutzt wird\\
\hline
CreatedDateTime & Timestamp, bei dem der Datensatz angelegt wurde & Ja\\
\hline
ChangedBy & ID eines Users, der den Datensatz ge\"andert hat & Ja, wenn SAM/UXS benutzt wird\\
\hline
ChangedDateTime & Timestamp, bei dem der Datensatz ge\"andert wurde & Ja\\
\hline
public          & Ist dieser Datensatz f\"ur alle zu sehen? & Ja, wenn UXS benutzt wird\\
\hline
\end{tabular*}

\medskip

Die Bedeutung der einzelnen Felder wird f\"ur UXS \wancisee{UXS} und
SAM \wancisee{SAM} an anderer Stelle erkl\"art.

%%
\chapter{Templates}
\label{Templates}

Dieses Kapitel beschreibt die dmerce Templates genauer. Templates
(engl., zu deutsch Vorlagen) bestehen aus einer HTML-Seite, in die
dmerce Makros und Konstrukte eingebettet werden.

\newpage
\section{Organisation von Templates auf dem Filesystem}

Templates sind in Verzeichnissen organisiert. Um z. B. das Template
\wancicode{templates/Hallo/dmerce/Welt.html} aufzurufen, lautet der
Parameter \wancicode{qTemplate} f\"ur dmerce
\wancicode{Hallo,dmerce,Welt}.
\wancisee{AdminGuideOrganisationTemplates}

\section{Aufbau und Struktur eines Templates}

Ein dmerce Template hat quasi keinen bestimmten Aufbau. Lediglich
Konstruke wie \wancicode{if/else/endif} \wancisee{BibMakroif} und
\wancicode{repeat/endrepeat} \wancisee{BibMakrorepeat} stellen
Bl\"ocke dar.

\section{Einf\"ugen von Templates in Templates zur Laufzeit}

\section{Das Fehlertemplate}

Im Falle eines schwereren Fehlers, der nicht per Meldung innerhalb
eines Templates angezeigt werden kann, wird das Template
\wancicode{templates/error.html} aufgerufen. Dieses Template ist vom
Templater erstellt worden, der z. B. pro Website die Anzeige solcher
Fehler in einem bestimmten Design darstellen m\"ochte. Existiert
dieses Template nicht, so wird eine HTML-Seite von dmerce automatisch
generiert, die dann den Fehler genauer beschreibt.

%%
\chapter{Variablen}

Dieses Kapitel beschreibt die Verwendung von Variablen in dmerce.
Variablen sind Container, die einen Namen und einem ihr zugeordneten
Wert besitzen.

\newpage
\section{Syntax}

Die Syntax f\"ur private Variablen lautet: \wancicodeblock{(var x)}
Die Syntax f\"ur Variablen lautet: \wancicodeblock{(form x)} wobei x
jeweils den Namen einer Variable bezeichnet.

\subsection{Geltungsbereich}

Variablen k\"onnen an jeder beliebigen Stelle innerhalb von Makros
angesprochen werden. Sie bilden die unterste Schicht der dmerce
Makros.

\subsection{Listen}

Wird eine Variable doppelt \"ubergeben, so gibt es zwei
M\"oglichkeiten: entweder dies ist ein Fehler vom Templater, der dazu
f\"uhrt, dass dmerce einen Fehler auswirft oder aber es handelt sich
um Absicht, da ein paar Funktionalit\"aten von dmerce mit Listen von
Werten (mehrere Werte unter ein und derselben Bezeichnung einer
Variable abgelegt sind eine Liste) umgehen k\"onnen. Solche Listen
werden bei Funktionen angewandt, die z. B. mehrere
Datenbankmanipulationen wie Insert \wancisee{MultipleInsert}, Update
\wancisee{MultipleUpdate} und Delete \wancisee{MultipleDelete}
ausf\"uhren.

\section{Werte aus Hyperlinks und Formularen}
\label{Variablen}

Auf Variablen aus Hyperlinks oder Formularen lassen sich unter dmerce
unter Verwendung des Makros \wancicode{(form)} zugreifen. Bitte
beachten Sie, dass Variablen, die mit dem Buchstaben \wancicode{q}
beginnen, immer als private Variable angesehen werden
\wancisee{PrivateVariablen}.

\wancidmerceqvardef{}{ignore\_}{ignore\_$<$feldname$>$=$<$wert$>$}{ab
  1.2.6}{BASIS}{Um Variablen zu \"ubergeben, die nicht bei
  automatischen Manipulationen der Datenbank ber\"ucksichtig werden
  sollen und auf die man im folgenden Template per \wancicode{(form
    $<$feldname$>$)} trotzdem zur\"uckgreifen m\"ochte, muss ein
  \wancicode{ignore\_} vor die Variable gesetzt werden}

\section{Private Variablen oder das heilige Q}
\label{PrivateVariablen}

dmerce hat private Variablen. Diese beginnen ausschliesslich mit dem
Buchstaben \wancicode{q}, um zu verhindern mit bestehenden Systemen in
Konflikt zu geraten. Variablen, die \"ubergeben werden und mit
\wancicode{q} beginnen sind somit per \wancicode{(var)} zug\"anglich.

Desweiteren werden alle Werte, die zur Steuerung von dmerce beitragen
\"uber diesen Weg \"ubergeben.

Es gibt eine Reihe von vordefinierten privaten Variablen, die Auskunft
\"uber das aktuelle System/die aktuelle Website geben.

\wancidmerceqvardef{qvarqTemplate}{qTemplate}{qTemplate={[}Verzeichnis{[},...{]}}{ab
  1.0}{BASIS}{Gibt dmerce das Template an, dass angezeigt werden
  soll}{}

\wancidmerceqvardef{VarqSuperSearch}{qSuperSearch}
{qSuperSearch=\wancicode{$<$Feld$>$$<$Operator$>$$<$Wert$>$
    {[}$<$logischer Operator$>$...{]}}}{ab 1.2.6}{BASIS}{Private
  Variable f\"ur einen Suchstring, der auf eine Datenbankabfrage
  abgebildet wird; wird zus\"atzlich aus qSearch{*}
  (\wancisee{qSearch}) zusammengesetzt. Da ab dmerce 2.0.0 die
  letztendliche Abfrage nur \"uber das Makro \wancicode{(repeat)}
  zusammengesetzt wird, wird \wancicode{qSuperSearch} nur noch
  automatisch von dmerce im Hinblick auf \wancicode{qSearch}
  (\wancisee{qSearch}) behandelt.  Wird aus Gr\"unden der
  Abw\"artskompatibilit\"at und des besseren Verst\"andnisses auch
  noch von Templatern benutzt.}

\subsubsection{Syntax}
\label{qSuperSearchOperatoren}

Der qSuperSearch versteht folgende Syntax. Diese Symbole wurde so
gew\"ahlt, dass sie immer leicht \"uber HTTP zu \"ubergeben sind.
Klammerung kann wie gewohnt verwendet werden.

\medskip

\begin{tabular*}{128mm}{p{30mm} p{90mm}}
\wancitableheader{Symbol} & \wancitableheader{\"Aquivalent in SQL}\\
\hline
{*} & gleich; =\\
\hline
\textasciitilde{} & LIKE '\%$<$Wert$>$\%'\\
\hline
: & und; AND\\
\hline
{|} & oder; OR\\
\hline
{!} & nicht; NOT\\
\hline
{!}{*} & ungleich; {!}=\\
\hline
{[} & kleiner; $<$\\
\hline
{]} & gr\"osser; $>$\\
\hline
{@} & begrenze; LIMIT\\
\hline
\{ & ordne nach Feld; ORDER BY\\
\hline
\} & gruppiere nach Feld; GROUP BY\\
\hline
\end{tabular*}

\wancidmerceqvardef{}{qSearch{*}}{keine}{ab 1.2.6}{BASIS}{qSearch{*}
  steht f\"ur eine Gruppierung von privaten Variablen, die alle mit
  dem Prefix qSearch beginnen. Der Stern kennzeichnet diese als
  Wildcard. Gemeint sind ferner \wancicode{qSearch\_},
  \wancicode{qSearchName\_} und \wancicode{qSearchOp\_}.
  
  Im Hinblick auf Datenbankabfragen ist \wancicode{qSearch}* die
  \wancicode{qSuperSearch}- Version aus HTML-Formularen heraus. Man
  hat in Formularen nicht wie in Hyperlinks die M\"oglichkeit, viele
  Werte zu einem Link zu formen und diesen dann an dmerce zu
  \"ubergeben, damit dmerce exakt weiss, was es zu tun hat. Alle
  \wancicode{qSearch}*-Variablen werden ausgewertet und dann (wie oben
  beschrieben; \wancisee{VarqSuperSearch} in der privaten Variable
  \wancicode{qSuperSearch} zusammengesetzt.}

\subsubsection{qSearch\_, qSearchName\_ und qSearchFields\_}
\label{qSearch}

Syntax: \wancicode{qSearchFields\_$<$Bezeichnung$>$=$<$Feld{[},...{]}$>$}

\medskip

Beschreibe mit \wancicode{qSearchFields\_$<$Bezeichnung$>$} eine Liste
von Datenbankfeldern, von denen \textit{eines} den in
\wancicode{qSearchName\_$<$Bezeichnung$>$} genannten Wert enth\"alt.
Bei der Suche wird der LIKE-Operator zur Suche verwendet, wenn nicht
mit \wancicode{qSearchOp\_$<$Bezeichnung$>$} etwas anderes angegeben
ist.  \wancicode{$<$Bezeichnung$>$} ist eine frei w\"ahlbare
Bezeichnung, mit der diese Gruppierung der bei
\wancicode{qSearchFields\_$<$Bezeichnung$>$} genannten Felder
eindeutig benannt wird. Werden in einem Aufruf mehrere Aufrufe von
diesem Typ verwendet, werden diese UND-verkettet!

\subsection{qSearchOp\_}

Enth\"alt den Operator \wancisee{qSuperSearchOperatoren}, der auf
diesen Teil einer Abfrage angewendet werden soll.

\wancidmerceqvardef{}{qs}{(var qs)}{ab 2.1.0}{SAM}{Enth\"alt die
  aktuelle Session ID. Damit das Feature der automatischen Erkennung
  von SAM, das auf IP-Adressen basiert, nicht von dmerce genutzt wird,
  kann der Templater die Session ID manuell \"ubergeben. Empfiehlt
  sich f\"ur Umgebungen, in denen Clients oft ihre IP-Adresse
  wechseln. Dazu geh\"oren DialUp- Clients, deren Timeout zum Beenden
  einer Verbindung kleiner ist, als das von SAM verwendete Timeout.
  Sp\"ater z\"ahlt auch IPv6 dazu, indem es die M\"oglichkeit geben
  wird, dass ein Client seine IP von Request zu Request \"andert, um
  seine Herkunft zu verbergen.}

\wancidmerceqvardef{}{qButton}{keine}{ab 1.0; seit dmerce 2.2.3
  \"uberfl\"ussig.}{BASIS}{Man verwendet \wancicode{qButton} um
  Datenbankmanipulationen wie INSERT, UPDATE oder DELETE ausf\"uhren
  zu lassen. Wird dieser Parameter nicht mit irgendeinem Wert
  \"ubergeben, so findet keine Manipulation von Daten statt.}

\wancidmerceqvardef{}{qDelete}{qDelete={[}$<$Tabelle$>${[},...{]}}{ab
  1.0}{BASIS}{}

\wancidmerceqvardef{}{qInsert}{qInsert={[}$<$Tabelle$>${[},...{]}}{ab
  1.0}{BASIS}{}

\wancidmerceqvardef{}{qUpdate}{qUpdate={[}$<$Tabelle$>${[},...{]}}{ab
  1.0}{BASIS}{}

\wancidmerceqvardef{}{qTrigger}{qTrigger={[}$<$Paket.Modul.Funktion$>${[},...{]}}{ab
  1.2.6}{BASIS}{}

\wancidmerceqvardef{}{qTriggerAfter}{qTrigger={[}$<$Paket.Modul.Funktion$>${[},...{]}}{ab
  1.2.6}{BASIS}{}

\wancidmerceqvardef{}{qLogin*}{}{ab 1.2.0}{BASIS}{}

\paragraph{qLogin}
\paragraph{qLoginTable}
\paragraph{qLoginField}
\paragraph{qPasswdField}
\paragraph{qPasswd}

\wancidmerceqvardef{qvarVERSION}{VERSION}{(var VERSION)}{ab 1.2.4}{BASIS}{Gibt
  die aktuelle Version von dmerce (z. B. \dmercever) aus.}

\wancidmerceqvardef{qvarSSL}{SSL}{(var SSL)}{ab 1.2.6}{BASIS}{Gibt den
  Status der HTTP-Verbindung im Bezug auf SSL zur\"uck. 0 = SSL wird
  nicht verwendet (Port 80), 1 = SSL wird verwendet (Port 443), -1 =
  Fehler, der Status der Verbindung konnte nicht erkannt werden}

\section{Session-Variablen mit myvar}
\label{myvar}

Die Intention der \wancislang{myvar} ist es, dem Templater die
M\"oglichkeit zu geben, pro Session und User einer Website eigens
definierte Variablen sozusagen automatisch mitzuschleifen. Eine
\wancislang{myvar} ist, einmal gesetzt, in jedem Template verf\"ugbar.
Hierbei ist es nicht n\"otig, die Variable \"uber \wancicode{(form)}
jedes Mal zu \"ubergeben. Ein Beispiel ist ein Online-Shop. Sie
m\"ochten auf den Seiten der Registrierung oder den Seiten der
Bestellung bestimmte Informationen vom User abfragen ohne diese gleich
in der Datenbank zu speichern (reduziert Belastung der Datenbank!), da
die Daten nur f\"ur den weiteren Ablauf der Seiten und nicht f\"ur Sie
als Shop-Betreiber wichtig sind. Sie speichern also nach einer
Abfrage, welche Lieferart der Kunde bevorzugt diese in einer
\wancislang{myvar} und rufen diese Informationen Seiten sp\"ater
wieder hervor.

\bigskip

Beispiel:

\wancicodeblock{(set-myvar $<$name$>$ $<$wert$>$)}

setzt die \wancislang{myvar} $<$name$>$ auf den Wert $<$wert$>$. Als
Wert kann hier jedes andere dmerce Makro genutzt werden: z. B.

\wancicodeblock{(set-myvar test (sql tabelle.feld))}

setzt die \wancislang{myvar} $<$test$>$ auf den Wert von Tabelle.Feld
aus der gerade laufenden Abfrage in einem \wancislang{repeat}
\wancisee{repeat}. Siehe hierzu auch Kurz\"uberblick in der Bibliothek
\vref{BibMakrosetmyvar} und \vref{BibMakrogetmyvar}

%%
\chapter{Kontrollstrukturen}

Dieses Kapitel beschreibt die in dmerce zur Ver\"ugung stehenden
Konstrollstrukturen.  Mit ihnen lassen sich Ausgaben per Template je
nach verschiedensten Zust\"anden, die Sie als Templater sich
ausdenken, reglementieren.

\newpage
\section{Fallabfragen}
\label{Fallabfragen}

Syntax: \wancicodeblock{(if $<$Ausdruck$>$)
\\
...\\
weiterer HTML- und dmerce-Code\\
...\\
\\
(endif)}

Mit \wancicode{(if)} / \wancicode{(endif)} erzeugen Sie Abschnitte in
einem Template, die nur dann ausgef\"uhrt werden, wenn der Ausdruck
WAHR ist. \wancicode{(if)} / \wancicode{(endif)} darf geschachtelt
werden.

Eine kurze Definition und \"Ubersicht finden Sie in der Bibliothek
\wancisee{BibMakroif}

\subsection{Logische Bedingungen}

Syntax: \wancicode{$<$Wert1$>$$<$Operator$>$$<$Wert2$>$}

\medskip

Eine Logische Bedingung ist der Vergleich von zwei oder mehreren
Werten $<$Wert1$>$, $<$Wert2$>$ anhand eines Logischen Operators
$<$Operator$>$ auf ein Ergebnis WAHR. Es k\"onnen nur gleichartige
Werte verglichen werden. d.h. entweder vergleichen Sie Zahlenwerte
oder Strings (Zeichenketten). Der Vergleich von Strings erfordert, das
die zu vergleichenden Werte in Hochkomma (') eingefasst werden.

\bigskip

\begin{tabular*}{128mm}{p{30mm} p{90mm}}
\wancitableheader{Operator} & \wancitableheader{Bedeutung}\\
\hline
$<$= & Kleiner gleich\\
\hline 
$>$= & Gr\"osser gleich\\
\hline 
== & gleich\\
\hline 
{!}= & ungleich\\
\hline 
\end{tabular*}

\bigskip

Beispiel 1: Vergleich von zwei Zahlenwerten

\wancicodeblock{(if (who ID) != 20)}

Wenn der Wert aus \wancicode{(who ID)} ungleich 20 ist, wird der
Inhalt des Templates zwischen diesem if/endif-Block ausgewertet und
ausgegeben.

\medskip

Beispiel 2: Vergleich von zwei Zeichenketten

\wancicodeblock{(if '(who Name)' == 'Meier')}

Wenn der String aus \wancicode{(who Name)} gleich Meier ist, wird der
Inhalt zwischen dem if/endif-Block ausgef\"uhrt.

\subsection{Logische Operatoren}

Syntax: \wancicode{$<$Ausdruck1$>$$<$logischer Operator$>$$<$Ausdruck2$>$ ...}

\medskip

Logische Bedinungungen ($<$Statement$>$) k\"onnen durch Operatoren
$<$Operator$>$ verkn\"upft werden. Das macht Sinn, wenn auf
verschiedene Bedingungen gepr\"uft werden soll.

\bigskip

\begin{tabular*}{128mm}{p{30mm} p{90mm}}
\wancitableheader{Operator} & \wancitableheader{Bedeutung}\\
\hline
and & UND-Verkn\"upfung. Beide Bedingungen m\"ussen WAHR ergeben, damit das if-Statement ausgef\"uhrt wird.\\
\hline
or & ODER-Verkn\"upfung. Eine der beiden Bedingungen muss WAHR ergeben, damit das if-Statement
ausgef\"uhrt wird.\\
\hline
\end{tabular*}

\bigskip

Beispiel 1: Vergleich von zwei mal zwei Zahlenwerten UND-verkn\"upft

\wancicodeblock{(if (who ID) != 20 and (who ID != 40)}

Wenn der Wert aus (who ID) ungleich 20 UND ungleich 40 ist, wird der
Inhalt des Templates zwischen diesem if/endif-Block ausgewertet und
ausgegeben.

\medskip

Beispiel 2: Vergleich von zwei mal zwei Zeichenketten, oder verkn\"upft

\wancicodeblock{(if '(who Name)' == 'Meier' or '(who Name)' ==
  'Schulte')}

Wenn der String aus (who Name) gleich Meier ODER Schulte ist, wird der
Inhalt zwischen dem if/endif-Block ausgef\"uhrt.

\medskip

Beispiel 3: Vergleich von unterschiedlichen Datentypen

\wancicodeblock{(if '(who Name)' == 'Meier' and (who ID) != 20)}

Wenn der String aus (who Name) gleich Meier UND der Zahlenwert aus
(who ID) ungleich 20 ist, wird der Inhalt zwischen dem if/endif-Block
ausgegeben.

%%
\chapter{Datenbankabfragen}

Ein Kernst\"uck und vor allem die Kernkompetenz von dmerce sind
Datenbankabfragen. Dieses Kapitel beschreibt Ihnen die
M\"oglichkeiten, die dmerce Ihnen in einem Projekt bietet.

\newpage
\section{Starten einer Datenbankabfrage mit \wancislang{repeat}}
\label{repeat}

Syntax: \wancicodeblock{(repeat ``$<$Bedingung$>$'')
\\
...\\
weiterer HTML- und dmerce-Code\\
...\\
\\
(endrepeat)}

\medskip

Der zwischen (repeat) und (endrepeat) stehende HTML- und dmerce-Code
wird solange ausgef\"uhrt, wie das Query zutrifft und Datens\"atze
zur\"uckgibt. Erst wenn alle Datens\"atze auszugeben sind, wird der
Code hinter (endrepeat) ausgef\"uhrt.

\subsection{Zugriff auf Ergebnisse einer Datenbankabfrage}

Innerhalb eines Repeat werden die Felder der aktiven Datenbank-Tabelle
in der Form (sql Tabelle.Feld) angesprochen.

\medskip

Beispiel:

\wancicodeblock{(repeat ``Artikel.ID{]}0'')\\
\\
(sql Artikel.ArtNo) (sql Artikel.Bezeichnung)<br>\\
\\
(endrepeat)}

\medskip

Ausgabe:

\bigskip

\begin{tabular*}{128mm}{p{30mm} p{90mm}}
12355 & 500 MHz Pentium II SECII\\
32123 & 700 MHz Pentium III\\
42121 & 800 MHz Athlon\\
95331 & 400 MHz PowerPC\\
\end{tabular*}

\bigskip

Alle Eintr\"age der Tabelle Artikel, auf die der verwendete Query
zutrifft, werden mit der Artikelnummer (Feldname ArtNo) und der
Bezeichnung (Feldname Bezeichnung) ausgegeben.

\subsection{(sql Tabelle.Feld ``Bedingung'')}
\label{sqltf}

Innerhalb eines Repeat werden die Felder einer Datenbank-Tabelle, die
nicht per qTemplate oder qTN angeprochen wurde, ausgegeben. Um
anzugeben, welcher Eintrag dieser Tabelle angesprochen werden soll,
ist eine eindeutige Bedingung anzugeben. Diese Cross-Referenzen sind
sehr n\"utzlich, um z.B. in einer Tabelle die Bezeichnung von
Produktgruppen zu speichern und in der Tabelle, in welcher die Artikel
gespeichert sind, nur noch die ID der Produktgruppe vorhalten zu
m\"ussen. Dadurch ist die Tabelle Artikel wesentlich schneller nach
Produktgruppen zu indizieren (schnellere Suche bei geringerer Last auf
dem System) und die Bezeichnungen der Produktgruppen k\"onnen an einer
zentralen Stelle gewartet werden.

\medskip

Beispiel:

\wancicodeblock{(repeat ``Artikel.ID{]}0'')\\
  \\
  (sql Artikel.ArtNo) (sql Produktgruppen.Name ``ID*(sql Artikel.prodgrp)'')
  (sql Artikel.Bezeichnung)<br>\\
  \\
  (endrepeat)}

\medskip

Ausgabe:

\bigskip

\begin{tabular*}{128mm}{p{30mm} p{30mm} p{60mm}}
12335 & Prozessor & 500 MHz Pentium II SECII\\
32123 & Prozessor & 700 MHz Pentium III\\
Z6352 & Festplatte & IBM SCSI 18,3 GB\\
42121 & Prozessor & 800 MHz Athlon\\
95331 & Prozessor & 400 MHz PowerPC\\
A7564 & L\"ufter & L\"ufter CPU-Cooler\\
\end{tabular*}

\bigskip

Alle Eintr\"age der Tabelle Artikel, auf die das verwendete Query
zutrifft, werden mit der Artikelnummer (Feldname ArtNo), der
Produktgruppe und der Bezeichnung (Feldname Bezeichnung) ausgegeben.

\medskip

Komplexes Beispiel:

\wancicodeblock{(repeat ``Artikel.ID{]}0'')\\
  \\
  (sql Artikel.Bezeichnung) (sql Lieferant.Land ``ID*(sql Artikel.LieferantID)'')<br>\\
  \\
  (endrepeat)}

\medskip

Ausgabe:

\medskip

IBM SCSI 18,3 GB Slowakei

\bigskip

In der Tabelle Artikel enth\"alt das Feld Lieferant die ID des
Lieferanten. Anhand dieser ID kann eindeutig auf die Tabelle Lieferant
zugegriffen werden. In dieser findet sich im Feld Land die
internationale Vorwahl des Lieferanten. Anhand dieser Vorwahl kann die
Tabelle Land mit dem Feld Name eindeutig angesprochen werden.

\subsection{Definition: Der \wancislang{inner-repeat}}

Der zwischen (repeat $<$qSuperSearch$>$) HTML- und dmerce-Code, wird
solange ausgef\"uhrt, wie der Query mit $<$qSuperSearch$>$ auf die
Datens\"atze zur\"uckgibt. Erst wenn alle Datens\"atze auszugeben
sind, wird der Code hinter (endrepeat) ausgef\"uhrt.

\medskip

Beispiel:

\wancicodeblock{(repeat ``Artikel.ArtNo\textasciitilde{}12'')\\
\\
(sql Artikel,ArtNo) (sql Artikel.Bezeichnung)<br>\\
\\
(endrepeat)}

\medskip

Ausgabe:

\bigskip

~~~~~~~~~~~~~~~~~ 12355~~ 500 MHZ Pentium II SECII

~~~~~~~~~~~~~~~~~ 32123~~ 700 MHZ Pentium III

~~~~~~~~~~~~~~~~~ 42121~~ 800 MHZ Athlon

\bigskip

Alle Eintr\"age in der Tabelle Artikel, auf die der verwendete Query
ArtNo\textasciitilde{}12 zutrifft, werden ausgegeben.

\subsection{Datenbankabfragen optimieren: SQL Join}

\section{Systematiken f\"ur Abfragen}

\subsection{Abfragen aus Formularen}
\subsubsection{Aus verschiedenen Werten eines Formulares}
\paragraph{qSearch{*}}

\subsubsection{Ignorieren von Variablen bei einer Datenbankoperation: ignore\_}

\section{Operatoren f\"ur Datenbankabfragen}

\subsection{Arithmetische Operatoren}
\subsubsection{gleich}
\subsubsection{ungleich}
\subsubsection{gr\"osser als}
\subsubsection{kleiner als}

\subsection{Logische Operatoren: Verkn\"upfungen}
\subsubsection{Wildcards}
\subsubsection{AND}
\subsubsection{OR}
\subsubsection{NOT}
\subsubsection{LIKE}

\subsection{Klammerung}

\subsection{Sortieren, Gruppieren und Einschr\"anken von Ergebnissen}
\subsubsection{Ordnen: ORDER BY}
\subsubsection{Gruppieren: GROUP BY}
\subsubsection{Limitieren: LIMIT}

%%
\chapter{Datenbankmanipulation}

\section{qInsert}
\label{qInsert}
\label{MultipleInsert}

\section{qUpdate}
\label{qUpdate}
\label{MultipleUpdate}

\section{qDelete}
\label{qDelete}
\label{MultipleDelete}

%%
\chapter{Bibliothek}

Dieser Abschnitt gibt als Nachschlagewerk einen kompletten \"Uberblick
\"uber alle Komponenten von dmerce.

\newpage
\section{Die Gesetze}

dmerce ist ein System das sich strenge Gesetze zu nutze macht, um
dadurch qualitative Anwendungen zu erstellen.

\section{Die Namen}

\section{Die Makros}

Makros werden in folgender Reihenfolge nacheinander ausgewertet:

\medskip

\begin{enumerate}
\item (form)
\item (var)
\item (who)
\item (sql dboid)
\item (sql)
\item (sqlf)
\item (sql) als Cross-Referenz
\item (sqlf) als Cross-Referenz
\item (sql count)
\item (sql rowcount)
\item (sql index)
\item (sql index1)
\item (sql SELECT)
\item (get-myvar)
\item (exec)
\item (set-skin)
\item (set-skin-stylesheet)
\item (skin-img)
\item (skin-imgtag)
\item (set-myvar)
\end{enumerate}

\medskip

Sie k\"onnen dabei in absteigender Reihenfolge verschachtelt werden.

\bigskip

\wancidmercemacrodef{BibMakroappendmyvar}{(append-myvar)}{(append-myvar
  $<$name$>$ $<$wert$>$)}{ab 2.2.0}{BASIS}{H\"angt einen Wert
  $<$wert$>$ an den Wert der bestehenden myvar $<$name$>$ an}{}

\wancidmercemacrodef{BibMakroexec}{(exec)}{(exec Modul.Paket.Funktion
  {[}$<$Argumente$>${]})}{ab 1.2.6}{BASIS}{Ruft eine Funktion
  auf.}{\vref{Funktionen}}

\wancidmercemacrodef{BibMakroform}{(form)}{(form $<$name$>$)}{ab
  1.0.0}{BASIS}{Gibt einen Wert aus einer Variable $<$name$>$
  zur\"uck, die per Hyperlink oder Formular \"ubergeben wurde.}{}

\wancidmercemacrodef{BibMakrogetmyvar}{(get-myvar $<$name$>$)}{}{ab
  2.2.0}{BASIS}{Liefert den Inhalt der myvar $<$name$>$ zur\"uck, die
  vorher mit \wancicode{(set-myvar)} \wancisee{BibMakrosetmyvar}
  gesetzt wurde.}{\vref{myvar}}

\wancidmercemacrodef{BibMakrogettext}{(get-text)}{(get-text $<$name$>$
  {[}$<$sprache$>${]})}{ab 2.2.5}{BASIS}{}{}

\wancidmercemacrodef{BibMakroif}{(if) / (else) / (endif)}{}{ab
  1.0.0}{BASIS}{Pr\"uft einen Ausdruck und f\"uhrt ggf. Code aus dem
  Block zwischen if/else, else/endif oder if/endif
  aus.}{\vref{Fallabfragen}}

\wancidmercemacrodef{BibMakroinclude}{(include)}{}{ab
  2.1.0}{BASIS}{}{}

\wancidmercemacrodef{BibMakroq}{(q)}{}{ab 2.1.0}{BASIS}{}{}

\wancidmercemacrodef{BibMakrosetmyvar}{(set-myvar)}{(set-myvar
  $<$name$>$ $<$wert$>$)}{ab 2.2.0}{BASIS}{Setzt die myvar $<$name$>$
  auf den Wert $<$wert$>$}{\vref{myvar}}

\wancidmercemacrodef{BibMakrosetskin}{(set-skin)}{(set-skin
  $<$verzeichnis$>$)}{ab 2.2.3}{BASIS}{Gibt das in den Makros
  \wancicode{(skin-img)}, \wancicode{(skin-stylesheet)} zu verwendende
  Verzeichnis an.}{}

\wancidmercemacrodef{BibMakroskinimg}{(skin-img)}{(skin-img
  $<$name$>$)}{ab 2.2.3}{BASIS}{Sucht aus dem durch
  \wancicode{(set-skin)} festgelegten Verzeichnis das Bild
  $<$name$>$.\{.jpeg,.jpg,.gif\} heraus und gibt denn vollen
  Dateinamen zur\"uck.}{}

\wancidmercemacrodef{BibMakroskinimgtag}{(skin-imgtag)}{(skin-imgtag
  $<$name$>$)}{ab 2.2.3}{BASIS}{verh\"alt sich wie
  \wancicode{(skin-img)}, liefert allerdings nicht nur den vollen
  Dateinamen, sondern erzeugt ein komplettes $<$img$>$-Tag.}{}

\wancidmercemacrodef{BibMakroskinstylesheet}{(skin-stylesheet)}{(skin-stylesheet)}{ab
  2.2.3}{BASIS}{Liefert ein $<$link$>$-Tag zur\"uck, dass auf die
  Stylesheet aus dem zuvor mit \wancicode{(set-skin)} gew\"ahlten
  Verzeichnis verweist.}{}

\wancidmercemacrodef{BibMakrorepeat}{(repeat) / (endrepeat)}{}{ab
  1.2.6}{BASIS}{Der zwischen \wancicode{(repeat)} und
  \wancicode{(endrepeat)} stehende HTML- und dmerce-Code, wird solange
  ausgef\"uhrt, wie der Query zutrifft und Datens\"atze zur\"uckgibt.
  Erst wenn alle Datens\"atze auszugeben sind, wird der Code hinter
  \wancicode{(endrepeat)} ausgef\"uhrt.}{}

\wancidmercesubmacrodef{BibMakrosql}{(sql)}{(sql
  $<$tabelle$>$.$<$feld$>$
  {[}$<$position-von$>${]}{[},$<$positon-bis$>${]})}{ab
  1.2.6}{BASIS}{Innerhalb eines \wancislang{repeat} werden die Felder
  eines Ergebnisses einer Datenbankabfrage in der Form Tabelle.Feld
  angesprochen. $<$position-von$>$ und $<$position-bis$>$ dienen dazu,
  die Ausgabe zu begrenzen. Wird $<$position-von$>$ weggelassen, so
  wird daf\"ur 0 angenommen.}{}

\wancidmercesubmacrodef{BibMakrosqlxref}{(sql) als
  Cross-Referenz}{(sql $<$tabelle$>$.$<$feld$>$
  ``$<$qSuperSearch$>$'')}{ab 1.2.6}{BASIS}{Innerhalb eines
  \wancislang{repeat} werden Felder aus dem Ergebniss einer
  Datenbankabfrage ausgegeben. Cross-Referenzen sind sehr n\"utzlich,
  um z. B. in einer Tabelle die Bezeichnung von Produktgruppen zu
  speichern und in der Tabelle, in welcher die Artikel gespeichert
  sind, nur noch die ID der Produktgruppe vorhalten zu m\"ussen.
  Dadurch ist die Tabelle Artikel wesentlich schneller nach
  Produktgruppen zu indizieren (schnellere Suche bei geringerer Last
  auf dem System) und die Bezeichnungen der Produktgruppen k\"onnen an
  einer zentralen Stelle gewartet werden.}{\vref{sqltf}}

\wancidmercesubmacrodef{BibMakrosqlrowcount}{(sql rowcount)}{(sql
  rowcount)}{ab 1.2.6}{BASIS}{Liefert die Anzahl der Ergebnisszeilen
  der aktuellen Abfrage/des aktuellen (repeat)}{\vref{repeat}}

\wancidmercesubmacrodef{BibMakrosqlcount}{(sql count)}{(sql count
  $<$bedingung$>$)}{ab 1.2.6}{BASIS}{Liefert die Anzahl der auf die
  SQL-Abfrage passenden Ergebnisszeilen zur\"uck.}{\vref{repeat}}

\wancidmercesubmacrodef{BibMakrosqldboid}{(sql dboid)}{(sql dboid
  $<$tabelle$>$)}{ab 2.1.0}{BASIS}{Liefert die n\"achsth\"ohere ID
  f\"ur die Tabelle $<$tabelle$>$ zur\"uck. Existiert noch keine DBOID
  f\"ur die Tabelle so wird mit 1 begonnen. Wurde innerhalb eines
  Templates bereits eine DBOID f\"ur eine Tabelle angefordert, so wird
  die DBOID nicht nochmal erh\"ot, sondern die gleiche
  zur\"uckgeben.}{}

\wancidmercesubmacrodef{BibMakrosqlindex}{(sql index)}{(sql index)}{ab
  2.1.0}{BASIS}{Liefert die aktuelle Zeile aus dem Ergebniss eines
  (repeat) zur\"uck, der gerade abgearbeitet wird. Der Wert ist
  0-indiziert, d. h. dass die erste Zeile hier als Zeile 0
  zur\"uckgegeben wird.}{\vref{repeat}}

\wancidmercesubmacrodef{BibMakrosqlindex1}{(sql index1)}{(sql
  index1)}{ab 2.1.0}{BASIS}{wie \wancicode{(sql index)} allerdings
  1-indiziert.}{\vref{repeat}}

\wancidmercemacrodef{BibMakrovar}{(var)}{(var $<$name$>$)}{ab
  1.0.0}{BASIS}{Gibt eine private Variable $<$name$>$ zur\"uck. Diese
  kann eine voreingestelle private Variable sein oder per Hyperlink
  oder Formular \"ubergeben worden sein.}{\vref{qvarVERSION},
  \vref{qvarSSL}}

\wancidmercemacrodef{BibMakrowho}{(who)}{(who $<$objekt$>$)}{ab
  1.2.0}{BASIS}{}{\vref{SAM}}

\newpage

\section{Die Funktionen}
\label{Funktionen}
\subsection{HTML-Konstrukte aus Datenbankabfragen: Db2HTML}

\newpage
\wancidmerceexecdef{}{Db2HTML.Select.Options}{Funktion}{}{}{}{}

\wancidmerceexecparamdefbegin
\wancidmerceexecparamdef{}{}{}
\wancidmerceexecparamdefend

\wancidmerceexecex{}{}{}{}

\newpage
\wancidmerceexecdef{}{Db2HTML.Checkbox.isChecked}{Funktion}{}{}{}{}

\wancidmerceexecparamdefbegin
\wancidmerceexecparamdef{}{}{}
\wancidmerceexecparamdefend

\wancidmerceexecex{}{}{}{}

\newpage
\wancidmerceexecdef{}{Db2HTML.Radiobox.isChecked}{Funktion}{}{}{}{}

\wancidmerceexecparamdefbegin
\wancidmerceexecparamdef{}{}{}
\wancidmerceexecparamdefend

\wancidmerceexecex{}{}{}{}

\newpage
\subsection{Sicherheit: Security}

\newpage
\wancidmerceexecdef{}{Security.Password.Generate}{Funktion}{}{}{}{}

\wancidmerceexecparamdefbegin
\wancidmerceexecparamdef{}{}{}
\wancidmerceexecparamdefend

\wancidmerceexecex{}{}{}{}

\newpage
\subsection{HTML}

\newpage
\wancidmerceexecdef{}{HTML.PlainToHTML.FormatFields}{Trigger}{}{}{}{}

\wancidmerceexecparamdefbegin
\wancidmerceexecparamdef{}{}{}
\wancidmerceexecparamdefend

\wancidmerceexecex{}{}{}{}

\newpage
\wancidmerceexecdef{}{HTML.HTMLToPlain.Format}{Funktion}{}{}{}{}

\wancidmerceexecparamdefbegin
\wancidmerceexecparamdef{}{}{}
\wancidmerceexecparamdefend

\wancidmerceexecex{}{}{}{}

\newpage
\subsection{Benutzer im SAM anmelden: Login}

\newpage
\wancidmerceexecdef{BibExecLoginUserCheck}{Login.User.Check}{Trigger}{}{}{}{}

\wancidmerceexecparamdefbegin
\wancidmerceexecparamdef{}{}{}
\wancidmerceexecparamdefend

\wancidmerceexecex{}{}{}{}

\newpage
\subsection{Datums- und Zeitfunktionen: Time}

Es gibt Conversion, Get, Select

\newpage
\wancidmerceexecdef{}{Time.Conversion.Actual}{Funktion}{}{Das aktuelle
  Datum formatiert oder als Timestamp
  zur\"uckgeben.}{\wancicode{\parbox{85mm}{Time.Conversion.Actual
      [formatDate='<Datumsformatierung>'
      [,formatTime='<Zeitformatierung>']]}}}{}

\wancidmerceexecparamdefbegin
\wancidmerceexecparamdef{formatDate}{Formatierung der Ausgabe des
  Datums. Standard: \%Y-\%m-\%d \wancisee{FormatierungDatumZeit}}{Nein}
\wancidmerceexecparamdef{formatTime}{Formatierung der Ausgabe der
  Zeit. Standard: \%H-\%M-\%S \wancisee{FormatierungZeit}}{Nein}
\wancidmerceexecparamdefend

\wancidmerceexecex{keine}{2001-11-11
  22:42:30}{Das heutige Datum: (exec Time.Conversion.Actual cexe)}{Das
  heutige Datum: 2001-11-11 22:42:30}

\wancidmerceexecex{formatDate='europe',formatTime=''}{11.11.2001}{Das
  heutige Datum: (exec Time.Conversion.Actual
  formatDate='europe',formatTime='' cexe)}{Das heutige Datum:
  11.11.2001}

\newpage
\wancidmerceexecdef{}{Time.Conversion.ActualDate}{Funktion}{}{Das aktuelle
  Datum formatiert
  zur\"uckgeben}{\wancicode{Time.Conversion.ActualDate
    [format='<format>']}}

\wancidmerceexecparamdefbegin
\wancidmerceexecparamdef{format}{Parameter f\"ur Formatierung
  Standard: \%Y-\%m-\%d \wancisee{FormatierungDatumZeit}}{Nein}
\wancidmerceexecparamdefend

\wancidmerceexecex{keine}{2001-11-11}{Das heutige Datum: (exec
  Time.Conversion.ActualDate cexe)}{Das heutige Datum: 2001-11-11}

\wancidmerceexecex{formatDate='europe'}{11.11.2001}{Das heutige Datum:
  (exec Time.Conversion.Actual format='europe' cexe)}{Das heutige
  Datum: 11.11.2001}

\newpage
\wancidmerceexecdef{}{Time.Conversion.ActualTime}{Funktion}{}{Das
  aktuelle Zeit formatiert
  zur\"uckgeben.}{\wancicode{Time.Conversion.ActualTime
    [format='<format>']}}
  
\wancidmerceexecparamdefbegin
\wancidmerceexecparamdef{format}{Formatierung der Ausgabe des Datums
  Standard: \%H-\%M-\%S (\wancisee{FormatierungDatumZeit})}{Nein}
\wancidmerceexecparamdefend
  
\wancidmerceexecex{keine}{22:42:30}{Die aktuelle Zeit: (exec
  Time.Conversion.ActualTime cexe)}{Die aktuelle Zeit: 22:42:30}
  
\wancidmerceexecex{format='\%H:\%M'}{22:42}{Die aktuelle Zeit: (exec
  Time.Conversion.ActualTime format='\%H:\%M' cexe)}{Die aktuelle
  Zeit: 22:42}
  
\newpage
\wancidmerceexecdef{}{Time.Conversion.Calc}{Funktion}{}{}{}{}

\wancidmerceexecparamdefbegin
\wancidmerceexecparamdef{}{}{}
\wancidmerceexecparamdefend

\wancidmerceexecex{}{}{}{}

\newpage
\wancidmerceexecdef{}{Time.Conversion.DiffInDays}{Funktion}{}{}{}{}

\wancidmerceexecparamdefbegin
\wancidmerceexecparamdef{}{}{}
\wancidmerceexecparamdefend

\wancidmerceexecex{}{}{}{}

\newpage
\wancidmerceexecdef{}{Time.Conversion.MergeFieldsToDateTime}{Trigger}{}{}{}{}

\wancidmerceexecparamdefbegin
\wancidmerceexecparamdef{}{}{}
\wancidmerceexecparamdefend

\wancidmerceexecex{}{}{}{}

\newpage
\wancidmerceexecdef{}{Time.Conversion.UnixToLocal}{Funktion}{}{Eine
  Zeitangabe als Unix-TimeStamp wird formatiert
  ausgegeben}{\wancicode{Time.Conversion.UnixToLocal
    secs=<Unix-Timestamp>[,format='<format>']}}{}

\wancidmerceexecparamdefbegin
\wancidmerceexecparamdef{secs}{Wert in
  Sekunden. Standard = jetzt}{Nein}
\wancidmerceexecparamdef{format}{Formatierung der Ausgabe.
  Standard-Format: \%Y-\%m-\%d \%H:\%M:\%S
  \wancisee{Formatierung}.}{Nein}
\wancidmerceexecparamdefend

\wancidmerceexecex{secs=1000000000.00}{2001-09-09 01:46:40}{Sie waren
  zuletzt eingeloggt am (exec Time.Conversion.UnixToLocal
  secs=1000000000.00 cexe)}{Sie waren zuletzt eingeloggt am 2001-09-09
  01:46:40}

\wancidmerceexecex{secs=1000000000.00,format='\%d.\%m.\%Y
  \%H:\%M:\%S'}{09.09.2001 01:46:40}{Sie waren zuletzt eingeloggt am
  (exec Time.Conversion.UnixToLocal
  secs=1000000000.00,format='\%d.\%m.\%Y \%H:\%M:\%S cexe)}{Sie waren
  zuletzt eingeloggt am 09.09.2001 01:46:40 01:46:40}

\newpage
\wancidmerceexecdef{}{Time.Conversion.isGermanDate}{Funktion}{}{}{}{}

\wancidmerceexecparamdefbegin
\wancidmerceexecparamdef{}{}{}
\wancidmerceexecparamdefend

\wancidmerceexecex{}{}{}{}

\newpage
\wancidmerceexecdef{}{Time.Conversion.isISODate}{Funktion}{}{}{}{}

\wancidmerceexecparamdefbegin
\wancidmerceexecparamdef{}{}{}
\wancidmerceexecparamdefend

\wancidmerceexecex{}{}{}{}

\newpage
\subsection{Time.Get}

\newpage
\wancidmerceexecdef{}{Time.Get.ActualDay}{Funktion}{}{}{}{}

\wancidmerceexecparamdefbegin
\wancidmerceexecparamdef{}{}{}
\wancidmerceexecparamdefend

\wancidmerceexecex{}{}{}{}

\newpage
\wancidmerceexecdef{}{Time.Get.ActualMonth}{Funktion}{}{}{}{}

\wancidmerceexecparamdefbegin
\wancidmerceexecparamdef{}{}{}
\wancidmerceexecparamdefend

\wancidmerceexecex{}{}{}{}

\newpage
\wancidmerceexecdef{}{Time.Get.ActualYear}{Funktion}{}{}{}{}

\wancidmerceexecparamdefbegin
\wancidmerceexecparamdef{}{}{}
\wancidmerceexecparamdefend

\wancidmerceexecex{}{}{}{}

\newpage
\wancidmerceexecdef{}{Time.Get.ActualHour}{Funktion}{}{}{}{}

\wancidmerceexecparamdefbegin
\wancidmerceexecparamdef{}{}{}
\wancidmerceexecparamdefend

\wancidmerceexecex{}{}{}{}

\newpage
\wancidmerceexecdef{}{Time.Get.ActualMinute}{Funktion}{}{}{}{}

\wancidmerceexecparamdefbegin
\wancidmerceexecparamdef{}{}{}
\wancidmerceexecparamdefend

\wancidmerceexecex{}{}{}{}

\newpage
\wancidmerceexecdef{}{Time.Get.ActualSecond}{Funktion}{}{}{}{}

\wancidmerceexecparamdefbegin
\wancidmerceexecparamdef{}{}{}
\wancidmerceexecparamdefend

\wancidmerceexecex{}{}{}{}

\newpage
\wancidmerceexecdef{}{Time.Select.ActualDay}{Funktion}{}{}{}{}

\wancidmerceexecparamdefbegin
\wancidmerceexecparamdef{}{}{}
\wancidmerceexecparamdefend

\wancidmerceexecex{}{}{}{}

\newpage
\wancidmerceexecdef{}{Time.Select.ActualMonth}{Funktion}{}{}{}{}

\wancidmerceexecparamdefbegin
\wancidmerceexecparamdef{}{}{}
\wancidmerceexecparamdefend

\wancidmerceexecex{}{}{}{}

\newpage
\wancidmerceexecdef{}{Time.Select.ActualYear}{Funktion}{}{}{}{}

\wancidmerceexecparamdefbegin
\wancidmerceexecparamdef{}{}{}
\wancidmerceexecparamdefend

\wancidmerceexecex{}{}{}{}

\newpage
\wancidmerceexecdef{}{Time.Select.ActualHour}{Funktion}{}{}{}{}

\wancidmerceexecparamdefbegin
\wancidmerceexecparamdef{}{}{}
\wancidmerceexecparamdefend

\wancidmerceexecex{}{}{}{}

\newpage
\wancidmerceexecdef{}{Time.Select.ActualMinute}{Funktion}{}{}{}{}

\wancidmerceexecparamdefbegin
\wancidmerceexecparamdef{}{}{}
\wancidmerceexecparamdefend

\wancidmerceexecex{}{}{}{}

\newpage
\wancidmerceexecdef{}{Time.Select.ActualSecond}{Funktion}{}{}{}{}

\wancidmerceexecparamdefbegin
\wancidmerceexecparamdef{}{}{}
\wancidmerceexecparamdefend

\wancidmerceexecex{}{}{}{}

\newpage
\subsection{OS}

\newpage
\wancidmerceexecdef{}{OS.File.Exists}{Funktion}{}{}{}{}

\wancidmerceexecparamdefbegin
\wancidmerceexecparamdef{}{}{}
\wancidmerceexecparamdefend

\wancidmerceexecex{}{}{}{}

\newpage
\wancidmerceexecdef{}{OS.File.Upload}{Trigger}{}{}{}{}

\wancidmerceexecparamdefbegin
\wancidmerceexecparamdef{}{}{}
\wancidmerceexecparamdefend

\wancidmerceexecex{}{}{}{}

\newpage
\subsection{Merchant}

\newpage
\subsubsection{Merchant.Article}

\newpage
\wancidmerceexecdef{}{Merchant.Article.isSpecialOffer}{Funktion}{ab 2.2.2}{}{}{}

\wancidmerceexecparamdefbegin
\wancidmerceexecparamdef{artNo}{}{}
\wancidmerceexecparamdefend

\wancidmerceexecex{}{}{}{}

\newpage
\wancidmerceexecdef{}{Merchant.Article.GetPriceNet}{Funktion}{ab 2.2.2}{}{}{}

\wancidmerceexecparamdefbegin
\wancidmerceexecparamdef{artNo}{}{}
\wancidmerceexecparamdefend

\wancidmerceexecex{}{}{}{}

\newpage
\wancidmerceexecdef{}{Merchant.Article.GetPriceGross}{Funktion}{ab 2.2.2}{}{}{}

\wancidmerceexecparamdefbegin
\wancidmerceexecparamdef{artNo}{}{}
\wancidmerceexecparamdefend

\wancidmerceexecex{}{}{}{}

\newpage
\wancidmerceexecdef{}{Merchant.Article.GetTaxRate}{Funktion}{ab 2.2.2}{}{}{}

\wancidmerceexecparamdefbegin
\wancidmerceexecparamdef{artNo}{}{}
\wancidmerceexecparamdefend

\wancidmerceexecex{}{}{}{}

\newpage
\wancidmerceexecdef{}{Merchant.Article.GetVAT}{Funktion}{ab 2.2.2}{}{}{}

\wancidmerceexecparamdefbegin
\wancidmerceexecparamdef{artNo}{}{}
\wancidmerceexecparamdefend

\wancidmerceexecex{}{}{}{}

\newpage
\subsubsection{Merchant.Basket}

\newpage
\wancidmerceexecdef{}{Merchant.Basket.hasArticle}{Funktion}{ab 2.2.2}{}{}{}

\wancidmerceexecparamdefbegin
\wancidmerceexecparamdef{artNo}{}{}
\wancidmerceexecparamdefend

\wancidmerceexecex{}{}{}{}

\newpage
\wancidmerceexecdef{}{Merchant.Basket.PutArticle}{Funktion / Trigger}{ab 2.2.2}{}{}{}

\wancidmerceexecparamdefbegin
\wancidmerceexecparamdef{artNo}{}{}
\wancidmerceexecparamdef{qty}{}{}
\wancidmerceexecparamdefend

\wancidmerceexecex{}{}{}{}

\newpage
\wancidmerceexecdef{}{Merchant.Basket.SetQty}{Funktion / Trigger}{ab 2.2.2}{}{}{}

\wancidmerceexecparamdefbegin
\wancidmerceexecparamdef{artNo}{}{}
\wancidmerceexecparamdef{qty}{}{}
\wancidmerceexecparamdefend

\wancidmerceexecex{}{}{}{}

\newpage
\wancidmerceexecdef{}{Merchant.Basket.IncQty}{Funktion / Trigger}{ab 2.2.2}{}{}{}

\wancidmerceexecparamdefbegin
\wancidmerceexecparamdef{artNo}{}{}
\wancidmerceexecparamdef{qty}{}{}
\wancidmerceexecparamdefend

\wancidmerceexecex{}{}{}{}

\newpage
\wancidmerceexecdef{}{Merchant.Basket.DecQty}{Funktion / Trigger}{ab 2.2.2}{}{}{}

\wancidmerceexecparamdefbegin
\wancidmerceexecparamdef{artNo}{}{}
\wancidmerceexecparamdef{qty}{}{}
\wancidmerceexecparamdefend

\wancidmerceexecex{}{}{}{}

\newpage
\wancidmerceexecdef{}{Merchant.Basket.SumQty}{Funktion / Trigger}{ab 2.2.2}{}{}{}

\wancidmerceexecparamdefbegin
\wancidmerceexecparamdef{artNo}{}{}
\wancidmerceexecparamdef{qty}{}{}
\wancidmerceexecparamdefend

\wancidmerceexecex{}{}{}{}

\newpage
\wancidmerceexecdef{}{Merchant.Basket.RemoveArticle}{Funktion / Trigger}{ab 2.2.2}{}{}{}

\wancidmerceexecparamdefbegin
\wancidmerceexecparamdef{artNo}{}{}
\wancidmerceexecparamdefend

\wancidmerceexecex{}{}{}{}

\newpage
\wancidmerceexecdef{}{Merchant.Basket.Remove}{Funktion / Trigger}{ab 2.2.2}{}{}{}

\wancidmerceexecparamdefbegin
\wancidmerceexecparamdef{keine}{}{}
\wancidmerceexecparamdefend

\wancidmerceexecex{}{}{}{}

\newpage
\wancidmerceexecdef{}{Merchant.Basket.GetSumNet}{Funktion}{ab 2.2.2}{}{}{}

\wancidmerceexecparamdefbegin
\wancidmerceexecparamdef{curr}{}{}
\wancidmerceexecparamdefend

\wancidmerceexecex{}{}{}{}

\newpage
\wancidmerceexecdef{}{Merchant.Basket.GetSumGross}{Funktion}{ab 2.2.2}{}{}{}

\wancidmerceexecparamdefbegin
\wancidmerceexecparamdef{curr}{}{}
\wancidmerceexecparamdefend

\wancidmerceexecex{}{}{}{}

\newpage
\wancidmerceexecdef{}{Merchant.Basket.GetSumVAT}{Funktion}{ab 2.2.2}{}{}{}

\wancidmerceexecparamdefbegin
\wancidmerceexecparamdef{curr}{}{}
\wancidmerceexecparamdefend

\wancidmerceexecex{}{}{}{}

\newpage
\subsection{Messaging}

\newpage
\wancidmerceexecdef{}{Messaging.Email.Send}{Funktion / Trigger}{}{}{}{}

\wancidmerceexecparamdefbegin
\wancidmerceexecparamdef{}{}{}
\wancidmerceexecparamdefend

\wancidmerceexecex{}{}{}{}

\newpage
\wancidmerceexecdef{}{Messaging.Memo}{Funktion / Trigger}{}{}{}{}

\wancidmerceexecparamdefbegin
\wancidmerceexecparamdef{}{}{}
\wancidmerceexecparamdefend

\wancidmerceexecex{}{}{}{}

\newpage
\subsection{Auction}

\newpage
\subsection{Formatierung von Datums- und Zeitangaben}
\label{FormatierungDatumZeit}
\index{Formatierung}

Innerhalb eines Formats sind alle ASCII-Zeichen erlaubt. Diese werden
1:1 ausgegeben.

\medskip Folgende Formatelemente sind bekannt:

\begin{tabular*}{128mm}{l p{100mm}}
\hline
\wancitableheader{Bezeichner} & \wancitableheader{Beschreibung}\\
\hline
\%a & Der systemspezifische Wert f\"ur den abgek\"urzten Wochentag.\\
\hline
\%A & Der systemspezifische Wert f\"ur den Wochentag.\\
\hline
\%b & Der systemspezifische Wert f\"ur den agek\"urzten Monat.\\
\hline
\%B & Der systemspezifische Wert f\"ur den Monat.\\
\hline
\%c & Der systemspezifische Wert f\"ur Datum und Zeit.\\
\hline
\%d & Der Tag im Monat als Ganzzahl {[}01, 31{]}.\\
\hline
\%H & Die Stunde (24-h Format) als Ganzzahl {[}00, 23{]}.\\
\hline
\%I & Die Stunde (12-h Format) als Ganzzahl {[}00, 12{]}.\\
\hline
\%j & Der Tag im Jahr als Ganzzahl {[}001, 366{]}.\\
\hline
\%m & Der Monat als Ganzzahl {[}01, 12{]}.\\
\hline
\%M & Die Minute als Ganzzahl {[}00, 59{]}.\\
\hline
\%p & Die Systemspezifische Entsprechung f\"ur AM und PM.\\
\hline
\%S & Die Sekunde als Ganzzahl {[}00, 59{]}.\\
\hline
\%U & Die Nummer der Woche im Jahr als Ganzzahl (Sonntag ist der erste 
      Tag in der Woche) als Ganzzahl {[}00, 53{]}. Alle Tage eines neuen
      Jahres vor dem ersten Sonntag im Jahr liegen in der Woche 0.\\
\hline
\%w & Der Wochentag als Ganzzahl {[}0 (Sonntag), 6 (Samstag){]}.\\
\hline
\%W & Die Nummer der Woche im Jahr als Ganzzahl (Montag ist der erste
      Tag in der Woche) als Ganzzahl {[}00,53{]}. Alle Tage eines neuen
      Jahres vor dem ersten Montag im Jahr liegen in der Woche 0.\\
\hline
\%x & Der Systemspezifische Wert f\"ur das Datum.\\
\hline
\%X & Der Systemspezifische Wert f\"ur die Zeit.\\
\hline
\%y & Das Jahr im Jahrhundert als Ganzzahl {[}00, 99{]}.\\
\hline
\%Y & Die Jahreszahl als Ganzzahl.\\
\hline
\%Z & Die Bezeichnung der Zeitzone\\
\hline
\%\% & Zur Ausgabe eines Prozentzeichens.\\
\hline
europe & Alias f\"ur \%D.\%M.\%Y \%H:\%M:\%S\\
\hline
ISO & Alias f\"ur \%Y-\%M-\%D \%H:\%M:\%S\\
\hline
\end{tabular*}


%%
\chapter{Sicherheit}

\newpage

\section{1{[}SAM{]}: Secure Area Management}
\label{SAM}

SAM stellt folgende Daten \"uber den Client und den authentifizierten
Benutzer zur Verf\"ugung \wancisee{BibMakrowho}:

\begin{enumerate}
\item Datensatz des eingeloggten Users \wancisee{BibExecLoginUserCheck};
  Beispiel: \wancicode{(who $<$Feld$>$)}, wobei $<$Feld$>$ jedes Feld
    aus der Tabelle sein kann, woraus der User authentifiziert wurde
\item IP-Adresse
\item Browser-Software/User-Agent
\end{enumerate}

\newpage

\section{1{[}UXS{]}: Unified Access System}
\label{UXS}
\label{UXSI}

\newpage

\section{1{[}DGP{]}: Dawn Good Privacy}
\subsection{Kryptographie? Paranoia?}
\label{KryptographieParanoia}

\begin{quote}
  Es gibt eigentlich nur zwei Arten von Kryptographie: die eine h"alt
  Ihre kleine Schwester davon ab, Ihre Daten zu lesen, w"ahrend die
  andere selbst einflu"sreichen Regierungen den Zugang zu Ihren Daten
  verwehrt. [...]
  
  Wenn ich von Ihnen verlange, einen Brief zu lesen, den ich in einen
  Safe gelegt und dann irgendwo in New York versteckt habe,
  \textit{dann hat das nichts mit Sicherheit zu tun, sondern ist ein
    einfaches Verstecken.} Wenn ich diesen Brief jedoch in den Safe
  lege, Ihnen dann die Entwicklungspl"ane des Safes und noch hundert
  Safes gleicher Bauart mitsamt ihren Kombinationen gebe, so da"s Sie
  mit Hilfe der weltbesten Safeknacker den Sperrmechanismus ausgiebig
  studieren k"onnen, aber immer noch nicht in der Lage sind, den Safe
  zu "offnen und den Brief zu lesen - \textit{dann ist das
    Sicherheit.} \\
  \cite{schneier96}, S. xvii
\end{quote}

Treffender kann man kaum formulieren, worum es bei Kryptographie geht:
Sicherheit zu gew"ahrleisten, obwohl die beteiligten Mechanismen
offenliegen und f"ur au"senstehende nachzuvollziehen sind; das einzig
geheime ist der Schl"ussel und - das ist die Zielsetzung - die
verschl"usselte Information.

Seit Anfang der neunziger Jahre stellt das Internet seine unbegrenzten
M\"oglichkeiten zur Verf\"ugung. Dies erm\"oglicht jedermann, in
bisher unbekanntem Umfang, sich zu informieren, bzw. Information
auszutauschen. Das zun\"achst explosionsartige Wachstum der sog.
\textit{New Econonmy} beweist das Potential der weltumspannenden
Datenkommunikation f\"ur die Gesch\"aftswelt.

Aber die gro"se Hebelwirkung dieser technischen Entwicklung fordert
auch Mi"sbrauch heraus.  Neben den in den Medien vielbeachteten
Aktionen von \textit{Hackern} oder \textit{Crackern}, die h\"aufig
wenig kommerziellen Schaden (au"ser Leistungserschleichungen)
anrichten und sogar Anla"s bieten, Sicherheitsl\"ucken zu entdecken
und zu schlie"sen, haben sich Betrug und Wirtschaftsspionage im
Bereich der elektronischen Kommunikation in beunruhigendem Ma"se
entwickelt.

Auf dem Gebiet der Wirtschaftsspionage sind nach dem Ende des Kalten
Krieges auch die Geheimdienste (v.a. des westlichen Auslands) aktiv
geworden. Diese Beh\"orden stellen durch ihre jahrzehntelange
Erfahrung mit Geheimhaltungsverfahren eine mehr als ernstzunehmende
Gr\"o"se im Bereich der gesch\"aftliche Datenkommunikation dar.

Solche Entwicklungen werden dadurch beg\"unstigt, da"s das Internet in
den ersten Jahrzehnten seines Bestehens kaum sensible Daten zu
\"ubertragen hatte (die Milit\"ars hatten eigene Kan\"ale !). So kam
es, da"s bei den damals geschaffenen und z.T. noch heute in Gebrauch
befindlichen Protokollen wie \texttt{TCP/IP} keine Ma"snahmen in
Fragen der Sicherheit ergiffen wurden.  Daher wurden im Verlauf der
Jahre zunehmend Anstrengungen unternommen, Sicherheitsfunktionen zu
entwickeln. Der Proze"s, diese Funktionen in die vorhandene Familie
von Protokollen zu integrieren, bzw. bei neuen Protokollen (z.B.
\texttt{IPv6}) zu ber\"ucksichtigen, dauert noch an.

\subsection{Was k\"onnen Sie tun?}

\begin{quote}
  \textit{Wenn jemand paranoid ist, bedeutet das nicht, da"s er keine Verfolger hat.} \\
  \textsc{Amos Oz}
\end{quote}

\begin{enumerate}
  
\item Sowohl das Inter- als auch die verschiedenen Intranets sind
  alles andere als abh\"orsicher. Gehen Sie im Zweifelsfalle immer
  davon aus, da"s jemand Ihre Kommunikation verfolgen kann.
  
\item Kryptographie hat, wie eingangs schon erw\"ahnt, nichts mit
  Verstecken zu tun.  Vertrauen Sie Verfahren, die von
  Kryptographie-Experten eingehend gepr\"uft wurden.  Wenn sich die
  Schw\"achen des Verfahrens in Grenzen halten, ist es besser, diese
  Schw\"achen zu kennen, als ein geheimgehaltenes Verfahren zu
  benutzen.
  
\item Gehen Sie davon aus, da"s ein Geheimdienst, der ein Verfahren
  entwickelt hat und es freiwillig ver\"offentlicht, dieses auch
  brechen kann.
  
\item Seien Sie paranoid. \textit{It's better to be safe than sorry.}

\end{enumerate}

\subsection{Was tut die 1Ci?}

Seit Ende des Jahres 2000 implementiert die 1Ci Verfahren zur
Verschl\"usselung von Daten zur Erg\"anzung von \texttt{dmerce}.  Zur
Zeit sind bereits folgende Verschl\"usselungsalgorithmen
einsatzf\"ahig:

\begin{itemize}
\item DES
\item Triple-DES
\item IDEA
\item AES
\item RSA
\end{itemize}

Im weiteren Verlauf werden weitere Verfahren (z.B. \textit{Signatur})
implementiert werden.

Das langfristige Ziel ist es, neben den erforderlichen weiteren Algorithmen, eine Entwicklungsumgebung
f\"ur kryptographische Komplettl\"osungen zur Verf\"ugung zu stellen, so da"s aufwendige Strukturen
wie \texttt{PKI} (\textit{Public Key Infrastructure}) problemlos und flexibel in Zusammenhang mit 
Datenbank- und Netztechnologien erstellt werden k\"onnen. Voraussetzung daf\"ur ist nat\"urlich,
da"s die zust\"andigen Organisationen die entsprechenden Standards verabschieden.

