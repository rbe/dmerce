%%%
\part{Vorwort}

%%
\chapter{Einleitung}

Diese Dokumentation ist die komplette Beschreibung von dmerce. Sie ist
gedruckt, als DVI, Postscript, HTML und PDF verf\"{u}gbar. Diese
Dokumentation ist nach bestem Wissen erstellt worden. F\"{u}r
eventuelle Fehler in der Dokumentation und er daraus resultierenden
Folgen \"{u}bernimmt die 1Ci GmbH keine Gew\"{a}hrleistung.

Weitere Resourcen zu dmerce:

\begin{itemize}
\item \href{dmerce}{http://www.1ci.de/dmerce}
\item \href{1Ci}{http://www.1ci.de}
\item \href{dmerce-ref}{http://www.1ci.de/dmerce/referenzen}
\end{itemize}

\section{Warum dmerce?}

dmerce stellt Inhalte aus Datenbanken im Web dar.

- Kosten/Nutzen
- leicht zu handhaben
- Geschwindigkeit
- gute kombination aus einem ausgegluegelten system auf der einen seite und wenig
  aber gezieltem aufwand auf der anderen seite (der aufwand liegt dort, wo auch
  die kontrolle liegt; beim HTMLer/Templater)
- gratwanderung zwischen aufwendiger auftragsprogrammierung und einem (nicht zu 
  kompliziertem) framework, bei sehr gutem funktionsumfang
- informationssysteme bestehen zu einem grossen teil daraus, wann wer (authentisierung,
  autohorisierung/rechtesystem) wie (art und weise der sicht auf daten) wo/von wo zugriff
  auf daten erhaelt

... ist aus diesen gruenden eine gute investition fuer die zukunft

TECHNIK:
--------

 Wie und auf welche Datenbank
zugegriffen wird, ergibt sich aus dem Aufruf.

dmerce basiert auf Templates.
Templates sind Vorlagen f\"{u}r Webseiten oder Emails. In die Templates werden
Elemente der dmerce Template Language eingef\"{u}gt. Durch die Verwendung von
Templates k\"{o}nnen viele, in der Struktur \"{a}hnliche, Datenbankabfragen
mit einem einzigen Template ausgegeben werden. Der Aufwand in der Gestaltung
und Erstellung von vielen gleichartigen Seiten entf\"{a}llt. Wird die Datenbank
erg\"{a}nzt oder ge\"{a}ndert sind die neuen Eintr\"{a}ge sofort im Web verf\"{u}gbar,
die aufwendige Pflege von vielen Seiten entf\"{a}llt damit ebenso.

Alle Templates
befinden sich in einem Verzeichnis templates, welches sich im Root des Webservers
befindet. Dieses Verhalten kann in der Konfiguration ver\"{a}ndert werden.
In diesem Verzeichnis k\"{o}nnen wiederum Unterverzeichnisse zur Strukturierung
der Templates eingerichtet werden. Ein Template enth\"{a}lt im Regelfall zwei
Inhalte, die Information \"{u}ber die Zugriffststeuerung und eine Schleifenbedingung,
um auch mehrfache Treffer - aufgrund des Datenbankaufrufs - auf die Datenbank
anzeigen zu k\"{o}nnen. Innerhalb der Schleifenbedingung finden sich Platzhalter
f\"{u}r die Datenbankfelder. In diesen wird der Inhalt eines Datenbankfeldes
ausgegeben.

%%
\section{Konventionen}

Wir beschreiben hier die in diesem Buch verwendeten Schreibweisen. Wir versuchen dabei,
uns an g\"angige Layouts zu halten, damit das Lesen sich entsprechend leicht gestaltet.

\section{Schreibweisen in diesem Buch}

\begin{tabular*}{128mm}{p{50mm} p{70mm}}
\hline
\wancitableheader{Typ} & \wancitableheader{Schrift} \\
\hline
UNIX-Befehle, Programmcode, 
Konfigurationsdirektiven      & \fontsize{9}{10pt}\usefont{OT1}{cmtt}{m}{n}
                                Werden in dieser Schrift gehalten (Courier 8 Punkt)\normalfont \\
\hline
Platzhalter                   & Sehen so aus: \wancicode{$<$platzhalter$>$} und dienen
                                dazu zu zeigen, dass hier ein Wert vom Leser eingesetzt
                                werden soll \\
\hline
Slang der 1Ci                 & W\"orter, die dem Wortschatz der 1Ci entspringen sind
                                \wancislang{in dieser Schrift kursiv} gehalten\\
\hline
\end{tabular*}
