\section{UNIX: Eine kleine gucken-nach}
\label{UNIX}
\index{UNIX}

Wir beschreiben nun alle Themen und Befehle, die Sie im Laufe einer
Installation oder dem Betrieb von dmerce grundlegend brauchen.
Nat\"urlich sei an dieser Stelle ganz klar gesagt, dass dies keine
vollst\"andige UNIX-Referenz ist und auch nicht jahrelange Erfahrung
im Umgang mit dem System ersetzen kann.

\subsection{Filesysteme}

\paragraph{Verzeichnisse}

Sie k\"onnen ein Verzeichnis anlegen in dem Sie den Befehl \wancicode{mkdir}
benutzen:

\wancicodeblock{dmerce@hosty:\~ \$ mkdir test}

\paragraph{Entpacken einer Distribution}

So entpacken Sie eine Distribution:

\wancicodeblock{
  root@hosty:/ \# cd /usr/local/src\\
  root@hosty:/usr/local/src \# tar xzf /path/to/your/download/$<$distribution$>$.tar.gz\\
  root@hosty:/usr/local/src \# cd $<$distribution$>$}
\subsection{User und Gruppen managen}

\subsubsection{User}

\paragraph{Auf Existenz pr\"ufen}

Sie k\"onnen pr\"ufen, ob ein User existiert:

\wancicodeblock{root@hosty:/ \# id $<$gruppe$>$}

Existiert der User, so bekommen Sie folgende Ausgabe:

\wancicodeblock{uid=1002($<$gruppe$>$) gid=1002($<$gruppe$>$)}

Existiert er nicht, bekommen Sie eine entsprechende Fehlermeldung:

\wancicodeblock{id: invalid user name}

\paragraph{Anlegen} Sie k\"onnen User anlegen ....

\medskip

Sun Solaris: \wancicode{root@hosty:/ \# adduser $<$gruppe$>$}\\
Linux: \wancicode{root@hosty:/ \# useradd $<$gruppe$>$}\\
FreeBSD: \wancicode{root@hosty:/ \# pw user add $<$gruppe$>$}

\subsection{Gruppen}

\paragraph{Auf Existenz pr\"ufen} blablabla.....

\wancicodeblock{root@hosty:/ \# grep $<$gruppe$>$ /etc/group\\
  $<$gruppe$>$::1002:}

\paragraph{Anlegen} blablabla...

\medskip

Sun Solaris: \wancicode{root@hosty:/ \# groupadd $<$gruppe$>$}\\
Linux: \wancicode{root@hosty:/ \# addgroup $<$gruppe$>$}\\
FreeBSD: \wancicode{root@hosty:/ \# pw group add $<$gruppe$>$}
