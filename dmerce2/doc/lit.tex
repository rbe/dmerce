\part{Literatur}

\chapter{Literaturverzeichnis}

\section{Logik und Grundlagen}

\subsection{Reine Logik}

\begin{description}
  
\item \textsc{Ebbinghaus, Heinz-Dieter; Flum, J\"org} und
  \textsc{Thomas, Wolfgang}: \\
  \textit{Einf\"uhrung in die mathematische Logik}, \\
  Mannheim u.a.: BI-Wissenschaftsverlag, 1992 \\
  ISBN 3-411-15603-1
  

\item \textsc{Hilbert, David} und \textsc{Ackermann, Wilhelm}: \\
  \textit{Grundz\"uge der theoretischen Logik}, \\
  Berlin u.a.: Springer, 1928 u. sp\"ater
  

\item \textsc{Hilbert, David} und \textsc{Bernays, Paul}: \\
  \textit{Grundlagen der Mathematik} (2 B\"ande), \\
  Berlin u.a.: Springer, $^2$1968-1970

\end{description}

\subsection{Mengenlehre}

\begin{description}
  

\item \textsc{Fraenkel, Abraham A.}: \\
  \textit{Abstract Set Theory}, \\
  Amsterdam: North-Holland, $^2$1961
  

\item \textsc{Jech, Thomas}: \\
  \textit{Set Theory}, \\
  New York u.a.: Academic Press, 1978 \\
  ISBN 0-12-381950-4 \\
  Standard-Referenz zur Mengenlehre
  

\item \textsc{Jech, Thomas}: \\
  \textit{Set Theory}, \\
  Berlin u.a.: Springer, $^2$1997 \\
  ISBN 3-540-63048-1 \\
  2. Auflage

\end{description}

\subsection{Beweistheorie}

\subsection{Modelltheorie}

\begin{description}
  

\item \textsc{Kreisel, Georg} und \textsc{Krivine, Jean-Louis}: \\
  \textit{Modelltheorie. Eine Einf\"uhrung in die mathematische Logik und Grundlagentheorie}, \\
  Berlin u.a: Springer, 1972 \\
  ISBN 3-540-05654-8 \\
  ISBN 0-387-05654-8

\end{description}

\subsection{Rekursionstheorie}

\begin{description}
  

\item \textsc{Hinman, Peter G.}: \\
  \textit{Recursion-theoretic Hierarchies}, \\
  Berlin u.a.: Springer, 1978 \\
  ISBN 3-540-07904-1 \\
  ISBN 0-387-07904-1
  

\item \textsc{Rogers, Hartley}: \\
  \textit{Theory of Recursive Functions and Effective Computability}, \\
  New York u.a.: McGraw-Hill, 1967 \\
  Standard-Referenz zur Rekursions- und Berechnbarkeitstheorie

\end{description}

\subsection{$\lambda$-Kalk\"ul und kombinatorische Logik}

\begin{description}
  

\item \textsc{Barendregt, Hendrik Pieter}: \\
  \textit{The Lambda-Calculus. Its Syntax and Semantics}, \\
  Amsterdam: North-Holland, $^2$1997 \\
  ISBN 0444875085 \\
  Standard-Referenz zum $\lambda$-Kalk\"ul

\end{description}

\subsection{Intuitionismus}

%%%%%%%%%%%%%%%%%%%%%%%%%%%%%%%%%%%%%%%%%%%%%%%%%%%%%%%%%%%%%%%%%%%%%

\section{Theoretische Informatik}

\subsection{Einf\"uhrungen}

\begin{description}

\item \textsc{Wegener, Ingo}: \\
\textit{Theoretische Informatik. Eine algorithmenorientierte 
Einf\"uhrung}, \\
Stuttgrat, Leipzig: Teubner, $^2$1999 \\
ISBN 3-519-12123-9

\end{description}


%%%%%%%%%%%%%%%%%%%%%%%%%%%%%%%%%%%%%%%%%%%%%%%%%%%%%%%%%%%%%%%%%%%%%

\section{Lineare Algebra}

\subsection{Lehrb\"ucher und Gesamtdarstellungen}

\begin{description}

\item \textsc{Koecher, Max}: \\
\textit{Lineare Algebra und Analytische Geometrie}, \\
Berlin u.a.: Springer, $^3$1992 \\
ISBN 3-540-55653-2

\end{description}

%%%%%%%%%%%%%%%%%%%%%%%%%%%%%%%%%%%%%%%%%%%%%%%%%%%%%%%%%%%%%%%%%%%%%
\section{Algebra} 

\subsection{Allgemein}

\begin{description}

\item \textsc{Bosch, Siegfried}: \\
\textit{Algebra}, \\ 
Berlin u.a.: Springer, 1993 \\
ISBN 3-540-56833-6

\item \textsc{Faith, Carl}: \\
\textit{Algebra (1). Rings, Modules, Categories}, \\
Berlin u.a: Springer, 1973 \\
ISBN 3-540-05551-7 \\
ISBN 0-387-05551-7

\item \textsc{Faith, Carl}: \\
\textit{Algebra (2). Ring Theory}, \\
Berlin u.a: Springer, 1976 \\
ISBN 3-540-05705-6 \\
ISBN 0-387-05705-6

\item \textsc{Lang, Serge}: \\ 
\textit{Algebra}, \\
Reading, MA u.a.: Addison-Wesley, $^3$1993 \\
ISBN 0-201-55540-9

\end{description}

\subsection{Kommutative Algebra}

\subsection{Katogorien und Topoi}

\subsection{Gruppen}

\begin{description}

\item \textsc{Hahn, Alexander J.; O'Meara, O. Timothy} \\
\textit{The Classical Groups and K-Theory}, \\
Berlin u.a.: Springer, 1989 \\
ISBN 3-540-17758-2 \\
ISBN 3-387-17758-2

\end{description}

\subsection{Ringe}

\subsection{K\"orper und K\"orpererweiterungen}

\subsection{Galoistheorie}

\subsection{Codierungstheorie}

\begin{description}

\item \textsc{Jungnickel, Dieter}: \\
\textit{Codierungstheorie}, \\
Heidelberg, Berlin, Oxford: Spektrum Akademischer Verlag, 1995 \\
ISBN 3-86025-432-4

\end{description}

%%%%%%%%%%%%%%%%%%%%%%%%%%%%%%%%%%%%%%%

\section{Zahlentheorie} 

\subsection{Allgemein}

\begin{description}

\item \textsc{Bundschuh, Peter}: \\ 
\textit{Einf"uhrung in die Zahlentheorie}, \\
Berlin u.a.: Springer, $^4$1998

\item \textsc{Koblitz, Neal}: \\ 
\textit{A Course in Number Theory and Cryptography}, \\ 
New York u.a.: Springer, 1987

\item \textsc{Lang, Serge}: \\ 
\textit{Algebraic Number Theory}, \\
New York u.a.: Springer, $^2$1994

\item \textsc{Leutbecher, Armin}: \\ 
\textit{Zahlentheorie. Eine Einf"uhrung in die Algebra}, \\ 
Berlin u.a: Springer, 1996

\item \textsc{Neukirch, J"urgen}: \\ 
\textit{Algebraische Zahlentheorie}, \\
Berlin u.a.: Springer 1992

\end{description}

\subsection{Zahlsiebe}

%%%%%%%%%%%%%%%%%%%%%%%%%%%%%%%%%%%%%%%

\section{Geometrie}

\subsection{Gesamtdarstellungen und Lehrb\"ucher}

\begin{description}

\item \textsc{Berger, Marcel}: \\ 
\textit{Geometry} (2 B\"ande),
Berlin u.a.: Springer, 1987 \\
ISBN 3-540-11658-3 (I), 3-540-17015-4 (II) \\
ISBN 0-387-11658-3 (I), 3-540-17015-4 (II)

\item \textsc{Filler, Andreas}: \\
\textit{Euklidische und nichteuklidische Geometetrie}, \\
Mannheim u.a.: BI-Wissenschaftsverlag, 1993 \\
ISBN 3-411-16371-2

\item \textsc{Giering, Oswald}: \\ 
\textit{Vorlesungen \"uber h\"ohere Geometrie}, \\
Braunschweig, Wiesbaden: Vieweg, 1982 \\
ISBN 3-528-08492-8

\item \textsc{Hilbert, David}: \\ 
\textit{Grundlagen der Geometrie}, \\
Stuttgart: Teubner, $^12$1977 \\
ISBN 3-519-22020-2

\item \textsc{Kn\"orrer, Horst}: \\ 
\textit{Geometrie}, \\
Braunschweig, Wiesbaden: Vieweg, 1996 \\
ISBN 3-528-07271-7

\item \textsc{Millman, Richard S.}: \\ 
\textit{Geometry. A Metric Approach with Models}, \\
New York u.a.: Springer, 1981 \\
ISBN 0-387-90610-X \\
ISBN 3-540-90610-X

\item \textsc{Neutsch, Wolfram}: \\ 
\textit{Koordinaten. Theorie und Anwendungen}, \\
Heidelberg: Spektrum Akademischer Verlag, 1995 \\
ISBN 3-96025-398-9

\end{description}

\subsection{Geschichte}

\begin{description}
  

\item \textsc{Karzel, Helmut} und \textsc{Kroll, Hans-Joachim}: \\
  \textit{Geschichte der Geometrie seit Hilbert}, \\
  Darmstadt: Wissenschaftliche Buchgesellschaft, 1988 \\
  ISBN 3-534-08524-8

\end{description}

\subsection{Riemannsche Geometrie}

\begin{description}

\item \textsc{Gallot, Sylvestre; Hulin, Dominique} und 
\textsc{Lafontaine, Jacques}: \\
\textit{Riemannian Geometry}, \\
Berlin u.a.: Springer, 1987 \\
ISBN 0-387-17923-2

\end{description}

%%%%%%%%%%%%%%%%%%%%%%%%%%%%%%%%%%%%%%%

\section{Algebraische Geometrie} 

\subsection{Allgemein}

\begin{description}
  

\item \textsc{Hartshorne, Robin}: \\
  \textit{Algebraic Geometry}, \\
  New York u.a.: Springer, 1993 \\
  ISBN 0-387-90244-9 \\
  ISBN 3-540-90244-9
  

\item \textsc{van der Waerden, Bartel Leendert}: \\
  \textit{Einf"uhrung in die Algebraische Geometrie}, \\
  Berlin u.a.: Springer, $^2$1973

\end{description}

%%%%%%%%%

\subsection{Elliptische Kurven}

\begin{description}
  

\item \textsc{Husem"oller, Dale}: \\
  \textit{Elliptic Curves}, \\
  New York u.a.: Springer, 1987
  

\item \textsc{Knapp, Anthony W.}: \\
  \textit{Elliptic Curves}, \\
  Princetopn NJ: Princeton University Press, 1992 \\
  ISBN 0-691-08559-5
  

\item \textsc{Koblitz, Neal}: \\
  \textit{A Course in Number Theory and
    Cryptography}, \\
  New York u.a.: Springer, 1987
  

\item \textsc{Lang, Serge}: \\
  \textit{Elliptic Curves - Diophantine Analysis}, \\
  Berlin u.a.: Springer, 1978
  

\item \textsc{Lang, Serge}: \\
  \textit{Elliptic Functions}, \\
  New York u.a.: Springer, $^2$1987
  

\item \textsc{Robert, Alain}: \\
  \textit{Elliptic Curves}, \\
  Berlin u.a.: Springer, 1973

\end{description}

%%%%%%%%%

\subsection{Hyperelliptische Kurven}

%%%%%%%%%

\subsection{Quadratische Formen}

\begin{description}
  

\item \textsc{Buchmann, Johannes}: \\
  \textit{Einf"uhrung in die Kryptographie}, \\
  Berlin u.a.: Springer, 1999

\end{description}

%%%%%%%%%%%%%%%%%%%%%%%%%%%%%%%%%%%%%%%

\section{Komplexit"atstheorie}

\begin{description}
  

\item \textsc{Reischuk, K. R"udiger}: \\
  \textit{Komplexit"atstheorie. Band I: Grundlagen. Maschinenmodelle,
    Zeit- und Platzkomplexit"at, Nichtdeterminismus}, \\
  Stuttgart, Leipzig: B.G. Teubner, $^2$1999

\end{description}

%%%%%%%%%%%%%%%%%%%%%%%%%%%%%%%%%%%%%%%

\section{Diskrete Mathematik}

\begin{description}
  

\item \textsc{Beth, Thomas} u.a.: \\
  \textit{Design Theory}, \\
  Mannheim u.a: Bibliographisches Institut, 1985 \\
  ISBN 3-411-01675-2
  

\item \textsc{Jungnickel, Dieter}: \\
  \textit{Graphen, Netzwerke, Algorithmen}, \\
  Mannheim u.a.: BI Wissenschaftsverlag, $^3$1994 \\
  ISBN 3-411-14263-4

\end{description}

%%%%%%%%%

\subsection{Kombinatorik}

\begin{description}
  

\item \textsc{Aigner, Martin}: \\
  \textit{Kombinatorik} 2 B"ande, \\
  Berlin u.a.: Springer, 1975/76
  

\item \textsc{Graver, Jack E.; Watkins, Mark E.}: \\
  \textit{Combinatorics with Emphasis on the Theory of Graphs}, \\
  New York u.a.: Springer, 1977 \\
  ISBN 0-387-90245-7 \\
  ISBN 3-540-90245-7

\end{description}

%%%%%%%%%

\subsection{Graphentheorie}

\begin{description}
  

\item \textsc{Bodendiek, Rainer; Fuhrmann, Hermann A.}: \\
  \textit{Graphen und Computer}, \\
  Heidelberg, Berlin: Spektrum Akademischer Verlag, 1998
  

\item \textsc{Halin, Rudolf}: \\
  \textit{Graphentheorie}, \\
  Darmstadt: Wissenschaftliche Buchgesellschaft, $^2$1989 \\
  ISBN 3-534-10140-5
  

\item \textsc{Tutte, W.T.}: \\
  \textit{Graph Theory}, \\
  Reading MA u.a.: Addison-Wesley, 1984 \\
  0-201-13520-5

\end{description}


%%%%%%%%%%%%%%%%%%%%%%%%%%%%%%%%%%%%%%%%%%%%%%%%%%%%%%%%%%%%%%%%%%%%%%%%%%%%%


\section{Betriebssysteme}

\subsection{UNIX}

\begin{description}
  
\item \textsc{Bach, Maurice J.}: \textit{The Design of the UNIX
    Operating System}, Englewood Cliffs, NJ, London u.a.: Prentice-Hall, 1986 \\
  ISBN 0-13-201799-7
  
\item \textsc{Rochkind, Marc J.}: \textit{UNIX Programmierung f''ur
    Fortgeschrittene. Systemaufrufe von access bis write.
    Unix-Versionen von Berkeley bis System V. Ein- und Ausgabe von
    Datei bis Terminal}, M''unchen, Wien: Hanser; London: Prentice
  Hall
  International, 1988 \\
  ISBN 3-446-15202-4 \\
  ISBN 0-13-011792-7

\end{description}

%%%%%%%%%%%%%%%%%%%%%%%%%%%%%%%%%%%%%%%%%%%%%%%%%%%%%%%%%%%%%%%%%%%%%%%%%%%%

\section{Programmierung}

\subsection{Lehrb\"ucher und Gesamtdarstellungen}

\begin{description}
  
\item \textsc{Knuth, Donald E.}: \textit{The Art of Computer
    Programming.  Vol. 1: Fundamental Algorithms}, Reading MA u.a.:
  Addison-Wesley, 1973

\end{description}

\subsection{Objektorientierung}

\paragraph{Grundlagen und Theorie}

\begin{description}

\item \textsc{Abadi, Mart\'{\i}n; Cardelli, Luca}:
\textit{A Theory of Objects},
New York u.a.: Springer, 1996 \\
ISBN 0-387-94775-2

\item \textsc{Castagna, Giuseppe}:
\textit{Object-Oriented Programming. A Unified Foundation},
Boston, Basel, Berlin: Birkh\"auser, 1997 \\
ISBN 0-8176-3905-5 \\
ISBN 3-7643-3905-5

\end{description}

\paragraph{UML}

\begin{description}

\item \textsc{Martin, James; Odell, James J.}:
\textit{Objektorientierte Modellierung mit UML: Das Fundament},
M\"unchen [i.e.] Haar u.a.: Prentice Hall, 1999 \\
ISBN 3-8272-9580-7

\end{description}

\paragraph{Analyse und Design}

\begin{description}
  
\item \textsc{Balzert, Heide}: \textit{Lehrbuch der
    Objektmodellierung. Analyse und Entwurf},
  Heidelberg, Berlin: Spektrum Akademischer Verlag, 1999 \\
  ISBN 3-8274-0285-9

\end{description}

\paragraph{Programmierung}

\begin{description}

\item \textsc{Claussen, Ute}:
\textit{Objektorientiertes Progammieren},
Berlin u.a.: Springer, 1998 \\
ISBN 3-540-57937-0

\end{description}


%%%%%%%%%%%%%%%%%%%%%%%%%%%%%%%%%%%%%%%

\section{Assembler (Mnemo-Code)}

\subsection{Allgemein}

\subsection{Alpha-Assembler}

\subsection{Intel-Assembler}

\begin{description}
  
\item \textsc{Dieterich, Ernst-Wolfgang}: \textit{Assembler.
    Grundlagen der PC-Programmierung},
  M\"unchen, Wien: Oldenbourg, $^4$2000 \\
  ISBN 3-486-25198-8
  
\item \textsc{Kammerer, Peter}: \textit{Von Pascal zu Assembler. Eine
    Einf\"uhrung in die maschinennahe Programmierung f\"ur Intel und
    Motorola},
  Braunschweig, Wiesbaden: Vieweg, 1998 \\
  ISBN 3-528-05590-1
  
\item \textsc{Link, Wolfgang}: \textit{Assembler Programmierung.  Eine
    gr''undliche Einf''uhrung. 8086, 80286 bis 80486 und Pentium},
  Poing: Franzis, $^7$1995 \\
  ISBN 3-7723-8837-X

\end{description}

\subsection{Motorola-Assembler}

\subsection{SPARC-Assembler}

%%%%%%%%%%%%%%%%%%%%%%%%%%%%%%%%%%%%%%%

\section{LISP}

\subsection{Allgemein}

\begin{description}

\item \textsc{Stoyan, Herbert}:
\textit{LISP - Anwendungsgebiete, Grundbegriffe, Geschichte},
Berlin: Akademie Verlag, 1980

\end{description}

\subsection{Common LISP}

\begin{description}
  
\item \textsc{Mayer, Otto}: \textit{Programmieren in Common Lisp},
  Mannheim u.a.: BI-Wissenschaftsverlag, 1988 \\
  ISBN 3-411-00638-2
  

\item \textsc{Steele, Guy L. Jr.} \\
  \textit{Common LISP, The Language\@. Second Edition} \\
  ?: Digital Press, $^2$1990 \\
  ISBN 1-55558-041-6

\end{description}


\subsection{Scheme}

%%%%%%%%%%%%%%%%%%%%%%%%%%%%%%%%%%%%%%%

\section{C/C++}

\begin{description}

\item \textsc{Louis, Dirk}: \textit{C/C++. Umfassendes Grundlagenwerk
mit Profi-Referenz},
Haar bei M\"unchen: Markt \& Technik, 1998

\item \textsc{Louis, Dirk}: \textit{C/C++ Kompendium. Bew''ahrtes, umfassendes
Grundlagenwerk[...]. Praxisnahe Beschreibung. Einf\"uhrung in Windows,
Assembler OOD und andere interessante Themen}, M\"unchen: Markt \& Technik, 2000 \\
ISBN 3-8272-5669-0

\end{description}

%%%%%%%

\subsection{C}

\begin{description}
  
\item \textsc{Kernighan, Brian W.; Ritchie, Dennis M.}:
  \textit{Programmieren in C. Mit dem C-Reference Manual in deutscher
    Sprache. Zweite Ausgabe
    ANSI C}, M''unchen, Wien: Hanser; London: Prentice-Hall, 1990 \\
  ISBN 3-446-15497-3 \\
  ISBN 0-13-110330-X
  
\item \textsc{Plauger, P.J.; Brodie, James}: \textit{Referenzhandbuch
    Standard C. F\"ur den neuen ANSI- und ISO-
    Standard}, \\
  Braunschweig u.a.: Vieweg, 1990 \\
  ISBN 3-528-04741-0
  

\item \textsc{Plum, Thomas}: \\
  \textit{Das C-Lernbuch}, \\
  M\"unchen u.a.: Hanser; London: Prentice Hall Intl., 1985 \\
  ISBN 3-446-14165-0 \\
  ISBN 0-13-527896-1
  
\item \textsc{Willms, Andr\'e}: \textit{C-Programmierung.
    Programmiersprache, Programmiertechnik, Datenorganisation},
  Bonn u.a.: Addison-Wesley, 1996 \\
  ISBN 3-3-8273-1112-8

\end{description}

%%%%%%%

\subsection{C++}

\begin{description}
  
\item \textsc{Josuttis, Nicolai}: \textit{Die C++-Standardbibliothek.
    Eine detaillierte Einf''uhrung in die vollst''andige
    ANSI/ISO-Schnittstelle},
  Bonn u.a.: Addison-Wesley, 1996 \\
  ISBN 3-8273-1023-7
  
\item \textsc{Seed, Graham}: \textit{An Introduction into
    Object-Oriented Programming in C++. With Applications in Computer
    Graphics}, Berlin u.a.:
  Springer, 1996 \\
  ISBN 3-540-76042-3
  
\item \textsc{Stroustrup, Bjarne}: \textit{Die C++ Programmiersprache.
    3. Aufl.},
  Bonn u.a.: Addisos-Wesley, $^3$1998 \\
  ISBN 0-201-88954-4

\end{description}

%%%%%%%%%%%%%%%%%%%%%%%%%%%%%%%%%%%%%%%

\section{Python}

\textbf{WWW}

\begin{description}
  
\item \texttt{http://www.python.org}

\end{description}

\noindent 
\textbf{Offline}

\begin{description}
  
\item \textsc{Himstedt, Tobias; M"atzel, Klaus}: \textit{Mit Python
    programmieren. Einf"uhrung und Anwendung skriptorientierter
    Programmierung},
  Heidelberg: dpunkt-Verlag, 1999 \\
  ISBN 3-920993-85-3
  
\item \textsc{Lutz, Mark}: \textit{Programming Python},
  Bonn u.a.: O'Reilly, 1996 \\
  ISBN 1-56592-197-6

\end{description}


\section{Kryptographie}

\textbf{WWW}

\begin{description}
  

\item \texttt{http://www.media-crypt.com} \\
  IDEA-Site mit Links zu vielen weiteren kryptographierelevanten
  Seiten
  

\item \texttt{http://www.rsa.com} \\
  HP der \textit{RSA Data Security Inc.}
  

\item \texttt{http://www.uni-siegen.de/security} \\
  \textit{Universit"at Siegen}

\end{description}

\noindent \textbf{Usenet}

\begin{description}
  
\item \texttt{sci.crypt}
  

\item \texttt{sci.crypt.research} \\
  Im Gegensatz zu \texttt{sci.crypt} moderiert, daher qualitativ
  bessere Beitr"age. Andererseits ist hier auch weniger los.
  
\item \texttt{alt.privacy}
  
\item \texttt{comp.security.misc}

\end{description}

%%%%%%%%%%%%%%%%%%%%%%%%%%%%%%%%%%%%%%%

\subsection{Einf"uhrungen, Gesamtdarstellungen und Standardwerke}

\begin{description}
  
\item \textsc{Buchmann, Johannes}: \textit{Einf"uhrung in die
    Kryptographie}, Berlin u.a.: Springer, 1999
  
\item \textsc{Koblitz, Neal}: \textit{A Course in Number Theory and
    Cryptography}, New York u.a.: Springer, 1987
  
\item \textsc{Schneier, Bruce}: \textit{Angewandte Kryptographie.
    Protokolle, Algorithmen und Sourcecode in C},
  Bonn, Reading MA u.a.: Addison-Wesley, 1996 \\
  ISBN 3-89319-854-7 \\
  Das wohl umfangreichste Buch zum Thema Kryptographie, das jede, aber
  auch jede Frage zumindest mit brauchbaren Literaturhinweisen
  beantworten kann. Allein das Literaturverzeichnis enth"alt 1653(!)
  Eintr"age!
  
\item \textsc{Welsh, Dominic}: \textit{Codes und Kryptographie},
  Weinheim u.a.: VCH, 1991
  
\item \textsc{Wobst, Reinhard}: \textit{Abenteuer Kryptographie.
    Methoden, Risiken und Nutzen der Datenverschl"usselung},
  Bonn, Reading MA u.a.: Addison-Wesley, 1998 \\
  ISBN 3-8273-1413-5


\end{description}

%%%%%%%%%%%%%%%%%%%%%%%%%%%%%%%%%%%%%%%%%%%%%%%%%%%%%%%%%%%%%%%%%%%%%

\section{Algorithmen im Einzelnen}


\subsection{Blockchiffren}

%%%%%%%%%%%%%%%%%%%%%%%%%%%%%%%%%%%%%%%

\paragraph{DES}

\texttt{http://csrc.nist.gov}

%%%%%%%%%%%%%%%%%%%%%%%%%%%%%%%%%%%%%%%

\paragraph{IDEA}

\textbf{WWW}

\begin{description}

\item \texttt{http://www.media-crypt.com}

\end{description}

%%%%%%%%%%%%%%%%%%%%%%%%%%%%%%%%%%%%%%%

\subsection{Public Key}

\textbf{WWW}

\begin{description}
  

\item \texttt{http://www.rsa.com} \\
  \textit{RSA Data Security Inc.}
  

\item \texttt{http://www.rsa.com/rsalabs/pubs/PKCS} \\
  Der PKCS (\textit{Public-Key Cryptography Standard}) Pseudo-Standard
  f"ur Public-Key Kryptographie, den RSA entworfen hat, bevor ANSI,
  oder gar die NSA, auf die Idee kommen, die Sache zu erledigen...

\end{description}

%%%%%%%%%%%%%%%%%%%%%%%%%%%%%%%%%%%%%%%

\subsection{Implementierungen und Standards}

%%%%%%%%%

\paragraph{PGP}

\textbf{WWW}

\begin{description}
  

\item \texttt{http://www.pgp.com} \\
  Amerikanische Site
  

\item \texttt{http://www.pgpi.com} \\
  Internationale Site

\end{description}

\noindent \textbf{FTP}

\begin{description}
  

\item \texttt{ftp://ftp.ox.ac.uk} \\
  Besonders vollst"andige und aktuelle Sammlung von PGP-Resourcen.

\end{description}

\noindent \textbf{Usenet}

\begin{description}
  
\item \texttt{alt.security.pgp}

\end{description}

\noindent \textbf{Offline}

\begin{description}
  
\item \textsc{Garfinkel, Simson}: \textit{PGP. Pretty Good Privacy},
  Sebastopol CA: O'Reilly, 1995 \\
  ISBN 1-56592-098-8
  
\item \textsc{Stallings, William}: \textit{Datensicherheit mit PGP},
  M"unchen u.a.: Prentice Hall, 1995 \\
  ISBN 3-930436-28-0
  
\item \textsc{Zimmermann, Phil R.}: \textit{The Official PGP User's
    Guide}, Boston: MIT Press, 1995
  
\item \textsc{Zimmermann, Phil R.}: \textit{PGP Source Code and
    Internals}, Boston: MIT Press, 1995

\end{description}

%%%%%%%%%%%%%%%%%%%%%%%%%%%%%%%%%%%%%%%%%%%%%%%%%%%%%%%%%%%%%%%%%%%%%%%%%%%%%%%%%%%

\subsection{Weitere Sicherheitslinks}

\textbf{WWW}

\begin{description}
  

\item \texttt{http://www.bsi.de} \\
  \textit{Bundesamt f"ur Sicherheit in der Informationstechnik}
  
\item \texttt{http://www.cs.purdue.edu/coast/hotlist}
  

\item \texttt{http://www.icsa.com} \\
  \textit{International Computer Security Association} v.a. Firewalls.
  

\item \texttt{http://www.alw.nih.gov/Security/security.html} \\
  \textit{National Institutes of Health}
  

\item \texttt{http://www.cs.princeton.edu/sip} \\
  \textit{Princeton Security Internet Programming}
  

\item \texttt{http://www.cs.purdue.edu/homes/spaf/hotlists/csec-top.html} \\
  Umfangreiche Sammlung von \textsc{Gene Spafford}
  

\item \texttt{http://www.internet-security.de} \\
  \textit{Internet Security-Forum}

\end{description}

\noindent 
\textbf{Usenet}

\begin{description}
  
\item \texttt{alt.privacy}
  
\item \texttt{comp.security.misc}

\end{description}

\noindent 
\textbf{Offline}

\begin{description}
  
\item \textsc{Fuhrberg, Kai}: \textit{Internet-Sicherheit. Browser,
    Firewalls und Verschl"usselung}, M"unchen, Wien: Carl Hanser
  Verlag, 1998
  
\item \textsc{Garfinkel, Simson; Spafford, Gene}: \textit{Web Security
    \& Commerce},
  Cambridge u.a: O'Reilly, 1997 \\
  ISBN 1-56592-269-7
  
\item \textsc{Raepple, Martin}: \textit{Sicherheitskonzepte f"ur das
    Internet. Grundlagen, Technologien und L"osungskonzepte f"ur die
    kommerzielle Nutzung}, Heidelberg: dpunkt-Verlag, 1998
  
\item \textsc{Stallings, William}: \textit{Sicherheit im Datennetz},
  M"unchen u.a.: Prentice Hall, 1995 \\
  ISBN 3-930436-29-9

\end{description}

%%%%%%%%%

\paragraph{Sicherheitsstandards: NIST, Rainbow Books und andere}

\textbf{WWW}

\begin{description}
  

\item \texttt{http://csrc.nist.gov/secpubs/rainbow} \\
  Die Rainbow Books.
  

\item \texttt{http://csrc.nist.gov/secpubs/itsec.txt} \\
  Die \textit{Information Technology Security Evaluation Criteria}
  

\item \texttt{http://csrc.nist.gov/cc} \\
  Die \textit{Common Criteria}

\end{description}

%%%%%%%%%%%%%%%%%%%%%%%%%%%%%%%%%%%%%%%

\subsection{Hacken, Cracken, Underground}

\textbf{WWW}

\begin{description}

\item \texttt{http://www.ccc.de} \\
  HP des Chaos Computer Clubs
  
\item \texttt{http://www.technotronic.com}
  
\item \texttt{http://www.nmrc.org}
  
\item \texttt{http://www.digicrime.com}
  
\item \texttt{http://www.ilf.net}
  
\item \texttt{http://10pht.com}
  
\item \texttt{http://www.2600.org}
  
\item \texttt{http://www.ntbugtraq.com}
  
\item \texttt{http://www.kryptocrew.de}
  
\item \texttt{http://www.paramind.de}


\end{description}

%%%%%%%%%%%%%%%%%%%%%%%%%%%%%%%%%%%%%%%%

\subsection{Viren, W''urmer und Trojaner}

\begin{description}

\item \textsc{Kane, Pamela}: \textit{PC Security and Virus Protection
Handbook. The Ongoing War Against Information Sabotage}, New York:
M \& T Books, 1994 \\
1-55851-390-6

\end{description}



\section{Netzwerke}

\subsection{RFC, Dokumente und Standards}

\textbf{WWW}

\begin{description}
  

\item \texttt{http://www.ietf.org} \\
  Quelle f"ur RFCs
  

\item \texttt{http://www.isi.edu} \\
  \textit{Information Sciences Institute, Univ. of Southern
    California}: Internet-Standards, Protokolle, RFCs u.a.
  

\item \texttt{http://info.internet.isi.edu:80/in-notes/rfc} \\
  Ein durchsuchbares Archiv f"ur RFCs von \textsc{Jon Postel}
  

\item \texttt{http://www.cis.ohiostate,edu:80/hypertext/information/rfc.html} \\
  RFCs und andere technische Dokumente.

\end{description}

\noindent \textbf{Offline}

\begin{description}
  
\item \textsc{Hafner, Katie; Lyon, Matthew}: \textit{Arpa Kadabra. Die
    Geschichte des Internet}, Heidelberg: dpunkt-Verlag, 1997

\end{description}

%%%%%%%%%%%%%%%%%%%%%%%%%%%%%%%%%%%%%%%

\subsection{TCP/IP}

\begin{description}
  
\item \textsc{Hein, Mathias}: \textit{TCP/IP.  Internet-Protokolle im
    professionellen Einsatz},
  Bonn u.a.: Intl. Thomson Publ., $^4$1998 \\
  ISBN 3-8266-4035-7

\end{description}

%%%%%%%%%%%%%%%%%%%%%%%%%%%%%%%%%%%%%%%

\subsection{Ethernet}

\begin{description}
  
\item \textsc{Held, Gilbert}: \textit{Ethernet Networks.  Third
    Edition. From 10Base-T to Gigabit}, New York u.a.:
  Wiley, $^3$1998 \\
  ISBN 0-471-25310-3

\end{description}

%%%%%%%%%%%%%%%%%%%%%%%%%%%%%%%%%%%%%%%%%%%%%%%%%%%%%%%%%%%%%%%%%%%%%%%%%%%%%%%%%%%%%%%%%

\section{Datenbanken}

\begin{description}
  

\item \textsc{Vossen, Gottfried}: \\
  \textit{Datenmodelle, Datenbanksprachen und Datenbankmanagementsysteme}, \\
  M\"unchen, Wien: Oldenbourg, $^4$2000 \\
  ISBN 3-486-25339-5

\end{description}

\subsection{Objektorientierte Datenbanken}

\begin{description}  

\item \textsc{Heuer, Andreas}: \\
  \textit{Objektorientierte Datenbanken. Konzepte, Modelle, Standards und Systeme}, \\
  Bonn u.a.: Addison-Wesley, $^2$1997 \\
  ISBN 3-89319-800-8

\end{description}

\subsection{Einzelne Implementationen}

\paragraph{Oracle}

\begin{description}
  
\item \textsc{Loney, Kevin}: \\
  \textit{Oracle 8. DBA-Handbuch. Version 7 bis Version 8}, \\
  M\"unchen, Wien: Hanser, 1999 \\
  ISBN 3-446-21167-5

\end{description} 

\section{Graphik}

\begin{description}
  
\item \textsc{Barth, Rainer; Beier, Ekkehard} und \textsc{Bahnke,
    Bettina}: \textit{Graphikprogrammierung mit OpenGL},
  Bonn u.a.: Addison-Wesley, 1996 \\
  ISBN 3-89319-975-6
  
\item \textsc{Brugger, Ralf}: \textit{Professionelle Bildgestaltung in
    der 3D-Computergraphik.  Grundlagen und Prinzipien f\"uer eine
    ausdrucksstarke Computer-Visualisierung},
  Bonn u.a.: Addison-Wesley, 1995 \\
  ISBN 3-89319-706-0
  
\item \textsc{Fellner, Wolf Dietrich}: \textit{Computergraphik},
  Mannheim u.a.: BI-Wissenschaftsverlag, $^2$1992 \\
  ISBN 3-411-15122-6
  
\item \textsc{Grieger, Ingolf}: \textit{Graphische Datenverarbeitung:
    Mathematische Methoden. Rechnerunterst\"utztes Entwerfen mit
    Geometriezellen},
  Berlin u.a.: Springer, 1987 \\
  ISBN 3-540-17895-3 \\
  ISBN 0-387-17895-3
  
\item \textsc{Luther, Wolfram; Ohsmannm, Martin}:
  \textit{Mathematische Grundlagen der Computergraphik},
  Braunschweig, Wiesbaden: Vieweg, 1988 \\
  ISBN 3-528-06302-5
  
\item \textsc{Marsh, Duncan}: \textit{Applied Geometry for Computer
    Graphics and CAD},
  London u.a.: Springer, 1999 \\
  ISBN 1-85233-080-5
  
\item \textsc{M\"uller, Heinrich}: \textit{Realistische
    Computergraphik. Algorithmen, Datenstrukturen und Maschinen},
  Berlin u.a.: Springer, 1988 \\
  ISBN 3-540-18924-6 \\
  ISBN 0-387-18924-6
  
\item \textsc{Wisskirchen, Peter}: \textit{Objectoriented Graphics.
    From GKS and PHIGS to Object-Oriented Systems},
  Berlin u.a.: Springer, 1990 \\
  ISBN 3-540-52859-8 \\
  ISBN 0-387-52859-8

\end{description}
