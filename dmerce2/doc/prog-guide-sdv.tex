%%  SDV UserDoc BEGINN %%
\chapter{1{[}SDV{]}: Statistical Data Viewer}

\section{Ein \"Uberblick}

Zum schnellen Zurechtfinden geben wir einen Beispielaufruf, wie er in
einem Python-Script zur Benutzung der SDV-API gegeben werden k\"onnte:

\begin{small}
\begin{quote}
\begin{verbatim}

##############################################################
# Ein dmerce-SDV Beispielaufruf
##############################################################

\end{verbatim}
\end{quote}
\end{small}

Erzeuge eine Farbliste (z.B. im RGB-Format) (nicht obligatorisch - man
kann selbstverst\"andlich jede andere geeignete Liste verwenden)

\begin{small}
\begin{quote}
\begin{verbatim}

bglist = []
for i in range(0, 50, 2):
    bglist.append((0, 0, 200))
    bglist.append((150, 150, 150))

\end{verbatim}
\end{quote}
\end{small}

Erzeuge Listen von x- und y-Werten (die Listen sollten dieselbe Laenge
haben - andernfalls erzeugt der folgende Schritt eine Fehlermeldung)

\begin{small}
\begin{quote}
\begin{verbatim}

x = ['x-Wert 1', 'x-Wert 1, 'x-Wert 1, 'x-Wert 1, 'x-Wert 1,
     'x-Wert 1', 'x-Wert 1, 'x-Wert 1, 'x-Wert 1, 'x-Wert 1,]

y1 = [26, 5, 15, 7, 6, 9, 12, 8, 6, 6]

\end{verbatim}
\end{quote}
\end{small}

Initialisiere ein Datenobjekt, die Liste erhaelt man mit den Methoden
\wancicode{GetX()}, bzw. \wancicode{GetY()} zur\"uck:

\begin{small}
\begin{quote}
\begin{verbatim}

d = data.Data()
d.Init()     
d.SetX(x)                          
d.SetY(y1)                         

\end{verbatim}
\end{quote}
\end{small}

Initialisiere verschiedene Farbobjekte, wieder erh\"alt man mit
\wancicode{GetValue()} die Werte, in diesem Fall im RGB-Format,
zur\"uck:

\begin{small}
\begin{quote}
\begin{verbatim}

bgc = colors.Color()               
bgc.Init()
bgc.SetValue((200,200,200))         
                                    
tc = colors.Color()                
tc.Init()
tc.SetValue((200,200,200))

lc = colors.Color()
lc.Init()
lc.SetValue((200,200,200))

\end{verbatim}
\end{quote}
\end{small}

Initialisiere ein Farblistenobjekt und erzeuge Abschattierungen.  Das
Objekt erhaelt mit \wancicode{SetMediumshade()} und
\wancicode{SetDarkshade()} Schattierungswerte (hier 50 und 100) aus
denen es mit \wancicode{SetMedium()} und \wancicode{SetDark()} Listen
mittlerer und starker Schattierung erzeugt.
  
\begin{small}
\begin{quote}
\begin{verbatim}

cl = colors.Colorlist()      
cl.Init(bglist)
cl.SetMediumshade(50)              
cl.SetDarkshade(100)               
cl.SetMedium()                     
cl.SetDark()

\end{verbatim}
\end{quote}
\end{small}

Initialisiere Fontobjekte: mit \wancicode{SetDir()} (default:
\wancicode{/fonts/}) wird das Verzeichnis der Fontfiles angegeben, mit
\wancicode{SetPath()} der gesamte Pfad samt Dateiname,
\wancicode{SetName()} setzt den Dateinamen alleine.
\wancicode{Name2Path()} setzt das Verzeichnis und den Dateinamen zu
einem kompletten Pfad zusammen.

\begin{small}
\begin{quote}
\begin{verbatim}

tf = font.Font()          
tf.SetDir()               
tf.SetName('helvB12.pil') 
tf.Name2Path()            
                          
lf = font.Font()          
lf.SetDir()
lf.SetName('helvB08.pil')
lf.Name2Path()

\end{verbatim}
\end{quote}
\end{small}

Inititialisiere ein SDV-Objekt.  \wancicode{Init()} uebergibt alle
gewuenschten Einstellungen, die auch mit den Set-Methoden einzeln
gegeben werden koennen.  \wancicode{SetUp()} veranlasst SDV zur
Ermittlung interner Parameter \wancicode{TopTitle()} erzeugt ein
Titelfeld (alternativ: \wancicode{BottomTitle()})
\wancicode{SetWallpaper()} setzt eine geeignete Datei als
Hintergrundbild.  \wancicode{BottomLegend()} erzeugt eine Legende am
unteren Rand (alternativ: \wancicode{LeftLegend()},
\wancicode{RightLegend()} \wancicode{Graphic()} plaziert eine Graphik
im verbleibenden freien Platz

\begin{small}
\begin{quote}
\begin{verbatim}

tf = font.Font()          
tf.SetDir()               
tf.SetName('helvB12.pil') 
tf.Name2Path()            

sdv = sdv.SDV()                    
sdv.Init()                         
sdv.SetSize((640,480))             
sdv.Setup()                        
sdv.SetColor(bgc.GetValue())       
sdv.TopTitle(tf.GetPath())
sdv.SetWallpaper('hannibal1.jpg')
sdv.BottomLegend(d.GetX(), d.GetY(0), lf.GetPath())  
sdv.Graphic()                                        
                                                     
\end{verbatim}
\end{quote}
\end{small}

Initialisiere ein Titel-Objekt: \wancicode{GetTitlesize()} \"ubergibt
die Gr\"o"se des Titelfeldes, Fontpfad und Farbwerte werden ebenfalls
mit \wancicode{Get}-Methoden ermittelt.  \wancicode{Draw()} veranlasst
das Objekt zum Erzeugen des eigentlichen Bildes.

\begin{small}
\begin{quote}
\begin{verbatim}

t = title.Title()
t.Init()
t.SetSize(sdv.GetTitlesize())
t.SetText('Schaeden durch Kannibalismus an Koerperteilen (in Prozent)')
t.SetFont(tf.GetPath())
t.SetColor(tc.GetValue())
t.Draw()

\end{verbatim}
\end{quote}
\end{small}

Initialisiere ein Legenden-Objekt, parallel zum Titelobjekt:

\begin{small}
\begin{quote}
\begin{verbatim}

#l = legend.Legend()
#l.Init(d.GetX(), d.GetY(0), size = sdv.GetLegendsize())
#l.SetFont(lf.GetPath())
#l.SetColor(lc.GetValue())
#l.SetColorlist(cl.GetMedium())
#l.ListDraw()

\end{verbatim}
\end{quote}
\end{small}


Initialisiere ein Graphik-Objekt wieder unter Zuhilfenahme von
\wancicode{Get}-Methoden \wancicode{SetDevc()} erm\"oglicht die
Einstellung des Koeffizienten der perspektivischen Verschiebung,
\wancicode{SetDev()} erzeugt die Verschiebung.  \wancicode{SetUp()}
erzeugt interne Einstellungen.  \wancicode{ColumnsMeasures()} erzeugt
weitere Einstellungen speziell f\"ur S\"aulendiagramme (alternativ:
\wancicode{GraphMeasures()} f\"ur
Graphen). \\
In diesem Beispiel wird ein 3-dimensionales S\"aulendiagramm
(\wancicode{CubicColumns()} mit Hintergrundlinien
\wancicode{BackgroundLining()} in einem Standardkoordiantensystem
f\"ur S\"aulendiagramme (\wancicode{StandardColumns()} erzeugt. Man
beachte, da"s zuerst die \wancicode{Draw()}-Anweisung erfolgen mu"s.

\begin{small}
\begin{quote}
\begin{verbatim}

g = graphic.Graphic()
g.Init(d.GetX(), d.GetY(0), sdv.GetGraphicsize())
g.SetColor((200,200,200))
g.SetLight(cl.GetLight())
g.SetMedium(cl.GetMedium())
g.SetDark(cl.GetDark())
g.SetDevc((0.5,0.5))
g.SetDev()
g.SetUp()
g.SetYname('Schaeden')
g.SetXname('Schadenort')
g.ColumnMeasures()
g.Draw()
g.BackgroundLining()
g.CubicColumns()
g.StandardColumns()

\end{verbatim}
\end{quote}
\end{small}

Fuege die Teile zusammen und unterlege einen Hintergrund Mit der
\wancicode{SDV}-Methode \wancicode{Paste()} werden die Bildelemente in
das Grundbild eingef\"ugt, mit \wancicode{AttachWallpaper()} kann ggf.
das Hintergrundbild unterlegt werden.  Mit \wancicode{Show()} wird das
Bild angezeigt, oder auch mit \wancicode{Save()} gespeichert.

\begin{small}
\begin{quote}
\begin{verbatim}

sdv.Draw()

sdv.Paste(t.GetPicture(), sdv.GetTitlelocation())
sdv.Paste(l.GetPicture(), sdv.GetLegendlocation())
sdv.Paste(g.GetPicture(), sdv.GetGraphiclocation())
sdv.AttachWallpaper(0.4)
sdv.Show()

\end{verbatim}
\end{quote}
\end{small}
%%  SDV UserDoc ENDE %%
