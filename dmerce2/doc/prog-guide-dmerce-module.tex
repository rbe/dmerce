%%
\chapter{dmerce Module}

\newpage

\section{Definition und Aufbau}

Ein dmerce Modul ist derzeit ein in der Programmiersprache Python
geschriebenes Paket, dass Klassen und Methoden enth\"ahlt.

Alle dmerce Module stammen von der Klasse \wancicode{Class} aus dem
Paket \wancicode{DMS.MBase} ab. \"Uber sogenannte \wancicode{Keywords}
wird der Modulklasse eine Reihe an Instanzen mitgegeben, die alle
Informationen \"uber das dmerce System zur Laufzeit eines Templates
enthalten.  Sie werden in der Tochterklasse (Modulklasse) dann als
Klassenvariablen zug\"anglich gemacht.

\subsection{Datenstrukturen und Instanzen}
\label{dmerceModulDatenstrukturenInstanzen}

\begin{itemize}
\item \wancicode{self.\_debug} Ist Debugging eingeschaltet?
  (Integer; 0=nein oder 1=ja)
\item \wancicode{self.\_log} Instanz von Log, loggt in aktuelles Logfile %(\wancisee{Log})
\item \wancicode{self.\_cgi} Instanz von DMS.HTTP.CGI, gew\"ahrt
  Zugriff auf private dmerce Variablen \wancicode{(var)} sowie
  Formular-Variablen \wancicode{(form)}
\item \wancicode{self.\_sqlSys} Instanz einer Datenbankverbindung zur
  Systemdatenbank
\item \wancicode{self.\_sqlData} Instanz einer Datenbankverbindung zur
  Datenbank, die die Daten h\"alt
\item \wancicode{self.\_prjCfg} Konfigurationsdaten aus der
  Systemdatenbank; Dictionary
\item \wancicode{self.\_sam} Instanz von Secure Area Management
  \wancisee{SAM}
\item \wancicode{self.\_uxs} Instanz von Unified Access System
  \wancisee{UXS}
\item \wancicode{self.\_uxsi} Unified Access System Identification
  \wancisee{UXSI}; String (UserRechte.Gruppen)
\end{itemize}

\paragraph{Allgemein}

Jede Klasse nimmt per Konstruktor die Keywords \wancicode{kw} auf und
leitet sie an den Konstruktor der Mutterklasse
\wancicode{DMS.MBase.Class} weiter. Alle Methoden, die als Funktion
oder als Trigger aufgerufen werden, m\"ussen einen positiven
Return-Wert liefern. Wird dieser nicht geliefert, so gilt die Funktion
oder der Trigger als gescheitert. dmerce verh\"alt sich entsprechend
und gibt eine Fehlermeldung. Die Return-Werte werden bei einer
Funktion dort angezeigt, wo sich das Makro des Templaters im
HTML-Quelltext befindet. dmerce wendet auf jeden Return-Wert vor der
Darstellung/Ausgabe im Template die Funktion \wancicode{str()} an, um
eine korrekte Darstellung (beim ersetzen der regul\"aren Ausdr\"ucke)
zu gew\"ahrleisten.

(beschreibung paket, modul, funktion)

\paragraph{Funktionen}

Bei Funktionen k\"onnen der entsprechenden Methode beliebige Parameter
definiert werden, die dann per \wancicode{(exec)}-Makro \"ubergeben
werden k\"onnen. Ihr stehen alle dmerce Datenstrukturen
\wancisee{dmerceModulDatenstrukturenInstanzen} zur Verf\"ugung.Es
empfiehlt sich f\"ur einen reibungslosen Ablauf, dass alle Parameter
mit vordefinierten Werten eingestellt sind. Somit vermeidet man zu
viele Fehlaufrufe, die vielleicht keine sein sollen.  Eine Funktion
kann besser per Return-Wert die Nachricht zur\"uckgeben, dass nicht
alle Parameter korrekt waren, als dass eine h\"assliche Fehlermeldung
angezeigt wird. Desweiteren empfiehlt sich ein Standard-R\"uckgabewert
f\"ur den Fall, dass Parameter nicht (korrekt) \"ubergeben wurden.

\paragraph{Trigger}

Ein Trigger hat keine direkten Parameter wie eine Funktion. Ihm stehen
alle dmerce Datenstrukturen
\wancisee{dmerceModulDatenstrukturenInstanzen} zur Verf\"ugung.

\section{Beispiele}

\subsection{Datei: meinModul.py}

Das Paket \wancicode{meinModul} enth\"alt folgenden Python-Code:

\begin{verbatim}
import DMS.MBase

class Test(DMS.MBase.Class):

    def __init__(self, kw):
        DMS.MBase.Class.__init__(self, kw)

    def FunktionEins(self, was):
        return 'Hier ist etwas:', str(was)

    def FunktionZwei(self, was):
        return 'Hier ist etwas mal 5 multipliziert:',
               was * 5

    def TriggerEins(self):

    def TriggerZwei(self):

\end{verbatim}

\paragraph{Aufrufe als \wancicode{(exec)}-Funktion} k\"onnen z. B. dann so
aus sehen:

\begin{itemize}
\item \wancicode{(exec meinModul.Test.FunktionEins was='boah, ey')}
  wird im Template zu Hier ist etwas: boah, ey
\item \wancicode{(exec meinModul.Test.FunktionEins was="und noch
    einer\")} wird im Template zu Hier ist etwas: und noch einer
\item \wancicode{(exec meinModul.Test.FunktionEins was=0190123456)}
  wird im Template zu Hier ist etwas: 0190123456
\end{itemize}

\begin{itemize}
\item \wancicode{(exec meinModul.Test.FunktionEins was='boah, ey')}
  wird im Template zu Hier ist etwas mal 5 multipliziert: boah,
  eyboah, eyboah, eyboah, eyboah, ey
\item \wancicode{(exec meinModul.Test.FunktionEins was=1.5)} wird im
  Template zu Hier ist etwas mal 5 multipliziert: 7.5
\item \wancicode{(exec meinModul.Test.FunktionEins was=5)} wird im
  Template zu Hier ist etwas mal 5 multipliziert: 25
\end{itemize}
