\chapter{1{[}NCC{]} - The API}

\newpage

\wanciprgpackage{user-account.py}{ Das Skript dient zu Erstellung der
  passwd und der group Datei von Unix/Linux Systemen.  Dabei werden
  die Systemeinstellungen nicht ber\"uhrt und beibehalten, d.h. alle
  User und Gruppen mit der UID/GID unter 1000 werden so gelassen, wie
  sie bei der installation des Systems angelegt wurden. Werden per NCC
  User und/oder Gruppen angelegt, so werde ID's nur oberhalb von 1000
  angelegt um keinen Konflikt hervozurufen. Wie bei den meisten
  Skripten des NCC's wird als erstes die MAC Adresse und die ServerID
  ermittelt. Dann wird \"uberpr\"uft, f\"ur welche Dienste der
  aktuelle Server zust\"andug ist.  Sind dann die Dienste ermittelt,
  so werden alle User und Gruppen angelegt, die f\"ur diesen Dienst
  eingetragen sind.  Sinn und Zweck dieser Pr\"ufung ist es, nur die
  User anzulegen, die die Berechtigung haben, diesen Dienst zu
  managen.  Je nach System werden die entsprechenden Dateien
  angelegt.}{}

\wancicliinv{create-default.py}{keine}{Legt eine Tabelle fuer eine
  dmerce-Datenbank an mit Standard Feldern (ID, CreatedDateTime,
  CreatedBy, ChangedDateTime, ChangedBy, active)}

\wanciprgpackage{http.py}{Generiert aus der Datenbank dmerc\_ncc eine komplette vhost.conf.}{}

\wanciprgclass{DBData}{keine}{d = DBData()}{}

\wanciprgmethodbegin{GetVhost}{Liefert eine Liste mit den allen
  konfigurierten VirtualHost aus den Tabellen DnsZones, DnsRecords,
  DnsRrtypes, SrvWebsrvVhosts, SrvWebsrvAlt.}
\wanciprgmethodend

\wanciprgmethodbegin{GetSrvAlias}{Holt die Alias Liste zu den
  entsprechenden VirtualHost.  -$>$ Uebergabe ist die ID}
\wanciprgmethodend

\wanciprgmethodbegin{GetIP}{Sucht per Rekursion die passende IP
  Adresse zum hostname und setzt setzt die innerhalb der Klasse.  -$>$
  Uebergabe zone, name}
\wanciprgmethodend

\wanciprgmethodbegin{ReturnIP}{Gibt den Wert der IP Adresse zurueck,
  die durch die Methode GetIP gesetzt wurde.}
\wanciprgmethodend

\wanciprgclass{Vhost}{keine}{v = Vhost()}{}

\wanciprgmethodbegin{SetNameVirtHost}{Der NameVirtualHost wird
  gesetzt.  -$>$ Uebergabe der IP}
\wanciprgmethodend

\wanciprgmethodbegin{SetListen}{Die Listen Direktive wird gesetzt.
  -$>$ Uebergabe IP}
\wanciprgmethodend

\wanciprgmethodbegin{SetVhost}{Der VirtualHost wird gesetzt.  -$>$
  Uebergabe ip}
\wanciprgmethodend

\wanciprgmethodbegin{SetDocRoot}{Das Document Root Verzeichnis wird
  gesetzt.  -$>$ Uebergabe des Webserververzeichnisses,
  Projektverzeichnis}
\wanciprgmethodend

\wanciprgmethodbegin{SetAliases}{Die ServerAliases werden gesetzt.
  -$>$ Uebergabe einer List der Aliases}
\wanciprgmethodend

\wanciprgmethodbegin{SetSrvName}{Der ServerName wird gesetzt.  -$>$
  Uebergabe vom ServerName (string)}
\wanciprgmethodend

\wanciprgmethodbegin{SetServerAdmin}{Der ServerAdmin wird gesetzt.
  -$>$ Uebergabe der ServerAdmin (string)}
\wanciprgmethodend

\wanciprgmethodbegin{SetAccLog}{Der AccessLog Dateiname wird gesetzt.
  -$>$ Uebergabe Dateiname (string)}
\wanciprgmethodend

\wanciprgmethodbegin{SetAccLogType}{Der AccessLog Typ wird gesetzt
  -$>$ Uebergabe des Typs (string)}
\wanciprgmethodend

\wanciprgmethodbegin{SetErrLog}{Der ErrorLog Dateiname wird gesetzt.
  -$>$ Uebergabe Dateinname}
\wanciprgmethodend

\wanciprgmethodbegin{SetFoot}{Die Fusszeile wird gesetzt.}
\wanciprgmethodend

\wanciprgmethodbegin{SetAddCfg}{Die zustzliche Konfiguration wird
  gesetzt.  -$>$ Uebergabe (string)}
\wanciprgmethodend

\wanciprgclass{File}{keine}{f = File()}{}

\wanciprgmethodbegin{AddLine}{Fuegt der Datei eine Zeile hinzu.  -$>$
  Uebergabe der Zeile in Form eines Strings}
\wanciprgmethodend

\wanciprgmethodbegin{Write}{Schreibt den Inhalt in eine Datei Pfad und
  Dateiname sind z.Zt. noch fest eingestellt.}
\wanciprgmethodend

\wanciprgclass{Function}{keine}{f = Function()}{}

\wanciprgmethodbegin{ChangeElement}{Ersetzt ein Element einer List an
  der angegebenen Stelle durch einen anderen Wert.  -$>$ Uebergabe der
  Liste, des neuen Wertes, der Stelle}
\wanciprgmethodend

\wanciprgmethodbegin{CreateList}{Erzeugt aus einer Liste mit Tupeln
  einen durch den angegebenen Seperator getrennten String, bestehend
  aus dem Element an der angegebenen Stelle -$>$ Uebergabe list,
  Seperator, Element Nummer}
\wanciprgmethodend

\subsection{Ablauf}

Im ersten Schritt wird die dmerce.cfg ausgelesen und die Verbindung
zur dmerce\_sys Datenbank hergestellt. Ist dann die Verbindung zur
Projektdatenbank erfolgreich, wird die Verbindung zur Projektdatenbank
hergestellt. Anhander der durch SnmpMac.GetServer ermittelten ServerID
und nach der \"Uberpr\"ufung, ob der Server den Dienst als Webserver
besitzt, wird das Hauptprogramm angestossen.

Im Hauptprogramm werden als erstes die n\"otigen Instanziierungen
vorgenommen.  Danach werde alle f\"ur den Server eingetragenen vhost
Konfigurationen aus der Datenbank ausgelesen. Sollte es sich bei dem
Vhosteintrag nicht um eine IP Adresse handeln, so wird durch eine
Funktion diese anhand der DNS Eintr\"age ermittelt.

Die Daten werden dann in der Klasse Vhost zu einem g\"ultigen
Datensatz zusammengesetzt.

\begin{itemize}
\item \wancicode{Vhost.SetNameVirtHost(ip)} Setzt die NameVirtualHost
  Einstellung
\item \wancicode{Vhost.SetListen(ip)} Setzt die Listen Einstellung
\item \wancicode{Vhost.SetVhost(ip)} Setzt die $<$VirtualHost
  $<$ip$>$$>$ Einstellung
\item \wancicode{Vhost.SetDocRoot(result{[}i{]}{[}'DocumentRoot'{]})}
  Setzt die DocumentRoot Einstellung
\item \wancicode{Vhost.SetAliases(aliases)} F\"ugt, wenn welche
  vorhanden, ServerAlias hinzu
\item \wancicode{Vhost.SetSrvName(result{[}i{]}{[}'RecordName'{]} +
    '.' + result{[}i{]}{[}'ZoneName'{]})} Erzeugt den ServerNamen
\item
  \wancicode{Vhost.SetServerAdmin(result{[}i{]}{[}'ServerAdmin'{]})}
  Tr\"agt den ServerAdmin ein
\item \wancicode{Vhost.SetAccLog('access\_log')} definiert den
  access\_log Namen
\item \wancicode{Vhost.SetAccLogType(result{[}i{]}{[}'AccLogType'{]})}
  Legt den access\_log Typen fest
\item \wancicode{Vhost.SetErrLog('error\_log')} Definiert den
  error\_log Name
\item \wancicode{Vhost.SetFoot()} Schliesst die Konfigurationsdatei
\item
  \wancicode{Vhost.SetAddCfg(result{[}i{]}{[}'AdditionalConfigs'{]})}
  F\"ugt manuelle Konfigurationselemente hinzu
\end{itemize}

Sind alle Daten generiert und der Klasse File(), die als Container
f\"ur die gesamte Konfigurationsdatei dient, hinzugef\"ugt, so wird
die Konfigurationsdatei im der dem Hauptprogramm \"ubergebenen Pfad
mit dem Dateinamen vhost.conf geschrieben.

\section{mail\_alias.py}

Modul zu Erstellung der sendmail alias datei.

DBData
        GetAliases
        ----------
        Liefert alle Aliases in Form einer Liste.
        Die Aliase-List wird mit Alias, Zone, AccUser gespeichert.

        GetServerInfo
        -------------
        Liest das Feld "SQL" zu einem FQHN aus und gibt den Wert zurueck.
        -$>$ Uebergabe FQHN


GenerateAliasLine

        SetAlias
        --------
        Der Aliasname wird gesetzt
        -$>$ Uebergabe AliasName
        
        SetZone
        -------
        Die Zone wird festgelegt
        -$>$ Uebergabe ZoneName

        SetLogin
        --------
        Der AccountName wird festgelegt
        -$>$ Uebergabe Login

GenerateAliasFile

        AddLine
        -------
        Fuegt dem Zeilen Container eine Zeile hinzu.
        -$>$ Uebergabe ist ein String

        Write
        -----
        Erzeugt die Datei virtusertable.txt in /etc/mail

        GenVirtDB
        ---------
        Generiert aus der erzeugten virtusertable.txt eine
        hash-Datenbank per Systemkommando

\section{mail\_conf.py}

Modul zu Erstellung der sendmail.cf Konfigurationsdatei

Shell Aufruf:

mail\_conf.py --serverID=$<$ID des Servers$>$
             --mcFile=$<$Konfigurations-Text-Datei$>$
             --cfFile=$<$sendmail cf Datei$>$

System

        GetOS
        -----
        Gibt den OperationSystem Namen des Systems wieder.

        SetcfDir
        --------
        Mit der Methode GetOS wird das System ermittelt und anhand
        dessen das Verzeichnis fuer die Sendmailkonfigurationsdatei
        gesetzt.
        
DBData

        GetMacros
        ---------
        Liefert eine Liste mit den fuer die ServerID eingetragenen
        Macros fuer sendmail
        -$>$ Uebergabe Macroname, ServerID

BuildConfFile

        SetMacro
        --------
        Macroname wird gesetzt.
        -$>$ Uebergabe Macroname

        SetValue
        --------
        Der Inhalt des Macros wird gesetzt.
        -$>$ Uebergabe Macroinhalt

        \_\_str\_\_
        -------
        Rueckgabe des kompletten Strings

SetConfFile

        AddLine
        -------
        Zeile wird dem File-Container hinzugefuegt.
        -$>$ Uebergabe Zeile(string)

        Write
        -----
        Schreibt die erstellte Konfigurstionsdatei.
        -$>$Uebergabe des Dateinamens   

        ConvertConfFile
        ---------------
        Auf dem System wird der m4 Befehl abgesetzt, um die
        erstellte Konfigurationsdatei zu konvertieren.
        -$>$ Uebergabe directory, mcFile, cfFile (Verzeichnis, erstellte Datei,
           konvertierte Datei)

\section{dns.py}

Erzeugt die Konfiguration f\"ur den Nameserver. Die
Verzeichnisstruktur ist in eine Struktur gegliedert. So wird f\"ur
jeden Agent ein Verzeichnis angelegt und darunter die Verzeichnisse
seiner Nameserver. In den Nameserververzeichnissen sind dann die
Zone-Files und named.conf vorhanden.

Ben\"otigte Module:

\begin{itemize}
\item\wancicode{DMS.SQL} Datenbankverbindungen
\item\wancicode{DMS.SysTbl} Analyse des Sys-Strings in der dmerce.cfg
\item\wancicode{Guardian.Config} ConfigParser
\item\wancicode{DNSArpa} Revers Mapping Modul, h\"alt die Eintr\"age f\"ur das ReverseMapping
\item\wancicode{DNSNS} Datenkontainer f\"ur Nameserver Informationen
\item\wancicode{DNSZone.ZoneData} ZoneInformationen Werden in diesem Modul gehalten
\item\wancicode{DMS.SnmpMac} Die MAC-Adresse und die ServerID f\"ur den Server werden ausgelesen
\end{itemize}

DBData
        
        GetNSList
        ---------
        Liefert alle Nameserver, die in einer zone angegeben sind.
        -$>$ \"Ubergabe der zone (string)

        GetZones
        --------
        Liefert Daten \"uber Zones, die zu einem speziellen Agent geh\"oren.
        Dies sind u.a. SOA\_Expire TTL Zone-C etc.
        -$>$ \"Ubergabe AgentID, ServerID

        GetAgents
        ---------
        Liefert eine Liste der gesamten Agents

        GetForwarders
        -------------
        Liefert alle Forwarder IP f\"ur den angegebenen Nameserver
        -$>$ \"Ubergabe Nameserver ID

        GetReverse
        ----------
        Holt alle ReverseMappings aus der Datenbank zum entsprechenden
        Nameserver
        -$>$ \"Ubergabe Nameserver ID

\subsection{Ablauf}

Die dmerce.cfg wird ausgelesen und analysiert. Dadurch wird die
Verbindung zur dmerce\_sys Datenbank hergestellt und die Eintr\"age
f\"ur das Projekt (z. B. hellfire.ms0.1ci.net) ausgelesen. Mit dem
f\"ur das Projekt g\"ultigen SQL-String wird die dmerce\_ncc Datenbank
konnektiert. Anhand der ServerID wird festgestellt, ob der aktuelle
Server f\"ur den DNS Dienst konfiguriert ist. Wenn dem so ist wird die
NameserverID f\"ur den Server ermittelt. Mit der NameserverID wird das
Hauptprogramm angestossen, das anhand der NS\_ID und einem
\"ubergebenen Pfad die f\"ur den Server spezifischen Daten aus der
Datenbank holt und in Konfigurationsdateien schreibt.

\subsection{Hauptprogramm}

Als erstes werden Instanzen allgemein ben\"otigter Klassen und Module
erzeugt.

\begin{itemize}
\item \wancicode{db = DBData()} beinhaltet Methoden mit DB-Abfragen
\item \wancicode{func = Function()} Klasse mit allgemein ben\"otigten Funktionen
\item \wancicode{DNSArpa.ReverseMap()} Modul das das Handling mit
  ReverseMapping erm\"oglicht
\end{itemize}

Anhand der NameserverID werden folgende Dinge aus der Datenbank im
ersten Schritt ausgelesen.  DBData.GetNsData($<$db-instance$>$,
$<$ns-server-id$>$) liefert folgende Daten:

\begin{itemize}
\item DnsNs.ID AS NS\_ID
\item DnsNs.Name AS NS\_Name
\item DnsNs.IP AS NS\_IP
\item Agent.ID AS AgentID
\item Agent.Name AS AgentName
\item Agent.Vorname AS AgentVorname
\end{itemize}

Zus\"atzlich werden Daten bez\"uglich der SlaveReverseMapping ermittelt.
Man erh\"alt durch DBData.GetSlaveReverse($<$db-instance$>$, $<$ns-server-id$>$).

\begin{itemize}
\item DnsReverse.arpa
\item DnsNs.IP AS MasterIP
\end{itemize}
