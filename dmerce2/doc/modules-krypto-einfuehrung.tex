\subsection{Kryptographie? Paranoia?}
\label{KryptographieParanoia}

\begin{quote}
  Es gibt eigentlich nur zwei Arten von Kryptographie: die eine h"alt
  Ihre kleine Schwester davon ab, Ihre Daten zu lesen, w"ahrend die
  andere selbst einflu"sreichen Regierungen den Zugang zu Ihren Daten
  verwehrt. [...]
  
  Wenn ich von Ihnen verlange, einen Brief zu lesen, den ich in einen
  Safe gelegt und dann irgendwo in New York versteckt habe,
  \textit{dann hat das nichts mit Sicherheit zu tun, sondern ist ein
    einfaches Verstecken.} Wenn ich diesen Brief jedoch in den Safe
  lege, Ihnen dann die Entwicklungspl"ane des Safes und noch hundert
  Safes gleicher Bauart mitsamt ihren Kombinationen gebe, so da"s Sie
  mit Hilfe der weltbesten Safeknacker den Sperrmechanismus ausgiebig
  studieren k"onnen, aber immer noch nicht in der Lage sind, den Safe
  zu "offnen und den Brief zu lesen - \textit{dann ist das
    Sicherheit.} \\
  \cite{schneier96}, S. xvii
\end{quote}

Treffender kann man kaum formulieren, worum es bei Kryptographie geht:
Sicherheit zu gew"ahrleisten, obwohl die beteiligten Mechanismen
offenliegen und f"ur au"senstehende nachzuvollziehen sind; das einzig
geheime ist der Schl"ussel und - das ist die Zielsetzung - die
verschl"usselte Information.

Seit Anfang der neunziger Jahre stellt das Internet seine unbegrenzten
M\"oglichkeiten zur Verf\"ugung. Dies erm\"oglicht jedermann, in
bisher unbekanntem Umfang, sich zu informieren, bzw. Information
auszutauschen. Das zun\"achst explosionsartige Wachstum der sog.
\textit{New Econonmy} beweist das Potential der weltumspannenden
Datenkommunikation f\"ur die Gesch\"aftswelt.

Aber die gro"se Hebelwirkung dieser technischen Entwicklung fordert
auch Mi"sbrauch heraus.  Neben den in den Medien vielbeachteten
Aktionen von \textit{Hackern} oder \textit{Crackern}, die h\"aufig
wenig kommerziellen Schaden (au"ser Leistungserschleichungen)
anrichten und sogar Anla"s bieten, Sicherheitsl\"ucken zu entdecken
und zu schlie"sen, haben sich Betrug und Wirtschaftsspionage im
Bereich der elektronischen Kommunikation in beunruhigendem Ma"se
entwickelt.

Auf dem Gebiet der Wirtschaftsspionage sind nach dem Ende des Kalten
Krieges auch die Geheimdienste (v.a. des westlichen Auslands) aktiv
geworden. Diese Beh\"orden stellen durch ihre jahrzehntelange
Erfahrung mit Geheimhaltungsverfahren eine mehr als ernstzunehmende
Gr\"o"se im Bereich der gesch\"aftliche Datenkommunikation dar.

Solche Entwicklungen werden dadurch beg\"unstigt, da"s das Internet in
den ersten Jahrzehnten seines Bestehens kaum sensible Daten zu
\"ubertragen hatte (die Milit\"ars hatten eigene Kan\"ale !). So kam
es, da"s bei den damals geschaffenen und z.T. noch heute in Gebrauch
befindlichen Protokollen wie \texttt{TCP/IP} keine Ma"snahmen in
Fragen der Sicherheit ergiffen wurden.  Daher wurden im Verlauf der
Jahre zunehmend Anstrengungen unternommen, Sicherheitsfunktionen zu
entwickeln. Der Proze"s, diese Funktionen in die vorhandene Familie
von Protokollen zu integrieren, bzw. bei neuen Protokollen (z.B.
\texttt{IPv6}) zu ber\"ucksichtigen, dauert noch an.

\subsection{Was k\"onnen Sie tun?}

\begin{quote}
  \textit{Wenn jemand paranoid ist, bedeutet das nicht, da"s er keine Verfolger hat.} \\
  \textsc{Amos Oz}
\end{quote}

\begin{enumerate}
  
\item Sowohl das Inter- als auch die verschiedenen Intranets sind
  alles andere als abh\"orsicher. Gehen Sie im Zweifelsfalle immer
  davon aus, da"s jemand Ihre Kommunikation verfolgen kann.
  
\item Kryptographie hat, wie eingangs schon erw\"ahnt, nichts mit
  Verstecken zu tun.  Vertrauen Sie Verfahren, die von
  Kryptographie-Experten eingehend gepr\"uft wurden.  Wenn sich die
  Schw\"achen des Verfahrens in Grenzen halten, ist es besser, diese
  Schw\"achen zu kennen, als ein geheimgehaltenes Verfahren zu
  benutzen.
  
\item Gehen Sie davon aus, da"s ein Geheimdienst, der ein Verfahren
  entwickelt hat und es freiwillig ver\"offentlicht, dieses auch
  brechen kann.
  
\item Seien Sie paranoid. \textit{It's better to be safe than sorry.}

\end{enumerate}

\subsection{Was tut die 1Ci?}

Seit Ende des Jahres 2000 implementiert die 1Ci Verfahren zur
Verschl\"usselung von Daten zur Erg\"anzung von \texttt{dmerce}.  Zur
Zeit sind bereits folgende Verschl\"usselungsalgorithmen
einsatzf\"ahig:

\begin{itemize}
\item DES
\item Triple-DES
\item IDEA
\item AES
\item RSA
\end{itemize}

Im weiteren Verlauf werden weitere Verfahren (z.B. \textit{Signatur})
implementiert werden.

Das langfristige Ziel ist es, neben den erforderlichen weiteren Algorithmen, eine Entwicklungsumgebung
f\"ur kryptographische Komplettl\"osungen zur Verf\"ugung zu stellen, so da"s aufwendige Strukturen
wie \texttt{PKI} (\textit{Public Key Infrastructure}) problemlos und flexibel in Zusammenhang mit 
Datenbank- und Netztechnologien erstellt werden k\"onnen. Voraussetzung daf\"ur ist nat\"urlich,
da"s die zust\"andigen Organisationen die entsprechenden Standards verabschieden.
