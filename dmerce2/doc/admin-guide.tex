%%%
\part{Administrators Guide - HOW-TO Operate dmerce}

%%
\chapter{Installation und Konfiguration}
\label{Installation}
\index{Installation dmerce!Voraussetzungen dmerce!Python, MySQL,
  Oracle, PostgreSQL}

\newpage
\section{Voraussetzungen}
\label{Voraussetzungen}


Stand: \date{\today}, dmerce Version \dmercever\\

Da der Einsatz von dmerce ohne Datenbank und Webserver nicht m\"oglich
ist, m\"ussen folgende Mindestkriterien erf\"ullt werden, um eine
akzeptable Antwortzeit von dmerce zu erhalten.

Gerade der Einsatz einer Oracle Datenbank erfordert h\"oheren Einsatz
von Hauptspeicher und h\"oherer Festplattenkapazit\"at.

Setzt man hingegen die nicht so performancelastige Datenbank MySQL
ein, so sinken die Anforderungen an die Hardware. Somit ist es schon
m\"oglich, auf kleinerer Hardware kurze Reaktionszeiten zu erreichen.

Die obigen Angaben beziehen sich auf den Fall, dass Datenbank,
Webserver und die dmerce-Umgebung auf ein und dem selben Server
agieren.  Somit kann die Austattung der Server variieren, wenn die
einzelnen Dienste auf mehrere Server verteilt werden. Nicht ratsam ist
die Einsparung im Bereich der Oracle Datenbank.

Die genauen Voraussetzungen f\"ur Hard- und Software erhalten Sie im
Kapitel \vref{HardwareVoraussetzungen} und
\vref{SoftwareVoraussetzungen}

\subsection{Hardware}
\label{HardwareVoraussetzungen}

Diese Tabellen beschreiben die einzelnen Hardware- und
Software-Vorraussetzungen der Teile, die von dmerce ben\"otigt werden.
Generell bleibt zu sagen, dass dmerce auf jedem UNIX-System
lauff\"ahig ist, dass OpenSource-Tools, Python Version 1.5.2 und eine
der auffgef\"uhrten Datenbanken unterst\"utzt.

Die hier beschriebenen Hardware- und Software-Plattformen sind im
Hinblick auf dmerce als stabil lauff\"ahig bekannt.

\medskip

Hardware:

\medskip

\begin{tabular*}{128mm}{p{30mm} p{40mm} p{50mm}}
\hline
\wancitableheader{Typ}          & \wancitableheader{Hersteller/Produkt}
                                & \wancitableheader{URL}\\
\hline
CISC-basierte Server            & Intel Pentium, empfohlen III ab 500 MHz  
                                & http://www.intel.com\\
                                & AMD empfohlen ab K6 xxx MHz
                                & http://www.amd.com\\
\hline
RISC-basierte Server            & Sun Microsystems, Inc. ab UltraSPARC-II 360 MHz mit Sun Solaris
                                & http://www.sun.com\\
                                & IBM Corp. Power-Prozessor mit AIX
                                & http://www.ibm.com\\
                                & Hewlett-Packard PA mit HP-UX
                                & http://www.hp.com\\
\hline
Generell                        & Fesplattenplatz ab 50 MByte
                                & \\
\hline
\end{tabular*}

\medskip Die Festplattenkapazit\"at h\"angt von der Anzahl der unter
dmerce betriebenen Projekte/Webs ab, da bei dem Einsatz des
Debug-Levels viel in Logfiles geschrieben wird. Deshalb sollte die
Festplattengr\"osse mit Zunahme an Projekten auch erweitert werden.

\medskip

\subsubsection{Sotware}
\label{SoftwareVoraussetzungen}

Software:

\medskip

\begin{tabular*}{128mm}{p{20mm} p{25mm} p{20mm} p{55mm}}
\hline
\wancitableheader{Typ} & \wancitableheader{Software} & \wancitableheader{Version} & \wancitableheader{URL}\\
\hline
Betriebs-system         & Sun Solaris      & 8                    & http://www.sun.com\\
                        & Linux            &                      & \\
                        & Debian GNU/Linux & potato/2.2           & http://www.sun.com\\
                        & FreeBSD          & ab 4.2               & http://www.freebsd.org\\
\hline
Interpreter             & Python           & 1.5.2                & http://www.python.org\\
\hline
Datenbank               & MySQL            & 3.22.x, 3.23.x       & http://www.mysql.com\\
                        & PostgreSQL       & 7.0.2                & http://www.postgresql.org\\
                        & Oracle           & 8i 8.1.7             & http://www.oracle.com\\
\hline
Module f\"ur Datenbank  & MySQLdb          & 0.1.1                & http://\\
                        & PyGreSQL         &                      & http://\\
                        & DCOracle2        &                      & http://\\
\hline
Webserver               & Apache           & 1.3.12               & http://httpd.apache.org\\
\hline
Webserver-Plugin        & PyApache         & 4.19                 & http://\\
\hline
\end{tabular*}

\medskip

Sie k\"onnen dmerce als Paket von 1Ci bekommen, dass per
Installationsskript \wancicode{setup.sh} die von Ihnen geforderten
Komponenten separat unter dem Betriebssystem-Benutzer
\wancicode{dmerce} vollst\"andig installiert.
\wancisee{Gettingdmerce}, \wancisee{InstallationKomponenten}

\subsection{Der User \wancicode{dmerce}}
\label{DerUserdmerce}

Sofern Sie dmerce manuell - also nicht per Installationsskript -
installiert haben, sollte der User, unter dem dmerce l\"auft, folgende
Einstellungen im \wancicode{Environment} haben.

\begin{itemize}
\item \wancicode{export DMERCE\_VER=\dmercever}
\item \wancicode{export
    DMERCE\_ROOT=/usr/local/1Ci/dmerce/\$\{DMERCE\_VER\}}
\end{itemize}

Der Befehl \wancicode{env} liefert folgendes Ergebnis:

\wancicodeblock{dmerce@hosty:\~ \$ env\\
  DMERCE\_VER=\dmercever\\
  DMERCE\_ROOT=/usr/local/1Ci/dmerce/\dmercever}

\section{Getting dmerce}
\label{Gettingdmerce}

Sie k\"onnen dmerce auf verschiendene Arten erhalten. Die
Voraussetzungen hierf\"ur finden Sie im Kapitel \wancisee{DerUserdmerce}.

\subsubsection{Download}

Gehen Sie zur URL http://dmerce.1ci.de/pDownload

\subsubsection{Concurrent Versions System: CVS}

Developer-Partner von 1Ci k\"onnen per CVS Module und deren Sourcen runterladen
und diese miteinander austauschen.

Das \wancicode{Environment} muss um Einstellungen f\"ur CVS erweitert werden.

\begin{itemize}
\item \wancicode{export CVSROOT=:pserver:anoncvs@cvs.1ci.de:/dmerce}
\end{itemize}

Der Befehl \wancicode{env} liefert folgendes Ergebnis:

\wancicodeblock{dmerce@hosty:\~ \$ env\\
  CVSROOT=:pserver:anoncvs@cvs.1ci.de:/dmerce\\

  dmerce@hosty: \$ cvs login\\
  password: anoncvs\\
  dmerce@hosty: \$ cvs checkout -r \$\{DMERCE\_VER\} dmerce}

\subsubsection{Secure Copy: scp}

Sie k\"onnen sich per SCP die dmerce Binaries auf Ihrem Server laden.
Geben Sie dazu folgendes Kommando in einer Shell ein:

\wancicodeblock{\$ scp -r scp@dmerce.1ci.de:/dmerce/\$\{DMERCE\_VER\} .}

oder auch\\

\wancicodeblock{\$ scp -r scp@dmerce.1ci.de:/dmerce/stable .}

\begin{itemize}
\item \wancicode{scp} Aufruf des Programms \wancicode{scp}
\item \wancicode{-r} Kopiere rekursiv
\item \wancicode{scp@dmerce.1ci.de:/dmerce/stable} Quelle: User scp auf
  entferntem Rechner \wancicode{dmerce.1ci.de}, Verzeichnis
  \wancicode{/dmerce/stable}
\item \wancicode{.} Ziel: lokales Verzeichnis \wancicode{.} = aktuelles
  Verzeichnis
\end{itemize}

\subsection{via CDROM}

Senden Sie eine Postkarte an:

\section{Installation der Komponenten mittels der dmerce-Distribution}
\label{InstallationKomponenten}

\begin{enumerate}
\item Legen Sie einen User \wancicode{dmerce} und eine Gruppe \wancicode{dmerce}
  an, der User \wancicode{dmerce} soll die Standard-Gruppe \wancicode{dmerce}
  haben
\item Entpacken Sie die dmerce-Distribution:\\
  \wancicode{\$ sh dmerce-\dmercever.bin.sh}
\item Starten Sie das Setup:\\
  \wancicode{\$ sh setup.sh}
\end{enumerate}

Wir haben hier die Schritte festgehalten, die zur Installation
der Server notwendig sind. Detailliertere Beschreibungen von 
UNIX-Befehlen erhalten Sie im Kapitel \wancisee{UNIX}.

\subsection{SQL-Datenbankserver}

Wir beschreiben hier die Ausf\"uhrung von Skripten, um:

\begin{itemize}
\item User
\item Tabellen
\item Schemata
\item Rechte
\end{itemize}

im Datenbankserver anzulegen.

\subsubsection{MySQL}

Gehen Sie in das Verzeichnis \$DMERCE\_ROOT/sql/mysql und laden
Sie die erforderliche SQL-Datei.\\

\wancicodeblock{mysql@hosty:\~ \$ cd \$DMERCE\_ROOT/sql/mysql\\
mysql@hosty:\~ \$ cat dmerce\_sys.sql | 
/path/to/mysql/bin/mysql -u dmerce -p\\
password: dmerce}

\subsubsection{PostgreSQL}
\subsubsection{Oracle}

Die Umgebung des Oracle-Users sollte \"ahnliche Werte enthalten:

\wancicodeblock{oracle@hosty:\~ \$ env\\
ORACLE\_HOME=/opt/oracle/app/oracle/product/8.1.7\\
ORACLE\_SID=ora0\\
PATH=/usr/bin:/usr/local/bin:/opt/oracle/app/oracle/product/8.1.7/bin}

Gehen Sie in das Verzeichnis \$DMERCE\_ROOT/sql/oracle und laden
Sie die erforderliche SQL-Datei.

\wancicodeblock{oracle@hosty:\~ \$ cd \$DMERCE\_ROOT/sql/oracle\\
oracle@hosty:\~ \$ sqlplus dmerce\_sys/dmerce\\
SQL$>$ @dmerce\_sys.sql}

\subsection{Die dmerce.cfg}

Die Konfigurationsdatei \wancicode{dmerce.cfg} liegt im
\$\{DOCUMENT\_ROOT\} des Webservers.\\

Ab dmerce Release 2 wird nur noch die Angabe einer Datenbankverbindung
zur Systemdatenbank \wancicode{dmerce\_sys} ben\"otigt, da der Rest
der Informationen in \wancicode{dmerce\_sys} gehalten wird.\\

dmerce Release 2:

\wancicodeblock{{[}q{]}\\
  Sys = $<$database type$>$:$<$database user$>$@$<$database
  host$>$:$<$database name$>$}

dmerce Release 1:

\wancicodeblock{{[}q{]}\\
  s = <SERIAL\_HOST\_NUMBER>\\
  DOCUMENT\_ROOT = /var/www/domain.tld/www/\\
  TEMPLATE\_ROOT = /var/www/domain.tld/www/\\
  \\
  {[}sam{]}\\
  failURL = /sub/passwd\_lost.html\\
  \\
  {[}login{]}\\
  failURL = passwd\_error.html\\
  sendPasswdLoginField = Login\\
  sendPasswdOKURL = passwd\_send.html\\
  sendPasswdErrorURL = passwd\_senderr.html\\
  \\
  \# General email settings\\
  {[}email{]}\\
  from = Muster GmbH $<$info@muster.de$>$\\
  \\
  {[}sql{]}\\
  host = $<$hostname$>$\\
  database = $<$database name$>$\\
  user = $<$database user$>$\\
  passwd = $<$database password$>$\\
  dbtype = $<$database type$>$}

\section{Integration von dmerce in den Webserver}

Hier wird die Verwendung von dmerce mit den unterschiedlichen
Webserver beschrieben. Zur Zeit gilt lediglich der Apache als
zertifizierter Server.

\subsection{Apache}

Generelle Eisntellungen in der \wancicode{httpd.conf}:

\wancicodeblock{\# PyApache\\
  AddHandler python-cgi-script \.pyc\\
  AddHandler application/x-python-httpd-cgi \.pyc\\
  \\
  \# Alle dmerce Konfigurationsdateien sollen von extern nicht einsehbar sein\\
  <Files *.cfg>\\
  deny from all\\
  </Files>\\
  \\
  \# Alle dmerce Templates sollen von extern nicht einsehbar sein\\
  <Directory templates/>\\
  deny from all\\
  </Directory>\\
}

dmerce Release 1:

\wancicodeblock{\# dmerce 1.2.6\\
  <Directory /usr/local/1Ci/dmerce/1.2.6>\\
  Options ExecCGI\\
  </Directory>\\
  \\
  \# server.domain.tld\\
  Listen 123.123.123.123:80\\
  NameVirtualHost 123.123.123.123\\
  <VirtualHost 123.123.123.123>\\
  ServerName server.domain.tld\\
  DocumentRoot /var/www/domain.tld/server\\
  CustomLog /usr/local/apache/logs/server.domain.tld-access\_log combined \\
  ErrorLog /usr/local/apache/logs/server.domain.tld-error\_log\\
  SetEnv PYTHONPATH /usr/local/1Ci/dmerce/1.2.6/:\\
  Alias /bin /usr/local/1Ci/dmerce/1.2.6\\
  </VirtualHost>\\
}

dmerce Release 2:

\wancicodeblock{\# dmerce \dmercever\\
  <Directory /usr/local/1Ci/dmerce/\dmercever>\\
  Options ExecCGI\\
  </Directory>\\
  \\
  \# server.domain.tld\\
  Listen 123.123.123.123:80\\
  NameVirtualHost 123.123.123.123\\
  <VirtualHost 123.123.123.123>\\
  ServerName server.domain.tld\\
  DocumentRoot /var/www/domain.tld/server\\
  CustomLog /usr/local/apache/logs/server.domain.tld-access\_log combined \\
  ErrorLog /usr/local/apache/logs/server.domain.tld-error\_log\\
  SetEnv PYTHONPATH /usr/local/1Ci/dmerce/\dmercever/:\\
  Alias /1Ci/dmerce /usr/local/1Ci/dmerce/\dmercever/DTL/dmerce.pyc\\
  Alias /1Ci/ShowError /usr/local/1Ci/dmerce/\dmercever/Core/ShowError.pyc\\
  </VirtualHost>
}

\section{Das Filesystem-Layout von dmerce}

Damit dmerce einwandfrei laufen kann, sollten bestimme Parameter bei
der Auslegung der Filesysteme eingehalten werden.

\subsection{Templates}
\label{AdminGuideOrganisationTemplates}

\subsubsection{dmerce 1 und 2}

Im \$\{DOCUMENT\_ROOT\} des Webservers muss ein Verzeichnis
\wancicode{templates} existieren. Alle Templates, die von dmerce
angesprochen werden sollen, m\"ussen in diesem Verzeichniss liegen.
Die Templates k\"onnen hierbei in einer beliebigen Verzeichnisstruktur
organisiert werden.

Bitte achten Sie darauf, dass gen\"ugend Platz f\"ur Ihre Templates in
dem Filesystem ist, indem die dmerce Templates liegen. Die Verwendung
von Links, um Templates ggf. auszulagern funktioniert, wird aber nicht
empfohlen.

\subsection{Logfiles}

dmerce gibt Entwicklern eine Reihe von Informationen \"uber ein
Logfile. Die Informationen aus den Logfiles helfen, wenn Templates
nicht korrekt ausgef\"uhrt werden oder Aufrufe nicht korrekt
formuliert wurden. Dazu z\"ahlen Meldungen wie INFO, WARNING und
ERROR. \"Uber eine Standard-Fehlerseite gibt dmerce erfahrenen
Entwicklern meist eine ausreichende Information dar\"uber, was nicht
geklappt hat. Detailliertere Informationen k\"onnen aus den Logfiles
gewonnen werden.

Dazu z\"ahlen Informationen aus den Bereichen:

\begin{itemize}
\item Templates
\item Datenbankabfragen
\item Datenbankmanipulation
\item Trigger
\item Makros
\item Funktionsaufrufe
\item SAM
\item UXS
\end{itemize}

\subsubsection{dmerce 1}

dmerce Release 1 logt seine Informationen im Filesystem
\wancicode{/tmp} Ihres Servers. Alle Informationen aus jeder Website
zu jedem Zugriff werden hier je nach ihrem Typ in verschiedene Dateien
aufgeteilt.

Die Dateien:

\begin{itemize}
\item \wancicode{dmerce-1.2.6a.log} enth\"alt alle Informationen
  \"uber Templates, Datenbankabfragen, Trigger und Funktionen
\item \wancicode{dmerce-SAM.log} enth\"alt alle Informationen bezogen
  auf SAM
\item \wancicode{dmerce-SMTP.log} enth\"alt alle Informationen bezogen
  auf das Versenden von Emails
\end{itemize}


Format des Logs:\\

\begin{center}\wancicode{Datum YYYY-MM-DD HH:MM:SS Meldung}\end{center}

\subsubsection{dmerce 2}

Unter dmerce Relase 2 sind die Logfiles im Filesystem
\wancicode{/var/dmerce/log} organisiert. F\"ur jede Website, das Jahr
und den Monat werden hier Verzeichnisse angelegt.
Der Tag im Monat wird als Datei mit der Endung \wancicode{.log} angelegt.\\


Beispiel:\\

\begin{center}\wancicode{/var/dmerce/log/www.1ci.de/2001/12/23.log}\end{center}

ist das Logfile f\"ur die dmerce-Website www.1ci.de am 23.12.2001. In
diesem Logfile werden dann s\"amtliche Informationen aller Sybsysteme
gehalten.


Format des Logs:\\

\begin{center}\wancicode{Timestamp Client/Proxy<-Client (Template Subsystem
  Log-Level) Meldung}\end{center}
