%%  DGP API BEGINN %%
\chapter{1{[}DGP{]} - Damn Good Privacy: Verschl\"usselung}

dmerce 1{[}DGP{]} stellt Ihnen eine umfangreiche
Library zur Verschl\"usselung von Daten zur Verf\"ugung.
Siehe Kapitel \vref{KryptographieParanoia}

\newpage
\section{Referenz}

Die f\"ur \textbf{dmerce-DGP} ben\"otigten Module sind:

\begin{itemize}
\item \wancicode{mainlib.py}
\item \wancicode{cipher.py}
\item \wancicode{des.py}
\item \wancicode{deslib.py}
\item \wancicode{idea.py}
\item \wancicode{idealib.py}
\item \wancicode{rijndael.py}
\item \wancicode{rijndaellib.py}
\end{itemize}

Dabei sind bei den Dateien f\"ur DES, IDEA, bzw. Rijndael
nur diejenigen n\"otig, deren Algorithmus benutzt werden soll.
In diesem Sinne ist dmerce-DGP v\"ollig modular.

\subsection{Algorithmen}

\subsubsection{Symmetrische Blockalgorithmen}

\paragraph{DES}

DES ist ein symmetrischer Blockalgorithmus mit folgenden Daten:

\begin{description}

\item[Blockl\"ange] 64 Bit in 1 Wort

\item[Wortl\"ange] 64 Bit

\item[Schl\"ussell\"ange] 64 Bit (effektiv 56 Bit)

\item[Rundenzahl] 16

\end{description}

\paragraph{IDEA}

IDEA ist ein symmetrischer Blockalgorithmus mit folgenden Daten:

\begin{description}

\item[Blockl\"ange] 64 Bit in 4 Worten

\item[Wortl\"ange] 16 Bit

\item[Schl\"ussell\"ange] 128 Bit

\item[Rundenzahl] frei w\"ahlbar (\textit{default}: 8)

\end{description}
\textbf{IDEA darf nur f\"ur nicht gewerbliche Zwecke frei genutzt
werden, f\"ur kommerzielle Nutzung ist eine kostenpflichtige
Lizensierung notwendig!}

\paragraph{Rijndael/AES}

Rijndael ist ein symmetrischer Blockalgorithmus mit folgenden Daten:

\begin{description}

\item[Blockl\"ange] $n\times32$ Bit in $n$ Worten und $n$ frei w\"ahlbar aus:
$\{4, 5, 6, 7, 8\}$ (\textit{default}: $n = 4$)

\item[Wortl\"ange] 32 Bit

\item[Schl\"ussell\"ange] $n\times32$ Bit in $n$ Worten $n$ frei w\"ahlbar 
aus: $\{4, 5, 6, 7, 8\}$ (\textit{default}: $n = 4$)

\item[Rundenzahl] frei w\"ahlbar. \textit{default} und dringend empfohlen:
$$\mbox{max}(\mbox{Schl\"ussell\"ange, Blockl\"ange}) + 6$$ (also nicht 
kleiner als 10)
\end{description}

\textbf{AES} ist eine Spezialisierung des urspr\"unglichen 
Rijndael-Algorithmus in folgendem Sinne: \\
Zugelassen ist nur die Blockl\"ange von 128 Bit, und die 
Schl\"ussell\"angen 128 (AES oder AES-128), 192 (AES-192) und 
256 (AES-256) Bit.

\subsection{Syntax}

Die Kryptographischen Algorithmen sind als Klassen implementiert.
D.h., man instantiiert ein Objekt zum betreffenden Algorithmus
und fuehrt Ver- und Entschluesselung dann als Methoden dieses
Objektes aus.

Z.B. fuer den Algorithmus DES: 

\wancicode{plain = '<klartext>'\\
  key = '<rohschluessel>'\\
  obj = des.DES()} \\

\wancicode{p = main\_lib.String\_Fillup(plain, obj.Blocklen()) \\
  obj.NewData(p) \\

  obj.E\_Key\_Create(key) \\

  x = obj.ECB\_Encrypt() \\
  cipher = obj.OutFormat(x)} \\

Dabei ist plain der zu verschluesselnde Klartext als String,
key der Roh-Schluessel als eine ganze Zahl zwischen 0 und
der maximalen Schluessellaenge (die Daten der einzelnen Algorithmen s.u.)
und obj das DES-Objekt (im Standardbetrieb: weitere Optionen werden
weiter unten beschrieben).

Mit der Funktion String\_Fillup aus dem Modul main\_lib.py wird der
Klartext auf die volle Blocklaenge des Algorithmus gebracht und dann
mit der Methode NewData dem DES-Objekt als Eingabe zugefuehrt.

Die Methode E\_Key\_Create erzeugt aus dem Rohschluessel einen 
fertigen Rundenschluessel.

Die Methode ECB\_Encrypt verschluesselt den Klartext, OutFormat
formatiert den resultierenden Chiffretext in einen String.

Die Entschluesselung verlaueft fast parallel:

\wancicode{cipher = <chiffretext>\\
  key = <schluessel> \\

  obj = des.DES() \\
  obj.NewData(cipher) \\
  obj.D\_Key\_Create(key) \\
  y = obj.ECB\_Decrypt() \\
  x = obj.OutFormat(y) \\
  plain = main\_lib.String\_Filldown(x)} \\

mit dem Unterschied, dass der Chiffretext zunaechst entschluesselt,
dann formatiert und zum Schluss die Auffuellung rueckgaengig gemacht
wird. Das ergebnis ist der urspruengliche Klartext als String.


\subsection{Chiffrier-Modi}

\subsubsection{ECB (Electronic Code Book)}

Benutzung durch Aufruf der Ver-/Entschluesselungsmethode: \\
 
\texttt{<kryptoobjekt>.ECB\_Encrypt()} \\

\wancicode{<kryptoobjekt>.ECB\_Decrypt()} \\


\subsubsection{CBC (Cipher Block Chaining)}

CBC benoetigt einen Initialisierungsvektor (IV). Dieser wird vor 
der Ver-/ Entschluesselung initialisiert: \\

\wancicode{<kryptoobjekt>.IVinit(<IV>)} \\

Da ein IV einen zusaetzlichen Block darstellt, wird er als positive
ganze Zahl angegeben, wobei $0 \leq IV < 2^b$
ist ($b$ = Blockl\"ange). \\

Danach wird parallel zu ECB verfahren: \\

\wancicode{<kryptoobjekt>.CBC\_Encrypt()} \\

\wancicode{<kryptoobjekt>.CBC\_Decrypt()} \\
%%  DGP API ENDE %%
